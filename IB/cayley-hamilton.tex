\documentclass[a4paper]{article}

\def\ntitle{Determinant trick, Cayley-Hamilton Theorem and Nakayama's Lemma}
\def\ndate{}

\ifx \nauthor\undefined
  \def\nauthor{Qiangru Kuang}
\else
\fi

\ifx \ntitle\undefined
  \def\ntitle{Template}
\else
\fi

\ifx \nauthoremail\undefined
  \def\nauthoremail{qk206@cam.ac.uk}
\else
\fi

\ifx \ndate\undefined
  \def\ndate{\today}
\else
\fi

\title{\ntitle}
\author{\nauthor}
\date{\ndate}

%\usepackage{microtype}
\usepackage{mathtools}
\usepackage{amsthm}
\usepackage{stmaryrd}%symbols used so far: \mapsfrom
\usepackage{empheq}
\usepackage{amssymb}
\let\mathbbalt\mathbb
\let\pitchforkold\pitchfork
\usepackage{unicode-math}
\let\mathbb\mathbbalt%reset to original \mathbb
\let\pitchfork\pitchforkold

\usepackage{imakeidx}
\makeindex[intoc]

%to address the problem that Latin modern doesn't have unicode support for setminus
%https://tex.stackexchange.com/a/55205/26707
\AtBeginDocument{\renewcommand*{\setminus}{\mathbin{\backslash}}}
\AtBeginDocument{\renewcommand*{\models}{\vDash}}%for \vDash is same size as \vdash but orginal \models is larger
\AtBeginDocument{\let\Re\relax}
\AtBeginDocument{\let\Im\relax}
\AtBeginDocument{\DeclareMathOperator{\Re}{Re}}
\AtBeginDocument{\DeclareMathOperator{\Im}{Im}}
\AtBeginDocument{\let\div\relax}
\AtBeginDocument{\DeclareMathOperator{\div}{div}}

\usepackage{tikz}
\usetikzlibrary{automata,positioning}
\usepackage{pgfplots}
%some preset styles
\pgfplotsset{compat=1.15}
\pgfplotsset{centre/.append style={axis x line=middle, axis y line=middle, xlabel={$x$}, ylabel={$y$}, axis equal}}
\usepackage{tikz-cd}
\usepackage{graphicx}
\usepackage{newunicodechar}

\usepackage{fancyhdr}

\fancypagestyle{mypagestyle}{
    \fancyhf{}
    \lhead{\emph{\nouppercase{\leftmark}}}
    \rhead{}
    \cfoot{\thepage}
}
\pagestyle{mypagestyle}

\usepackage{titlesec}
\newcommand{\sectionbreak}{\clearpage} % clear page after each section
\usepackage[perpage]{footmisc}
\usepackage{blindtext}

%\reallywidehat
%https://tex.stackexchange.com/a/101136/26707
\usepackage{scalerel,stackengine}
\stackMath
\newcommand\reallywidehat[1]{%
\savestack{\tmpbox}{\stretchto{%
  \scaleto{%
    \scalerel*[\widthof{\ensuremath{#1}}]{\kern-.6pt\bigwedge\kern-.6pt}%
    {\rule[-\textheight/2]{1ex}{\textheight}}%WIDTH-LIMITED BIG WEDGE
  }{\textheight}% 
}{0.5ex}}%
\stackon[1pt]{#1}{\tmpbox}%
}

%\usepackage{braket}
\usepackage{thmtools}%restate theorem
\usepackage{hyperref}

% https://en.wikibooks.org/wiki/LaTeX/Hyperlinks
\hypersetup{
    %bookmarks=true,
    unicode=true,
    pdftitle={\ntitle},
    pdfauthor={\nauthor},
    pdfsubject={Mathematics},
    pdfcreator={\nauthor},
    pdfproducer={\nauthor},
    pdfkeywords={math maths \ntitle},
    colorlinks=true,
    linkcolor={red!50!black},
    citecolor={blue!50!black},
    urlcolor={blue!80!black}
}

\usepackage{cleveref}



% TODO: mdframed often gives bad breaks that cause empty lines. Would like to switch to tcolorbox.
% The current workaround is to set innerbottommargin=0pt.

%\usepackage[theorems]{tcolorbox}





\usepackage[framemethod=tikz]{mdframed}
\mdfdefinestyle{leftbar}{
  %nobreak=true, %dirty hack
  linewidth=1.5pt,
  linecolor=gray,
  hidealllines=true,
  leftline=true,
  leftmargin=0pt,
  innerleftmargin=5pt,
  innerrightmargin=10pt,
  innertopmargin=-5pt,
  % innerbottommargin=5pt, % original
  innerbottommargin=0pt, % temporary hack 
}
%\newmdtheoremenv[style=leftbar]{theorem}{Theorem}[section]
%\newmdtheoremenv[style=leftbar]{proposition}[theorem]{proposition}
%\newmdtheoremenv[style=leftbar]{lemma}[theorem]{Lemma}
%\newmdtheoremenv[style=leftbar]{corollary}[theorem]{corollary}

\newtheorem{theorem}{Theorem}[section]
\newtheorem{proposition}[theorem]{Proposition}
\newtheorem{lemma}[theorem]{Lemma}
\newtheorem{corollary}[theorem]{Corollary}
\newtheorem{axiom}[theorem]{Axiom}
\newtheorem*{axiom*}{Axiom}

\surroundwithmdframed[style=leftbar]{theorem}
\surroundwithmdframed[style=leftbar]{proposition}
\surroundwithmdframed[style=leftbar]{lemma}
\surroundwithmdframed[style=leftbar]{corollary}
\surroundwithmdframed[style=leftbar]{axiom}
\surroundwithmdframed[style=leftbar]{axiom*}

\theoremstyle{definition}

\newtheorem*{definition}{Definition}
\surroundwithmdframed[style=leftbar]{definition}

\newtheorem*{slogan}{Slogan}
\newtheorem*{eg}{Example}
\newtheorem*{ex}{Exercise}
\newtheorem*{remark}{Remark}
\newtheorem*{notation}{Notation}
\newtheorem*{convention}{Convention}
\newtheorem*{assumption}{Assumption}
\newtheorem*{question}{Question}
\newtheorem*{answer}{Answer}
\newtheorem*{note}{Note}
\newtheorem*{application}{Application}

%operator macros

%basic
\DeclareMathOperator{\lcm}{lcm}

%matrix
\DeclareMathOperator{\tr}{tr}
\DeclareMathOperator{\Tr}{Tr}
\DeclareMathOperator{\adj}{adj}

%algebra
\DeclareMathOperator{\Hom}{Hom}
\DeclareMathOperator{\End}{End}
\DeclareMathOperator{\id}{id}
\DeclareMathOperator{\im}{im}
\DeclarePairedDelimiter{\generation}{\langle}{\rangle}

%groups
\DeclareMathOperator{\sym}{Sym}
\DeclareMathOperator{\sgn}{sgn}
\DeclareMathOperator{\inn}{Inn}
\DeclareMathOperator{\aut}{Aut}
\DeclareMathOperator{\GL}{GL}
\DeclareMathOperator{\SL}{SL}
\DeclareMathOperator{\PGL}{PGL}
\DeclareMathOperator{\PSL}{PSL}
\DeclareMathOperator{\SU}{SU}
\DeclareMathOperator{\UU}{U}
\DeclareMathOperator{\SO}{SO}
\DeclareMathOperator{\OO}{O}
\DeclareMathOperator{\PSU}{PSU}

%hyperbolic
\DeclareMathOperator{\sech}{sech}

%field, galois heory
\DeclareMathOperator{\ch}{ch}
\DeclareMathOperator{\gal}{Gal}
\DeclareMathOperator{\emb}{Emb}



%ceiling and floor
%https://tex.stackexchange.com/a/118217/26707
\DeclarePairedDelimiter\ceil{\lceil}{\rceil}
\DeclarePairedDelimiter\floor{\lfloor}{\rfloor}


\DeclarePairedDelimiter{\innerproduct}{\langle}{\rangle}

%\DeclarePairedDelimiterX{\norm}[1]{\lVert}{\rVert}{#1}
\DeclarePairedDelimiter{\norm}{\lVert}{\rVert}



%Dirac notation
%TODO: rewrite for variable number of arguments
\DeclarePairedDelimiterX{\braket}[2]{\langle}{\rangle}{#1 \delimsize\vert #2}
\DeclarePairedDelimiterX{\braketthree}[3]{\langle}{\rangle}{#1 \delimsize\vert #2 \delimsize\vert #3}

\DeclarePairedDelimiter{\bra}{\langle}{\rvert}
\DeclarePairedDelimiter{\ket}{\lvert}{\rangle}




%macros

%general

%divide, not divide
\newcommand*{\divides}{\mid}
\newcommand*{\ndivides}{\nmid}
%vector, i.e. mathbf
%https://tex.stackexchange.com/a/45746/26707
\newcommand*{\V}[1]{{\ensuremath{\symbf{#1}}}}
%closure
\newcommand*{\cl}[1]{\overline{#1}}
%conjugate
\newcommand*{\conj}[1]{\overline{#1}}
%set complement
\newcommand*{\stcomp}[1]{\overline{#1}}
\newcommand*{\compose}{\circ}
\newcommand*{\nto}{\nrightarrow}
\newcommand*{\p}{\partial}
%embed
\newcommand*{\embed}{\hookrightarrow}
%surjection
\newcommand*{\surj}{\twoheadrightarrow}
%power set
\newcommand*{\powerset}{\mathcal{P}}

%matrix
\newcommand*{\matrixring}{\mathcal{M}}

%groups
\newcommand*{\normal}{\trianglelefteq}
%rings
\newcommand*{\ideal}{\trianglelefteq}

%fields
\renewcommand*{\C}{{\mathbb{C}}}
\newcommand*{\R}{{\mathbb{R}}}
\newcommand*{\Q}{{\mathbb{Q}}}
\newcommand*{\Z}{{\mathbb{Z}}}
\newcommand*{\N}{{\mathbb{N}}}
\newcommand*{\F}{{\mathbb{F}}}
%not really but I think this belongs here
\newcommand*{\A}{{\mathbb{A}}}

%asymptotic
\newcommand*{\bigO}{O}
\newcommand*{\smallo}{o}

%probability
\newcommand*{\prob}{\mathbb{P}}
\newcommand*{\E}{\mathbb{E}}

%vector calculus
\newcommand*{\gradient}{\V \nabla}
\newcommand*{\divergence}{\gradient \cdot}
\newcommand*{\curl}{\gradient \cdot}

%logic
\newcommand*{\yields}{\vdash}
\newcommand*{\nyields}{\nvdash}

%differential geometry
\renewcommand*{\H}{\mathbb{H}}
\newcommand*{\transversal}{\pitchfork}
\renewcommand{\d}{\mathrm{d}} % exterior derivative

%number theory
\newcommand*{\legendre}[2]{\genfrac{(}{)}{}{}{#1}{#2}}%Legendre symbol


\begin{document}

\maketitle

Let \(A\) be a commutative ring and \(M\) be an \(A\)-module generated by \(\{m_1, \dots, m_n\}\). Note that \(M\) is naturally an \(\End(M)\)-module and for all \(f \in \End(M)\), write \([f] \in \matrixring_n(A)\) for its representation with respect to the generators above, i.e.\ \(f(m_i) = \sum_j [f]_{ij}m_j\). In particular, there is a ring homomorphism \(\mu: A \to \End(M), a \mapsto a \cdot -\) sending an element to its multiplication action. Let \(A' = \mu(A)\).

There is a technical remark to make: later we will use determinant of matrices over \(\End(M)\), which is non-commutative. However, throughout the discussion we are concerned with only one endomorphism \(\varphi\) (besides multiplication, of course) so we can restrict the scalars to \(A'[\varphi]\), a subring contained in the centre of \(\End(M)\).

Given a module endomorphism \(\varphi: M \to M\), its characteristic polynomial is defined to be
\[
  \chi_{[\varphi]}(x) = \det (x \cdot I - [\varphi]) \in A[x]
\]
where \(I\) is the \(n \times n\) identity matrix and the product \(x \cdot I\) is multiplication of a matrix by a scalar. We have

\begin{theorem}[Cayley-Hamilton]
  \[
    \chi_{[\varphi]}(\varphi) = 0.
  \]
\end{theorem}
This is a slight generalisation of the result one might be familiar with from linear algebra. Note that this is a relation of endomorphisms with coefficients in \(A\).

\begin{proof}
  Let \([\varphi]_{ij} = a_{ij}\) and view \(M\) as an \(A'[\varphi]\)-module. Since
  \[
    \varphi m_i = \sum_j a_{ij}m_j,
  \]
  we have
  \begin{equation*}
    \label{eqn:*}
    \sum_j \underbrace{(\varphi \delta_{ij} - a_{ij})}_{\Delta_{ij}} m_j = 0
    \tag{\(*\)}
  \end{equation*}
  with
  \[
    \Delta = \varphi \cdot I - N \in \matrixring_n(A'[\varphi]).
  \]
  Again, the multiplication is by scalar \(\varphi\), viewed as an element of the ring \(\End(M)\).

  Claim that if \(\det \Delta = 0 \in \End(M)\) then we are done: consider the ring homomorphism
  \begin{align*}
    A[x] &\to \End(M) \\
    x &\mapsto \varphi
  \end{align*}
  which maps \(\chi_{[\varphi]}(t) \mapsto \chi_{[\varphi]}(\varphi) = \det \Delta\) since \(\det\) is a polynomial function. So done.

  To show this, recall that
  \[
    (\adj \Delta) \cdot \Delta = \det \Delta \cdot I \in \matrixring_n(A'[\varphi])
  \]
  where multiplication on the left is between matrices. Let \((\adj \Delta)_{ij} = b_{ij}\). Then multiply \eqref{eqn:*} by \(b_{ki}\) and apply the identity,
  \[
    \sum_{i, j} (b_{ki} \Delta_{ij}) m_j = \sum_j (\det \Delta \delta_{kj}) m_j = (\det \Delta) m_k = 0.
  \]
  so \(\det \Delta = 0\) as required.
\end{proof}

We extract the key idea in the proof, which some authors call the \emph{determinant trick}, which has many applications in commutative algebra:

\begin{theorem}
  Let \(M\) be an \(A\)-module generated by \(n\) elements and \(\varphi: M \to M\) a homomorphism. Suppose \(I\) is an ideal of \(A\) such that \(\varphi(M) \subseteq IM\), then there is a relation
  \[
    \varphi^n + a_1 \varphi^{n - 1} + \dots + a_{n - 1} \varphi + a_n = 0
  \]
  where \(a_i \in I^i\) for all \(i\).
\end{theorem}

\begin{proof}
  Let \(\{m_1, \dots, m_n\}\) be a set of generators of \(M\). Since \(\varphi(m_i) \in IM\), we can write
  \[
    \varphi m_i = \sum_j a_{ij}m_j
  \]
  with \(a_{ij} \in I\). Multiply
  \[
    \sum_j \underbrace{(\varphi \delta_{ij} - a_{ij})}_{\Delta_{ij}} m_j = 0 
  \]
  by \(\adj \Delta\), we deduce that \((\det \Delta) m_j = 0\) so \(\det \Delta = 0 \in \End(M)\). Expand.
\end{proof}

\begin{corollary}[Nakayama's Lemma]
  \label{cor:Nakayama}
  If \(M\) is a finitely generated \(A\)-module and \(I \ideal R\) is such that \(M = IM\) then there exists \(x \in A\) such that \(x - 1 \in I\) and \(xM = 0\).
\end{corollary}

\begin{proof}
  Apply the trick to \(\id_M\). Since \(\id_M^i = \id_M\) and \(a_n = a_n\id_M\), we get
  \[
    \left(1 + \sum_{i = 1}^n a_i\right) \id_M = 0.
  \]
\end{proof}

We use the result to prove a rather interesting fact about module homomorphism:

\begin{proposition}
  Let \(M\) be a finitely generated \(A\)-module. Then every surjective module homomorphism on \(M\) is also injective.
\end{proposition}

\begin{proof}
  Let \(\varphi: M \to M\) be surjective. Let \(M\) be an \(A'[\varphi]\) module and \(I = (\varphi) \ideal A'[\varphi]\). Then \(M = IM\) by surjectivity of \(\varphi\). Thus by \nameref{cor:Nakayama}, there exists \(x = 1 + \varphi\psi\), \(\psi \in A'[\varphi]\) such that \((1 + \varphi\psi)M = 0\), i.e.\ for all \(m \in M\), \((1 + \varphi\psi)m = 0\). It follows that \(\varphi^{-1} = -\psi\).
\end{proof}

\end{document}
