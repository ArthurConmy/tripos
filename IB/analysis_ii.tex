\documentclass[a4paper]{article}

\def\ntitle{Analysis II}
\def\ndate{Michaelmas, 2017 -- 2018}

\ifx \nauthor\undefined
  \def\nauthor{Qiangru Kuang}
\else
\fi

\ifx \ntitle\undefined
  \def\ntitle{Template}
\else
\fi

\ifx \nauthoremail\undefined
  \def\nauthoremail{qk206@cam.ac.uk}
\else
\fi

\ifx \ndate\undefined
  \def\ndate{\today}
\else
\fi

\title{\ntitle}
\author{\nauthor}
\date{\ndate}

%\usepackage{microtype}
\usepackage{mathtools}
\usepackage{amsthm}
\usepackage{stmaryrd}%symbols used so far: \mapsfrom
\usepackage{empheq}
\usepackage{amssymb}
\let\mathbbalt\mathbb
\let\pitchforkold\pitchfork
\usepackage{unicode-math}
\let\mathbb\mathbbalt%reset to original \mathbb
\let\pitchfork\pitchforkold

\usepackage{imakeidx}
\makeindex[intoc]

%to address the problem that Latin modern doesn't have unicode support for setminus
%https://tex.stackexchange.com/a/55205/26707
\AtBeginDocument{\renewcommand*{\setminus}{\mathbin{\backslash}}}
\AtBeginDocument{\renewcommand*{\models}{\vDash}}%for \vDash is same size as \vdash but orginal \models is larger
\AtBeginDocument{\let\Re\relax}
\AtBeginDocument{\let\Im\relax}
\AtBeginDocument{\DeclareMathOperator{\Re}{Re}}
\AtBeginDocument{\DeclareMathOperator{\Im}{Im}}
\AtBeginDocument{\let\div\relax}
\AtBeginDocument{\DeclareMathOperator{\div}{div}}

\usepackage{tikz}
\usetikzlibrary{automata,positioning}
\usepackage{pgfplots}
%some preset styles
\pgfplotsset{compat=1.15}
\pgfplotsset{centre/.append style={axis x line=middle, axis y line=middle, xlabel={$x$}, ylabel={$y$}, axis equal}}
\usepackage{tikz-cd}
\usepackage{graphicx}
\usepackage{newunicodechar}

\usepackage{fancyhdr}

\fancypagestyle{mypagestyle}{
    \fancyhf{}
    \lhead{\emph{\nouppercase{\leftmark}}}
    \rhead{}
    \cfoot{\thepage}
}
\pagestyle{mypagestyle}

\usepackage{titlesec}
\newcommand{\sectionbreak}{\clearpage} % clear page after each section
\usepackage[perpage]{footmisc}
\usepackage{blindtext}

%\reallywidehat
%https://tex.stackexchange.com/a/101136/26707
\usepackage{scalerel,stackengine}
\stackMath
\newcommand\reallywidehat[1]{%
\savestack{\tmpbox}{\stretchto{%
  \scaleto{%
    \scalerel*[\widthof{\ensuremath{#1}}]{\kern-.6pt\bigwedge\kern-.6pt}%
    {\rule[-\textheight/2]{1ex}{\textheight}}%WIDTH-LIMITED BIG WEDGE
  }{\textheight}% 
}{0.5ex}}%
\stackon[1pt]{#1}{\tmpbox}%
}

%\usepackage{braket}
\usepackage{thmtools}%restate theorem
\usepackage{hyperref}

% https://en.wikibooks.org/wiki/LaTeX/Hyperlinks
\hypersetup{
    %bookmarks=true,
    unicode=true,
    pdftitle={\ntitle},
    pdfauthor={\nauthor},
    pdfsubject={Mathematics},
    pdfcreator={\nauthor},
    pdfproducer={\nauthor},
    pdfkeywords={math maths \ntitle},
    colorlinks=true,
    linkcolor={red!50!black},
    citecolor={blue!50!black},
    urlcolor={blue!80!black}
}

\usepackage{cleveref}



% TODO: mdframed often gives bad breaks that cause empty lines. Would like to switch to tcolorbox.
% The current workaround is to set innerbottommargin=0pt.

%\usepackage[theorems]{tcolorbox}





\usepackage[framemethod=tikz]{mdframed}
\mdfdefinestyle{leftbar}{
  %nobreak=true, %dirty hack
  linewidth=1.5pt,
  linecolor=gray,
  hidealllines=true,
  leftline=true,
  leftmargin=0pt,
  innerleftmargin=5pt,
  innerrightmargin=10pt,
  innertopmargin=-5pt,
  % innerbottommargin=5pt, % original
  innerbottommargin=0pt, % temporary hack 
}
%\newmdtheoremenv[style=leftbar]{theorem}{Theorem}[section]
%\newmdtheoremenv[style=leftbar]{proposition}[theorem]{proposition}
%\newmdtheoremenv[style=leftbar]{lemma}[theorem]{Lemma}
%\newmdtheoremenv[style=leftbar]{corollary}[theorem]{corollary}

\newtheorem{theorem}{Theorem}[section]
\newtheorem{proposition}[theorem]{Proposition}
\newtheorem{lemma}[theorem]{Lemma}
\newtheorem{corollary}[theorem]{Corollary}
\newtheorem{axiom}[theorem]{Axiom}
\newtheorem*{axiom*}{Axiom}

\surroundwithmdframed[style=leftbar]{theorem}
\surroundwithmdframed[style=leftbar]{proposition}
\surroundwithmdframed[style=leftbar]{lemma}
\surroundwithmdframed[style=leftbar]{corollary}
\surroundwithmdframed[style=leftbar]{axiom}
\surroundwithmdframed[style=leftbar]{axiom*}

\theoremstyle{definition}

\newtheorem*{definition}{Definition}
\surroundwithmdframed[style=leftbar]{definition}

\newtheorem*{slogan}{Slogan}
\newtheorem*{eg}{Example}
\newtheorem*{ex}{Exercise}
\newtheorem*{remark}{Remark}
\newtheorem*{notation}{Notation}
\newtheorem*{convention}{Convention}
\newtheorem*{assumption}{Assumption}
\newtheorem*{question}{Question}
\newtheorem*{answer}{Answer}
\newtheorem*{note}{Note}
\newtheorem*{application}{Application}

%operator macros

%basic
\DeclareMathOperator{\lcm}{lcm}

%matrix
\DeclareMathOperator{\tr}{tr}
\DeclareMathOperator{\Tr}{Tr}
\DeclareMathOperator{\adj}{adj}

%algebra
\DeclareMathOperator{\Hom}{Hom}
\DeclareMathOperator{\End}{End}
\DeclareMathOperator{\id}{id}
\DeclareMathOperator{\im}{im}
\DeclarePairedDelimiter{\generation}{\langle}{\rangle}

%groups
\DeclareMathOperator{\sym}{Sym}
\DeclareMathOperator{\sgn}{sgn}
\DeclareMathOperator{\inn}{Inn}
\DeclareMathOperator{\aut}{Aut}
\DeclareMathOperator{\GL}{GL}
\DeclareMathOperator{\SL}{SL}
\DeclareMathOperator{\PGL}{PGL}
\DeclareMathOperator{\PSL}{PSL}
\DeclareMathOperator{\SU}{SU}
\DeclareMathOperator{\UU}{U}
\DeclareMathOperator{\SO}{SO}
\DeclareMathOperator{\OO}{O}
\DeclareMathOperator{\PSU}{PSU}

%hyperbolic
\DeclareMathOperator{\sech}{sech}

%field, galois heory
\DeclareMathOperator{\ch}{ch}
\DeclareMathOperator{\gal}{Gal}
\DeclareMathOperator{\emb}{Emb}



%ceiling and floor
%https://tex.stackexchange.com/a/118217/26707
\DeclarePairedDelimiter\ceil{\lceil}{\rceil}
\DeclarePairedDelimiter\floor{\lfloor}{\rfloor}


\DeclarePairedDelimiter{\innerproduct}{\langle}{\rangle}

%\DeclarePairedDelimiterX{\norm}[1]{\lVert}{\rVert}{#1}
\DeclarePairedDelimiter{\norm}{\lVert}{\rVert}



%Dirac notation
%TODO: rewrite for variable number of arguments
\DeclarePairedDelimiterX{\braket}[2]{\langle}{\rangle}{#1 \delimsize\vert #2}
\DeclarePairedDelimiterX{\braketthree}[3]{\langle}{\rangle}{#1 \delimsize\vert #2 \delimsize\vert #3}

\DeclarePairedDelimiter{\bra}{\langle}{\rvert}
\DeclarePairedDelimiter{\ket}{\lvert}{\rangle}




%macros

%general

%divide, not divide
\newcommand*{\divides}{\mid}
\newcommand*{\ndivides}{\nmid}
%vector, i.e. mathbf
%https://tex.stackexchange.com/a/45746/26707
\newcommand*{\V}[1]{{\ensuremath{\symbf{#1}}}}
%closure
\newcommand*{\cl}[1]{\overline{#1}}
%conjugate
\newcommand*{\conj}[1]{\overline{#1}}
%set complement
\newcommand*{\stcomp}[1]{\overline{#1}}
\newcommand*{\compose}{\circ}
\newcommand*{\nto}{\nrightarrow}
\newcommand*{\p}{\partial}
%embed
\newcommand*{\embed}{\hookrightarrow}
%surjection
\newcommand*{\surj}{\twoheadrightarrow}
%power set
\newcommand*{\powerset}{\mathcal{P}}

%matrix
\newcommand*{\matrixring}{\mathcal{M}}

%groups
\newcommand*{\normal}{\trianglelefteq}
%rings
\newcommand*{\ideal}{\trianglelefteq}

%fields
\renewcommand*{\C}{{\mathbb{C}}}
\newcommand*{\R}{{\mathbb{R}}}
\newcommand*{\Q}{{\mathbb{Q}}}
\newcommand*{\Z}{{\mathbb{Z}}}
\newcommand*{\N}{{\mathbb{N}}}
\newcommand*{\F}{{\mathbb{F}}}
%not really but I think this belongs here
\newcommand*{\A}{{\mathbb{A}}}

%asymptotic
\newcommand*{\bigO}{O}
\newcommand*{\smallo}{o}

%probability
\newcommand*{\prob}{\mathbb{P}}
\newcommand*{\E}{\mathbb{E}}

%vector calculus
\newcommand*{\gradient}{\V \nabla}
\newcommand*{\divergence}{\gradient \cdot}
\newcommand*{\curl}{\gradient \cdot}

%logic
\newcommand*{\yields}{\vdash}
\newcommand*{\nyields}{\nvdash}

%differential geometry
\renewcommand*{\H}{\mathbb{H}}
\newcommand*{\transversal}{\pitchfork}
\renewcommand{\d}{\mathrm{d}} % exterior derivative

%number theory
\newcommand*{\legendre}[2]{\genfrac{(}{)}{}{}{#1}{#2}}%Legendre symbol


\theoremstyle{definition}
\newtheorem*{joke}{Joke}

\begin{document}
% TODO: rewrite \norm




\maketitle

\tableofcontents

\setcounter{section}{-1}

\section{Introduction}

In Analysis I, the primary space we are interested in is \(\R\) and we studied notions such as continuity, convergence, differentiation, integration and solving equaiton through, for example, Intermediate Value Theorem. In Analysis II, we moved to the study general function space.

\begin{table}[htbp]
  \centering
  \begin{tabular}{|c|p{4cm}|p{4cm}|}
    \hline
    & $\mathbb{R}^m$ & Function space \\ \hline
    Continuity \& convergence & $\checkmark$ & $\checkmark$ \\ \hline
    Differentiation & $\checkmark$ & Calculus of variations \\ \hline
    Integration & Probability and measure & ??? (ask physicists) \\ \hline
    Solving equations & inverse function theorem & existence of solutions for ODEs \\ \hline
  \end{tabular}
  \caption{Comparison of Euclidean space and function space}
\end{table}

\section{Normed Vector Spaces}

A motivating example: if $(a_n)$ is a sequence of real numbers, then $(a_n)\to a$ if
\[
  \forall \varepsilon>0,\exists N s.t. \forall n>N, |a_n-a|<\varepsilon.
\]
Now if I replace $\mathbb{R}$ by a real vector space $V$, what do I replace $|\cdot|$ with?

\begin{defi}
  If $V$ is a real vector space, a \emph{norm} on $V$ is a function $\|\cdot\|:V\to\mathbb{R}$ satisfying
  \begin{enumerate}
  \item $\forall \V v \in V, \|\V v\| \geq 0$ with equality if and only if $\V v =\V 0$.
  \item $\forall \V v,\forall \lambda \in \R \|\lambda \V v\| = |\lambda| \|\V v\|$
    \item $\forall \V v,\V w\in V, \|\V v+\V w\| \leq \|\V v\| + \|\V w\|$ (triangle inequality).
  \end{enumerate}
\end{defi}

\begin{eg}\leavevmode
  \begin{enumerate}
  \item $V= \mathbb{R}^m, \V v = (v_1,\ldots,v_m)$,
    \begin{enumerate}
    \item $\|\V v\| = (\sum_{i=1}^m v_i^2)^{1/2}$, the Euclidean norm,
    \item $\|\V v\|_\infty = \max |v_i|$, the max norm,
      \item $\|\V v\|_1 = \sum_{i=1}^m |v_i|$.
    \end{enumerate}
  \item $V=C[0,1]$,
    \begin{enumerate}
    \item $\|f\|_\infty=\max_{x\in[0,1]} |f(x)|$,
    \item $\|f\|_2=(\int_0^1 f(x)^2 dx)^{1/2}$, which comes from $\langle f,g\rangle = \int_0^1f(x)g(x) dx$,
      \item $\|f\|_1=\int_0^1|f(x)| dx$, the $L^1$ norm.
    \end{enumerate}
  \end{enumerate}
\end{eg}

\begin{defi}
  Suppose $(V, \|\cdot\|)$ is a normed vector space and $(\V v_n)$ is a sequence of elements of $V$. We say $(\V v_n)$ converges to $\V v\in V$, denoted $(\V v_n)\to \V v$, if $\forall\varepsilon>0,\exists N s.t. \forall n>N, \|\V v_n-\V v\|<\varepsilon$. Equivalently, $(\V v_n)\to \V v$ if $(\|\V v_n-\V v\|)\to 0$.
\end{defi}

\begin{ex}
  Suppose $V=\mathbb{R}^m, (\V v_n) = (v_{n,1},\ldots,v_{n,m})$. Then $(\V v_n)\to \V v$ w.r.t. $\|\cdot\|_\infty$ means
  \begin{align*}
    & (\max_{1\leq i \leq m}|v_{n,i}-v_i|) \to 0 \\
    \Longleftrightarrow & (|v_{n,i}-v_i|)\to 0 \text{ for all } 1\leq i\leq m \\
    \Longleftrightarrow & (v_{n,i})\to v_i \text{ for all } 1\leq i\leq m
  \end{align*}

   The convergence w.r.t. $\|\cdot\|_1$ means
   \begin{align*}
     & (\sum_{i=1}^m |v_{n,i}-v_i|)\to 0 \\
     \Longleftrightarrow & (|v_{n,i}-v_i)\to 0 \text{ for all } 1\leq i\leq m \\ 
    \Longleftrightarrow & (v_{n,i})\to v_i \text{ for all } 1\leq i\leq m.
  \end{align*}
\end{ex}

\begin{rmk}
  Two different norms, same notion of convergence.
\end{rmk}

\begin{convention}
  If I say $(\V v_n)\to \V v$ where $(\V v_n)$ is a sequence in $\mathbb{R}^m$ without specifying the norm, I mean w.r.t. $\|\cdot\|_1,\|\cdot\|_2,\|\cdot\|_\infty$. (all give the same notion for convergence).
\end{convention}

\begin{eg}\label{eg:fdown}
  $V=C[0,1]$,
  \[
    f_n(x) =
    \begin{cases}
      1-nx & x\in[0,1/n] \\
      0 & x\geq 1/n
    \end{cases}
  \]
  Then $\|f_n\|_1=\int_0^1|f_n(x)| dx=\frac{1}{2n}\to 0$ so $(f_n)\to 0$ w.r.t. $\|\cdot\|_1$.

  But $\|f_n\|_\infty=1$ so $(\|f_n\|_\infty) \nto 0$ i.e.\ $(f_n) \nto 0$ w.r.t. $\|\cdot\|_\infty$.
\end{eg}

\begin{rmk}
  Two different norms, two different notions of convergence.
\end{rmk}

\subsection{Continuity}

\begin{defi}
  Suppose $V$ and $W$ are normed vector spaces. We say a function $f:V\to W$ is \emph{continuous} if
  \[
    (f(\V v_n))\to f(\V v) \text{ in } W \text{ whenever } (\V v_n)\to \V v \text{ in } V.
  \]
  
\end{defi}

\begin{eg}\leavevmode
  \label{eg:continuity}
  \begin{enumerate}
  \item $f:V\to \mathbb{R}^m, f(\V v) = (f_1(\V v), \ldots, f_m(\V v))$ is continuous if and only if $f_i:V\to \mathbb{R}$ is continuous for all $1\leq i \leq m$.
  \item $\rho_i:\mathbb{R}^m\to \mathbb{R}, \rho_i(\V v) =v_i$ is continuous.
  \item $F:C[0,1]\to\mathbb{R}, F(f)=f(0)$,
    \begin{enumerate}
    \item If $(f_n)$ is the sequence from example~\ref{eg:fdown}, then $F(f_n)=1$. Now $(f_n)\to \V 0$ w.r.t. $\|\cdot\|_1$. But $(F(f_n))\nrightarrow 0=F(\V 0)$. So $F$ is not continuous w.r.t. $\|\cdot\|_1$.
      \item If $(g_n)\to g$ w.r.t. $\|\cdot\|_\infty$, then $(\max|g_n(x)-g(x)|)\to 0$ so $(|g_n(0)-g(0))|\to 0$, $(|F(g_n)-F(g)|)\to 0$, so $F(g_n)\to F(g)$.
    \end{enumerate}
    $F$ is continuous w.r.t. $\|\cdot\|_\infty$ but not w.r.t. $\|\cdot\|_1$.
  \item If $f:V_1\to V_2$ and $g:V_2\to V_3$ are continuous then $g\compose f: V_1\to V_3$ are continuous. Proof: if $(\V v_n) \to (\V v)$ in $V_1$ , then as $f$ is continuous, $(f(\V v_n)) \to (f(\V v))$, then as $g$ is continuous, $(g(f(\V v_n))) \to (g(f(\V v)))$ in $V_3$.
  \item $\|\cdot\|: V \to \R$ is continuous. Proof: if $(\V v_n)\to \V v$, then $(\|\V v_n - \V v\|) \to 0$. Now
    \[
      0 \leq |\|\V v_n\| - \|\V v\|| \leq \|\V v_n - \V v\|
    \]
    by Lemma~\ref{lem:alternate triangle}. So $(|\|\V v_n - \V v\||) \to 0$ by squeeze rule, i.e.\ $\|\V v_n\| \to \|\V v\|$.
  \end{enumerate}
\end{eg}

\begin{lem}
  \label{lem:alternate triangle}
  $\|\V v- \V w\| \geq |\|\V v\| - \|\V w\||$ for all $\V v, \V w \in V$.
\end{lem}

\begin{proof}
  By triangle inequality, $\|\V v- \V w\| - \|\V w\| \geq \|\V v \|$ so $\|\V v - \V w\| \geq \|\V w - \V v\| \geq \|\V w - \V v\|$.
\end{proof}

More generally, if $X\subset V$ is a subset, we say $f:X\to W$ is continuous if
\[
  (f(\V x_n)) \to f(\V x)
\]
in $W$ whenever $(\V x_n) \to \V x$ in $V$ for $\V x$ and all $\V x_n \in X$.

\begin{eg}
  $f:\R \setminus \{0\} \to \R, x \mapsto \frac{1}{x}$ is continuous.
\end{eg}

\subsection{Open and Closed Subsets}

Let $(V, \|\cdot\|)$ be a normed vector space.

\begin{defi}
  If $\V v_0 \in V$ and $r \in \R$,
  \[
    B_r(\V v_0) = \{\V v\in V: \|\V v - \V v_0\| < r \}
  \]
  is the \emph{open ball} of radius $r$ centred at $\V v_0$, and
  \[
    \cl B_r(\V v_0) = \{\V v\in V: \|\V v - \V v_0\| < r \}
  \]
  is the \emph{closed ball} of radius $r$ centred at $\V v_0$.
\end{defi}

\begin{eg}\leavevmode
  \begin{enumerate}
  \item $(V, \|\cdot\|) = (\R, |\cdot|)$, then
    \begin{align*}
      B_r(a) &= (a-r, a+r) \\
      \cl B_r(a) &= [a-r, a+r] 
    \end{align*}
  \item $V=\R^2$, then $\cl B_1(\V 0)$ with respect to \texttt{to be filled in}.
\item $V = \R^3, \|\cdot\|_2$ is the ``three-dimensional ball''.
\item $(V, \|\cdot\|) = (C[0,1], \|\cdot\|_\infty)$,
  \[
    \cl B_r(f) = \{g\in C[0,1]: f(x)-r \leq g(x) \leq f(x) + r \: \forall x \in [0,1]\}.
  \]
  \end{enumerate}
\end{eg}

\begin{prop}[Alternate characterisation of Continuity]
  $f:V\to W$ is continuous if and only if
  \begin{equation}
    \label{eqn:alternate continuity}
    \forall \V v_0 \in V, \forall \varepsilon>0, \exists \delta>0 s.t. \|\V v- \V v_0\| \leq \delta \Rightarrow \|f(\V v) - f(\V v_0)\| \tag{$\ast$}
    \end{equation}
  i.e.
  \[
    f(B_\delta(\V v_0)) \subset B_\varepsilon(f(\V v_0)).
    \]
\end{prop}

\begin{proof}
  Suppose \eqref{eqn:alternate continuity} holds. Given $(v_n)\to v$, must show $(f(v_n))\to f(v)$. Given $\varepsilon>0$, pick $\delta$ such that $(f(B_\delta(v))\subseteq B_\epsilon(f(v))$. Since $(v_n)\to (v)$, exists $N$ such that whenever $n>N$, $\|v_n - v\| < \delta$, i.e.\ $v_n \in B_\delta(v)$, so $f(v_n) \in B_\epsilon(f(v))$. In other words, whenever $n>N$, $\|f(v_n)-f(v)\| <\varepsilon$.

  Suppose \eqref{eqn:alternate continuity} does not hold. Then exists some $v \in V$ and $\varepsilon>0$ such that there is no $\delta>0$ with $f(B_\delta(v)) \subseteq B_\varepsilon(f(v))$. In particular $f(B_{1/n}(v)) \nsubseteq B_\varepsilon(f(v))$ for all $n$. Pick $v_n \in B_{1/n}(v)$ with $f(v_n) \notin B_\varepsilon(f(v))$. Then $(v_n)\to v$, but $(f(v_n)) \nto f(v)$, since $\|f(v_n) - f(v) \| \geq \varepsilon$ for all $n$. $f$ is not continuous.
\end{proof}

\begin{defi}
  $U \subseteq V$ is an \emph{open subset} of $V$ if for every $u \in U$ there is some $\varepsilon>0$ with $B_\varepsilon(u) \subseteq U$.
\end{defi}

\begin{prop}
  If $f:V\to W$ is continuous and $U\subseteq W$ is open then
  \[
    f^{-1}(U) = \{v \in V: f(v) \in V\}
  \]
  is an open subset of $V$.
\end{prop}

\begin{proof}
  If $v\in f^{-1}(U)$, then $f(v) \in U$. $U$ is open in $W$ so exists $\varepsilon>0$ such that $B_\varepsilon(f(v)) \subseteq U$. $F$ is continuous so exist $\delta>0$ such that $f(B_\delta(v)) \subseteq B_\varepsilon(f(v)) \subseteq U$. So $B_\delta(v) \subseteq f^{-1}(U)$. $f^{-1}(U)$ is open.
\end{proof}

\begin{rmk}
  The converse statement is also true: if for any $U \subseteq W$ open $f^{-1}(U)$ open in $V$, then $f$ is continuous.
\end{rmk}

\begin{eg}\leavevmode
  \begin{enumerate}
  \item $(0,1)$ is open in $\R$.
    \item The function $h(v) = \|v-v_0\|$ is continuous: $h(v) = g \compose f(v)$ where $f(v) = v-v_0$, and $g(v)=\|v\|$. so $B_r(r) = h^{-1}((-r,r))$ is open in $V$.
  \end{enumerate}
\end{eg}

\begin{defi}
  $C\subseteq V$ is a \emph{closed subset} of $V$ if $V\setminus C$ is open in $V$.
\end{defi}

\begin{cor}
  If $f:V\to W$ is continuous and $C\subseteq W$ is closed, then $f^{-1}(C)$ is closed in $V$.
\end{cor}

\begin{proof}
  \[
    f^{-1}(W\setminus C) = V\setminus f^{-1}(C)
  \]
  so if $C\subseteq W$ is closed, $W\setminus C$ is open, so $f^{-1}(W\setminus C) = V\setminus f^{-1}(C)$ is open. Thus $f^{-1}(C)\subseteq V$ is closed.
\end{proof}

\begin{eg}\leavevmode
  \begin{enumerate}
  \item $[a,b]$ is closed in $\R$.
  \item $h(v) = \|v-v_0\|, \cl B_r(v_0) = h^{-1}([0,r])$.
  \item $V, \emptyset$ are both open and closed in $V$.
    \item $\Q \subseteq \R$ is neither open nor closed.
  \end{enumerate}
\end{eg}

\begin{prop}
  \(C \subseteq V\) is closed if and only if for every sequence \((v_n) \to v\) with all \(v_n \in C\), \(v\in C\).
\end{prop}

\begin{proof}
  Suppose \(C\) is closed and \((v_n)\to v \notin C\). Then \(v\in V\setminus C\) which is open, so \(\exists \varepsilon>0\) with \(B_\varepsilon(v) \subseteq V\setminus C\), i.e.\ \(B_\varepsilon(v) \cap C =\emptyset\). \((v_n)\to v\) so \(\exists N\) such that \(v_n\in B_\varepsilon(v)\) for all \(n>N\). Thus \(v_n \notin C\) for all \(n>N\). In other word, if \((v_n)\to v\), all but finitely many of \(v_n \notin C\).

  Conversely, suppose \(C\) is not closed. Then \(V\setminus C\) is not open so \(\exists c \in V\setminus C\) such that there is no \(\varepsilon>0\) with \(B_\varepsilon(v) \subseteq V\setminus C\). In other words, \(B_\varepsilon(v) \cap V \neq \emptyset\) for all \(\varepsilon>0\). Pick \(v_n\in B_{1/n}(v)\cap C\) for all \(n>0\). Then \(\|v_n-v\|< 1/n\) so \((v_n)\to v\). All \(v_n \in C\) but \(v\in V\setminus C\).
\end{proof}

\begin{eg}
  The set \(X=\{f\in C[0,1]: \forall x, f(x)>0\}\) is not closed with respect to \(\|\cdot\|_1\) or \(\|\cdot\|_\infty\) since \(f_n(x) = \frac{1}{n} \in X\), \((f_n) \to 0\) with respect to either norm but \(0 \notin X\).
\end{eg}

For future use, suppose \(U_\alpha \subseteq V, \alpha\in A\) are open subsets of \(V\). Given \(U = \bigcup_{\alpha\in A}U_\alpha\) and \(f: U \to W\),

\begin{prop}
  If \(f|_{U_\alpha}:U_\alpha \to W\) is continuous for all \(\alpha \in A\), then \(f: U \to W\) is continuous.
\end{prop}

\begin{note}
  The hypothesis that \(U_\alpha\) is open is important. For example, let \(f:\R\to\R, f(x) = 1 \text{ if } x\in\Q, f(x)=0\) otherwise, then \(f|_\Q\) and \(f|_{\R\setminus\Q}\) are both continuous but \(f\) is not.
\end{note}

\begin{proof}
  Suppose \(v_n, v\in U\) with \((v_n)\to v\). Must show \((f(v_n))\to f(v)\). \(v\in U\) so \(v\in U_\alpha\) for some \(\alpha\). \(U_\alpha\) is open so \(\exists \varepsilon >0\) with \(B_\varepsilon(v) \subseteq U_\alpha\), \((v_n)\to v\) so \(\exists N\) with \(v_n \in B_\varepsilon(v)\) for all \(n>N\). Let \(u_i=v_{N+1}\), then \(u_i\in U_\alpha\) and \((u_i)\to v\). Since \(f|_{U_\alpha}\) is continuous, \((f(u_i))\to f(v)\) which implies that \((f(v_n)) \to f(v)\).
\end{proof}

\subsection{Lipschitz Equivalence}

Recall from the introduction of norms that \(\|\cdot\|_1, \|\cdot\|_2, \|\cdot\|_\infty\) on \(\R^n\) all induce the same notion of convergence. We want to generalise this idea.

Suppose \(\|\cdot\|\) and \(\|\cdot\|'\) are two norms on \(V\). Consider
\begin{align*}
  \id_V:(V,\|\cdot\|) &\to (V,\|\cdot\|') \\
  v &\mapsto v
\end{align*}

\begin{prop}
  \label{prop:continuity of id}
  \(\id_V\) as above is continuous if and only if \(\exists C \in \R\) with \(\|v\|' \leq C \|v\|\) for all \(v \in V\).
\end{prop}

\begin{proof}
  Suppose \(\|v\|' \leq C \|v\|\) for all \(v\). To show \(\id_V\) is continuous, must show \((v_n)\to v\) with respect to \(\|\cdot\|'\) whenever \((v_n)\to v\) with respect to \(\|\cdot\|\).

  If \((v_n)\to v\) with respect ot \(\|\cdot\|\), then \((\|v_n-v\|)\to 0\) so \((C\|v_n-v\|)\to 0\). We know
  \[
    0 \leq \|v_n-v\|' \leq C \|v_n-v\|,
  \]
  so by squeeze rule \(\|v_n-v\|\to 0\). Thus \((v_n)\to v\) with respect to \(\|\cdot\|'\).

  Conversely, suppose \(\id_V\) is continuous. There exist \(\delta>0\) such that
  \[
    \id_V(B_\delta(0, \|\cdot\|) \subseteq B_1(0, \|\cdot\|')
  \]
  Given \(v \in V, v \neq 0\), exists \(K\in\R\) with \(\|Kv\| = \delta/2\) (take \(K=\frac{\delta}{2 \|v\|}\)). Then \(Kv \in B_\delta(0,\|\cdot\|) \Rightarrow Kv\in B_1(0,\|\cdot\|')\), i.e.\ \(\|Kv\| = \delta/2, \|Kv\|' < 1\), so \(\|Kv\|' \leq \frac{2}{\delta}\|Kv\| \Rightarrow K \|v\|' \leq \frac{2}{\delta}K \|v\|\). Let \(C = \frac{2}{\delta}\).
\end{proof}

\begin{joke}
  \texttt{the joke about a mathematician going for a firefighter interview... Well you should know it by now if you are a mathematician.}
\end{joke}

\begin{defi}
  \label{def:Lipschitz equivalence}
  If \(\|\cdot\|\) and \(\|\cdot\|'\) are two norms on \(V\), they are said to be \emph{Lipschitz equivalent} if
  \begin{align*}
    & \exists C>0 s.t. \forall v \in V, \frac{1}{C}\|v\| \leq \|v\|' \leq C \|v\| \\
    \Longleftrightarrow & \exists C_1,C_2 s.t. \|v\|' \leq C_1 \|v\|, \|v\| \leq C_2 \|v\|' \\
    \Longleftrightarrow & \id_V:(V,\|\cdot\|)\to (V,\|\cdot\|') \text{ and } \id_V:(V,\|\cdot\|')\to (V,\|\cdot\|) \\
    & \text{are both continuous.}
  \end{align*}
\end{defi}

\begin{cor}
  If \((V, \|\cdot\|)\) and \(V, \|\cdot\|')\) are Lipschitz equivalent, then
  \begin{enumerate}
  \item \((v_n)\to v\) with respect to \(\|\cdot\|\) if and only if with respect to \(\|\cdot\|'\),
  \item \(f:V\to W\) is continuous with respect to \(\|\cdot\|\) if and only if with respect to \(\|\cdot\|'\),
  \item \(F:W\to V\) is continuous with respect to \(\|\cdot\|\) if and only if with respect to \(\|\cdot\|'\).
  \end{enumerate}
\end{cor}

\begin{proof}
  Example proof: if \(f:(V, \|\cdot\|)\to W\) is continuous, \(f:(V,\|\cdot\|')\to W = f:(V, \|\cdot\|)\to W \compose \id_V:(V, \|\cdot\|) \to (V, \|\cdot\|')\).
\end{proof}

\begin{eg}\leavevmode
  \begin{enumerate}
  \item \(V=\R^n\), \(\|v\|_\infty \leq \|v\|_2 \leq \|v\|_1 \leq n \|v\|_\infty\) so all three are Lipschitz equivalent.
  \item \(V=C[0,1]\), \(\id_V:(V, \|\cdot\|_\infty)\to (V,\|\cdot\|_1)\) is continuous but \(\id_V:(V, \|\cdot\|_1)\to (V,\|\cdot\|_\infty)\) is not so not Lipschitz equivalent.
  \end{enumerate}
\end{eg}

\section{Uniform Convergence}

\subsection{Notions of Convergence}

Suppose \(A\subseteq \R, f, f_n: A\to \R\). We say \(f\) is \emph{continuous} if given \(x\in A\) and \(\varepsilon>0\), \(\exists \delta>0\) such that \(|f(x)-f(y)| < \varepsilon\) whenever \(y\in A\) and \(|x-y| < \delta\). We say \(f\) is \emph{bounded} if exists \(M\) such that \(|f(x)| \leq M\) for all \(x\in A\). Define
\begin{itemize}
\item \(C(A) = \{f: A\to \R: f \text{ is continuous}\}\),
\item \(B(A) = \{f:A\to \R: f \text{ is bounded}\}\),
\end{itemize}
which are both vector space.

\begin{eg}
  \(C[0,1] \subseteq B[0,1]\) by Maximum Value Theorem. \(g(x) = \frac{1}{x} \in C(0,1]\) so \(C(0,1] \nsubseteq B(0,1]\).
\end{eg}

\begin{defi}[Pointwise Convergence]
  \((f_n) \to f\) \emph{pointwise} if
  \[
    (f_n(x)) \to f(x) \text{ for all } x\in \R.
    \]
\end{defi}

\begin{defi}
  The \emph{uniform norm} on \(B(A)\) is given by
  \[
    \|f\|_\infty = \sup_{x\in A} |f(x)|.
  \]
\end{defi}

\begin{defi}[Uniform Convergence]
  If \(f, f_n: A\to \R\), we say \(f(x) \to f\) \emph{uniformly on \(A\)} if \((f_n-f) \in B(A)\) for all \(n\) and \((\|f_n-f\|_\infty)\to0\).
\end{defi}
    In other words,
    \begin{itemize}
    \item \((f_n)\to f\) pointwise means: you give me \(x\in A\) and \(\varepsilon >0\), I have to find \(N\) such that \(|f_n(x) - f(x)| < \varepsilon\) whenever \(n> N\). \emph{This \(N\) only has to work for that particular value of \(x\).}
    \item \((f_n)\to f\) uniformly means: you give me \(\varepsilon>0\), I have to find \(N\) such that \(|f_n(x) - f(x)|<\varepsilon\) for all \(x\in A\) and \(n>N\). \emph{Same \(N\) works for all \(x\in A\).}
\end{itemize}

\begin{ex}
  If \((f_n)\to f \) uniformly, then \((f_n)\to f\) pointwise. The converse is false.
\end{ex}

\begin{eg}\leavevmode
  \begin{itemize}
  \item Suppose \(A=\R,f_n(x)=x+\frac{1}{n},f(x)=x\). Then \(f_n(x)-f(x)=\frac{1}{n}\) so \((f_n)\to f\) uniformly.
  \item Let \(g_n(x) = (x+\frac{1}{n})^2, g(x) = x^2\). Then \((g_n)\to g\) pointwise but \(g_n(x) - g(x) = \frac{2x}{n} + \frac{1}{n^2}\) is not even bounded. So \((g_n)\nto g\) uniformly on \(\R\).
  \end{itemize}
\end{eg}

\begin{thm}
  Suppose \(f_n \in C(A)\) for all \(n\) and \((f_n)\to f\) uniformly on \(A\). Then \(f\in C(A)\) as well.
\end{thm}

\begin{slogan}
  The uniform limit of continuous functions is continuous.
\end{slogan}

\begin{proof}
  Suppose \(f_n\) are continuous and \((f_n)\to f\) uniformly. Given \(x\in A\) and \(\varepsilon>0\), must find \(\delta>0\) such that \(|f(x)-f(y)| < \varepsilon\) whenever \(y\in A\) and \(|x-y|<\delta\).

  Since \((f_n)\to f\) uniformly, there exists \(N\) such that \(|f_n(x)-f(x)|< \varepsilon/4\) for all \(x\in A\) and \(n\geq N\). Since \(f_N\) is continuous, exists \(\delta>0\) such that \(|f_N(x)-f_n(y)|<\varepsilon/2\) whenever \(y\in A\) and \(|x-y|<\delta\). Then if \(|x-y| < \delta\),
  \begin{align*}
    |f(x)-f(y)| &\leq |f(x)-f_N(x)| + |f_N(x)-f_N(y)| + |f_N(y)-f(y)| \\
    &\leq \varepsilon/4 + \varepsilon/2 + \varepsilon/4 \\
    &= \varepsilon
  \end{align*}
\end{proof}

\begin{eg}
  Take \(A=[0,1]\),
  \begin{itemize}
  \item \(f_n(x)=x^n, f(x) = 1\text{ if } x=1, f(x)=0 \text{ if } x\neq 1\). Then \((f_n)\to f\) pointwise on \([0,1]\) but \(f_n\in C[0,1], f\notin C[0,1]\) so \((f_n)\nto f\) uniformly on \([0,1]\).
  \item \(g_n(x) = x^n(1-x), g(x) =0\). Then \((g_n)\to g\) uniformly.
\begin{proof}
  given \(\varepsilon>0, 1-\varepsilon<1\) so \((1-\varepsilon)^n\to 0\). Pick \(N\) such that \((1-\varepsilon)^n < \varepsilon\) for all \(n> N\). Then \(|f_n(x)| = |(1-x)x^n| \leq 1\cdot(1-\varepsilon)^n < \varepsilon\) for \(x\in[0,1-\varepsilon]\) and \(|f_n(x)| = |(1-x)x^n| < \varepsilon\cdot1^n=\varepsilon\) for \(x\in (1-\varepsilon,1]\). Thus \(|f_n(x)| < \varepsilon\) for all \(x\in[0,1]\).
\end{proof}
  \end{itemize}
\end{eg}

\begin{rmk}
  Everything I have said so far works fine for \(A\subseteq V, f: A\to W\), where \(V, W\) are normed vector spaces.
\end{rmk}

\begin{joke}
  A mathematician named Cliff measured his room for painting. His wife went off to the paint store and told the counter how much paint she needed. The counter said: ``Thats a lot of paint. Are you sure you want that much?'' To which the wife answered: ``Well my husband is a mathematician. I'm sure he gets the numbers correct.''

  She arrived back home with really a lot of paint. Cliff moved all the paint in the house and suddenly said:

  ``Oh, damn! I measured the volumn instead of the area!''
  \end{joke}

Recall that if \(f\in C[a,b]\) then \(\|f\|_1=\int_a^b|f(x)|dx\). 
\begin{defi}
  \(f_n\) converges \emph{in measure} to \(f\) if \((f_n)\to f\) with respect to \(\|\cdot\|_1\), 
\end{defi}

\begin{lem}
  If \((f_n)\in C[a,b]\) and \((f_n)\to f\) uniformly then \((f_n)\to f\) in measure.
\end{lem}

\begin{proof}
  Given \(\varepsilon>0\), pick \(N\) such that \(|f_n(x)-f(x)|<\varepsilon/2(b-a)\) for all \(x\in [a,b]\). Then
  \begin{align*}
    \|f_n-f\| &= \int_a^b |f_n(x)-f(x)| dx \\
              &\leq \int_a^b\varepsilon/2(b-a) dx \\
              &= (\varepsilon/2(b-a))(b-a) \\
              &= \varepsilon/2.
\end{align*}
\end{proof}

Equivalently, the map \(\id: (C[a,b], \|\cdot\|_\infty)\to (C[a,b], \|\cdot\|_1)\) is continuous.

\begin{eg}
  Let \(A=[0,1]\),
  \begin{enumerate}
  \item \(f(x) =
      \begin{cases}
        nx & x\in [0,1/n] \\ 2-nx & x\in [1/n,2/n] \\0 & x\geq2/n
      \end{cases}
  \)
  Then \((f_n)\to 0\) pointwise and in measure but not uniformly.
\item \(g_n(x) =
    \begin{cases}
      n^2 x & x\in [0,1/n]\\
      2n-n^2 x& x\in[1/n, 2/n] \\
      0 & x \geq 2/n
    \end{cases}
  \)
  Then \((g_n)\to 0\) pointwise but \((g_n)\nto 0\) in measure or uniformly.
  \end{enumerate}
\end{eg}

\subsection{Power Series}

\begin{question}
  Given
  \[
    f(x) = \sum_{i=0}^{\infty} \frac{x^i}{i!},
  \]
  how do I know if \(f(x)\) is continuous/differentiable?
\end{question}

Recall some facts about series from Analysis I:
\begin{enumerate}
\item The series \(\sum_{i=0}^{\infty}c_i = c\in\C \) means that \((\sum_{i=0}^{\infty})\to c\), as real vector space \((\C, \|\cdot\|) \cong (\R^2, \|\cdot\|)\).
\item \(\sum_{i=0}^{\infty}c_i \) converges if and only if there exists \(N\in\N\) such that \(\sum_{i=N}^{\infty}c_i \) converges.
\item Geometric series: \(\sum_{i=k}^{\infty}\alpha^i = \frac{\alpha^k}{1-\alpha} \) for \(|\alpha|<1\).
\item If \(\sum_{i=0}^{\infty}c_i \) converges then \((c_i)\to 0\).
\item Comparison test: if \(0 \leq a_i \leq b_i\) for all \(i\) and \(\sum_{i=0}^{\infty}b_i \) converges then \(\sum_{i=0}^{\infty}a_i \) converges and \(\sum_{i=0}^{\infty}a_i \leq \sum_{i=0}^{\infty}b_i \).
\item Absolute convergence: if \(\sum_{i=0}^{\infty}|c_i| \) converges then \(\sum_{i=0}^{\infty}c_i \) converges and \(|\sum_{i=0}^{\infty}c_i| \leq \sum_{i=0}^{\infty}|c_i| \).
\end{enumerate}

\begin{lem}
  If \(0 \leq |c_i| \leq b_i \) for all \(i\) and \(\sum_{i=0}^{\infty}b_i \) converges then \(\sum_{i=0}^{\infty}c_i \) converges.
\end{lem}

\begin{proof}
  Combine property 5 and 6.
\end{proof}

\begin{defi}[Power series]
A series of the form
\[
\sum_{i=0}^{\infty}a_i(z-c)^i,
\]
where \(a_i,z,c\in C\) is called a \emph{power series}. \(C\) is the \emph{centre}.
\end{defi}

\begin{prop}[Pointwise convergence]
  \label{prop:pointwise convergence of series}
  If
  \[
    \sum_{i=0}^{\infty}a_i(z_0-c)^i
  \]
  converges then
  \[
    \sum_{i=0}^{\infty}a_i(z-c)^i
  \]
  converges whenever
  \[
|z-c| < |z_0-c|.
  \]
\end{prop}

\begin{proof}
  By property 4 \((a_i(z_0-c)^i)\to 0\). Pick \(N\) such that \(|a_i(z_0-c)^i|<1\) for all \(i\geq N\). By Property 2 it suffices to show that
  \[
    \sum_{i=N}^{\infty}a_i(z-c)^i
  \]
  converges. Now for \(i\geq N\),
  \[
    \label{eqn:fundamental est}
    \boxed{
    |a_i(z-c)^i| = |a_i(z_0-c)^i| \cdot \Big| \frac{z-c}{z_0-c} \Big|^i < 1\cdot \alpha^i
    }
    \tag*{Fundamental Estimate for Power Series}
    \]
  where \(\alpha = |\frac{z-c}{z_0-c}|\). So if \(|z-c| < |z_0-c|,\alpha<1\), \(\sum_{i=N}^{\infty}\alpha^i \) converges by property 3.

  In summary, we have
  \[
|a_i(z-c)^i| < \alpha^i
\]
for all \(i\geq N\) and \(\sum_{i=N}^{\infty}\alpha^i \) converges. By the lemma \(\sum_{i=N}^{\infty}a_i(z-c)^i \) converges.
\end{proof}

\begin{defi}[Radius of convergence]
  \[
    R := \sup \{ |z-c|: \sum_{i=0}^{\infty}a_i (z-c)^i \text{ converges}\}
  \]
  is the \emph{radius of convergence} of \(\sum_{i=0}^{\infty}a_i(z-c)^i \).
\end{defi}

Proposition~\ref{prop:pointwise convergence of series} implies that if \(z\in B_R(c)\) then \(\sum_{i=0}^{\infty}a_i(z-c)^i \) converges. In other words, if we define
\begin{align*}
  f: B_R(c) &\to \C \\
  z &\mapsto \sum_{i=0}^{\infty} a_i(z-c)^i \\
  P_n: B_R(c) &\to \C \\
  z &\mapsto \sum_{i=0}^{n} a_i(z-c)^i
\end{align*}

Proposition~\ref{prop:pointwise convergence of series} says that \((P_n)\to f\) pointwise on \(B_R(c)\). As \(P_n\) are polynomials so they are continuous. A natural question is, is \(f\) continuous as well? We know this answer will be yes if we can prove that the convergence is uniform. How do we do that?

\begin{thm}
  With notations as above,
  \[
(P_n)\to f
\]
uniformly on \(\cl B_r(c)\) whenever \(r<R\).
\end{thm}

\begin{note}
  Equivalently, we can say \((P_n)\to f\) uniformly on \(B_r(c)\) for \(r<R\). The closed ball \(\cl B_r(c)\) is just a convention when talking about uniform convergence on a compact set.
\end{note}

\begin{proof}
  Define
  \[
    E_n(z) = f(z) - P_n(z) = \sum_{i=n+1}^{\infty}a_i(z-c)^i. 
  \]
  Fix \(r< R\). Given \(\varepsilon>0\), need to find \(N \) such that \(|E_n(z)| < \varepsilon\) whenever \(n\geq N\) and \(z\in \cl B_r(c)\).

  Choose \(z_0\) with \(r< |z_0-c| < R\) as in the proof of Proposition~\ref{prop:pointwise convergence of series}, pick \(N_0\) such that \(|a_i(z_0-c)|^i<1\) for \(i\geq N_0\). Now we use \ref{eqn:fundamental est}. For \(i\geq N_0\), we have \(|a_i(z-c)^i| < \alpha(z)^i\) where \(\alpha(z) = |\frac{z-c}{z_0-c}|\). For \(z\in \cl B_r(c)\),
  \[
    \alpha(z) = \Big| \frac{z-c}{z_0-c} \Big| \leq \frac{r}{|z_0-c|} = \alpha_0 < 1
  \]
  since \(r < |z_0-c|\). Hence for \(n>N_0\),
  \[
    |E_n(z)| \leq \sum_{i=n+1}^{\infty}|a_i(z-c)^i| \leq \sum_{i=n+1}^{\infty}\alpha_0^i = \frac{\alpha_0^{n+1}}{1-\alpha_0}.
  \]
  As \(\alpha_0 < 1\), \((\alpha_0^i)\to 0\). Pick \(N\geq N_0\) such that
  \[
\alpha_0^i < \varepsilon (1-\alpha_0)
  \]
  for \(i\geq N\). So for \(n>N\),
  \[
    |E_n(z)| < \frac{\varepsilon (1-\alpha_0)}{1-\alpha_0} = \varepsilon
  \]
  for all \(z\in \cl B_r(c)\).
  This is what we wanted.
\end{proof}

\begin{note}
  It need not be true that \((P_n)\to f\) uniformly on \(B_R(c)\). For example,
  \[
    \sum_{i=0}^{\infty}z^i
  \]
  does not converge uniformly on \(B_1(0)\).
\end{note}

\begin{cor}
  \(f\) as above is continuous on \(B_R(c)\).
\end{cor}

\begin{proof}
  Let \(U_r = B_r(c), r<R\). Then \(U_r\) is open in \(\C\). \((P_n)\to f\) uniformly on \(U_r\) for \(r< R\). Since the \(P_n\) are continuous \(f|_{U_r}\) is continuous. By gluing lemma \(f\) is continuous on
  \[
    U = \bigcup_{r<R} U_r = B_R(c).
  \]
\end{proof}

To summarise, power series are continuous on domain of convergence \(B_R(c)\).

\subsection{Integration \& Differentiation}

Recall from Analysis I
\begin{thm}[Fundamental Theorem of Calculus]
  \label{thm:FTC}
  Suppose \(f\in C[a,b]\), \(c\in [a,b]\), then
  \[
    F(x) = \int_{c}^{x} f(y) dy
  \]
  is well-defined for \(x\in[a,b]\) and
  \[
    F'(x) = f(x).
  \]
\end{thm}
and the following properties of (Riemann) integral:
\begin{enumerate}
\item If \(f(x) \leq g(x) \) for \(x\in[a,b]\), \(\int_{a}^{b} f(x) dx \leq \int_{a}^{b} g(x) dx \).
\item \(\big|\int_{a}^{b} f(x) dx \big) \leq \int_{a}^{b} |f(x)| dx \).
\item If \(b<a\), \(\int_{a}^{b} f(x) dx = - \int_{b}^{a} f(x) dx \).
\end{enumerate}

\begin{lem}
  If \(|f(x)|\leq C\) for all \(x\in[a,b]\) then
  \[
    \Big| \int_{c}^{x} f(t) dt \Big| \leq C|x-c|.
  \]
\end{lem}

\begin{proof}
  If \(x\geq c\) then
  \[
    \Big| \int_{c}^{x} f(t) dt \Big| \leq \int_{c}^{x} |f(t)| dt \leq \int_{c}^{x} C dt = C(x-c).
  \]
  If \(x\leq c\) then
   \[
    \Big| \int_{c}^{x} f(t) dt \Big| \leq \int_{c}^{x} |-f(t)| dt \leq C|x-c|.
  \]
\end{proof}

Now suppose \(f_n\in C[a,b]\) and \((f_n)\to f\) uniformly on \([a,b]\) so \(f\in C[a,b]\). Define
\begin{align*}
  F_n(x) &= \int_{c}^{x} f_n(t) dt \\
  F(x) &= \int_{c}^{x} f(t) dt
\end{align*}

\begin{prop}
  \label{prop:power series integral convergence}
  With notations above,
  \[
    (F_n) \to F
  \]
  uniformly on \([a,b]\).
\end{prop}

\begin{proof}
  Given \(\varepsilon>0\), there exists \(N\) such that \(|f_n(x) - f(x)| < \varepsilon/(b-a)\) for all \(n\geq N\) and \(x\in[a,b]\). Then
  \begin{align*}
    |F_n(x) - F(x)| &= \Big| \int_{c}^{x} f_n(t) dt - \int_{c}^{x} f(t) dt \Big| \\
                    &\leq \Big| \int_{c}^{x} \big( f_n(t) - f(t) \big) dt \Big| \\
                    &\leq |x-c| \cdot\frac{\varepsilon}{b-a} \text{ by lemma} \\
                    &= \varepsilon
  \end{align*}
  since \(x,c\in[a,b]\), \(|x-c|\leq b-a\). Thus \((F_n)\to F\) uniformly on \([a,b]\).
\end{proof}

Now suppose \(f(x) = \sum_{i=0}^{\infty}a_i(x-c)^i \) is a real power series with radius of convergence \(R\). Then for \(r< R\) and \(P_n(x) = \sum_{i=0}^{n}a_i(x-c)^i \), \((P_n)\to f\) uniformly on \(\cl B_r(c) = [c-r,c+r]\).

\begin{cor}
  \[
    \int_{c}^{x} f(t) dt = \sum_{i=0}^{\infty}\frac{a_i}{i+1}(x-c)^{i+1}
  \]
  for \(x\in (c-R,c+R)\).
\end{cor}

\begin{proof}
  Given \(x\), choose \(r\) with \(|x-c| < r < R\). Then \((P_n)\to f\) on \([c-r, c+r]\) so by Proposition~\ref{prop:power series integral convergence}
  \[
    (\mathbf{P}_n) \to \int_{c}^{x} f(t) dt
  \]
  uniformly on \([c-r, c+r]\) where
  \[
    \mathbf{P}_n(x) = \int_{c}^{x} \sum_{i=0}^{n}a_i(t-c)^i dt = \sum_{i=0}^{n} \frac{a_i}{i+1} (x-c)^{i+1}
  \]
  Since uniform convergence implies pointwise convergence,
  \[
    \sum_{i=0}^{\infty}\frac{a_i}{i+1}(x-c)^{i+1} = \int_{c}^{x} f(t) dt.
  \]
\end{proof}

\begin{question}
  If \((f_n)\to f\) uniformly on \([a,b]\) and \(f_n\) are differentiable, what can we say about \((f_n')\)?
\end{question}

The answer is, surprisingly, \emph{absolutely nothing}.

\begin{eg}
  Let \(f(x) = \frac{1}{n}\sin nx\). Then \((f_n)\to \V 0\) uniformly on \([0,\pi]\). But \(f_n'(x) = \cos nx\) does not even converge for any \(x\neq 0\).
\end{eg}

Nevertheless, if \(f(x) = \sum_{i=0}^{\infty}a_i(x-c)^i \) has radius of convergence \(R\), we still have

\begin{prop}
  \(f\) is differentiable on \((c-R, c+R)\) and
  \[
    f'(x) = \sum_{i=1}^{\infty}i a_i(x-c)^{i-1}. 
  \]
\end{prop}

In other words, power series can be differentiated term-by-term.

\begin{lem}
  \(g(x) = \sum_{i=1}^{\infty}i a_i(x-c)^{i-1} \) converges for all \(y\in(c-R,c+R)\).
\end{lem}

\begin{proof}
  Given \(x\in(c-R,c+R)\), pick \(x_0\) with \(|x-c| < |x_0-c| < R\). Then \(\sum_{i=0}^{\infty}a_i(x_0-c)^i \) converges, so by \ref{eqn:fundamental est}, there exists \(N\) such that
  \[
    |a_i(x-c)^i| \leq \alpha^i
  \]
  for all \(i\geq N\), where \(\alpha = \frac{|x-c|}{|x_0-c|}<1\). Then
  \[
    b_i:= |i a_i (x-c)^{i-1}| \leq \Big| \frac{i a_i}{x-c}\cdot(x-c)^i \Big| \leq \frac{i}{|x-c|}\alpha^i
  \]
  where we assume \(y\neq c\). Now
  \[
    \lim_{i\to\infty} \frac{b_{i+1}}{b_i} = \lim_{i\to\infty} \frac{i+1}{i}\cdot\alpha = \alpha<1
  \]
  so \(\sum_{i=1}^{\infty}\frac{i}{|x-c|}\alpha^i \) converges by ratio test. Since
  \[
    |i a_i(x-c)^{i-1}| \leq \frac{i}{|x-c|}\alpha^i,
  \]
  \(\sum_{i=1}^{\infty}i a_i(x-c)^{i-1} \) converges by comparison test. If \(y=c\) then the convergence is obvious.
\end{proof}

\begin{proof}[Proof of proposition]
  \(g(x) = \sum_{i=1}^{\infty}i a_i (x-c)^{i-1} \) converges on \((c-R,c+R)\), so by term-by-term integration
  \[
    \int_{c}^{x} g(t) dt = \sum_{i=1}^{\infty}a_i(x-c)^i = f(x) - f(c).
  \]
  Now \(g(x)\) is continuous on \((c-R, c+R)\) so we can apply~\ref{thm:FTC} so \(f'(x) = g(x)\) for all \(x\in(c-R,c+R)\).
\end{proof}

\begin{application}
  Power series solutions of ODEs are legit as long as you check the radius of convergence.
\end{application}

\section{Compactness}

\subsection{Compact subsets of \(\R^n\)}

Let \((V, \norm \cdot)\) be a normed vector space. If \((v_n)\) is a sequence in \(V\) and \((n_j)\) is an increasing sequence of positive integers (i.e.\ \(n_{j+1}>n_j\)) then \((v_{n_j})\) is a \emph{subsequence} of \((v_n)\).

\begin{ex}
  if \((v_n)\to v\) in \(v\) then any subsequence \((v_{n_j})\) converges to \(v\) as well.
\end{ex}

\begin{defi}[Boundedness]
  \(A \subseteq V\) is bounded if there exists \(m\) such that \(\norm v \leq m \) for all \(v\in A\).
\end{defi}

\begin{rmk}
  if \(\norm \cdot\) and \(\norm \cdot'\) are Lipschitz equivalent then \(A\) is bounded with respect to \(\norm \cdot\) if and only if with respect to \(\norm \cdot'\). It follows that boundedness in \(\R^n\) means with respect to any one of \(\norm \cdot_1, \norm \cdot_2, \norm \cdot_\infty\).
\end{rmk}

Recall from analysis i:
\begin{thm}[Bolzano-Weistrass]
  A bounded sequence in \(\R\) has a convergent subsequence.
\end{thm}

\begin{cor}[Bolzano-Weistrass for \(\R^m\)]
  A bounded sequence in \(\R^m\) has a convergent subsequence.
\end{cor}

\begin{proof}
  Inducton on \(m\): if \(m=1\) then done by Bolzano-Weistrass. Suppose it holds for \(\R^{m-1}\) and let \((v_n)\) be a bounded sequence in \(\R^m\). Write \(v_n = (v_{n,1},\dots,v_{n,n}) = (w_n,v_{n,m})\) for some \(w_n\in \R^{m-1}\). \(\norm{w_n}\) and \(|v_{n,m}| \leq \norm{v_n}\) so \((v_n)\) is bounded implies that \((w_n)\) and \((v_{n,m})\) are bounded. By induction \((w_n)\) has a subsequence \((w_{n_j}) \to w \in \R^{m-1}\). Now consider \((v_{n_j,m})\). This is a bounded sequence in \(\R\) so by Bolzano-Weistrass there is a subsequence \((v_{n_{j_k},m}) \to v \in\R\). By Exercise \((w_{n_{j_k}}) \to w\) so
  \[
    (v_{n_{j_k}}) = ((w_{n_{j_k}},v_{n_{j_k},m})) \to (w, v) \in \R^m.
  \]
\end{proof}

\begin{defi}[Sequential compactness]
  \(C \subseteq V\) is \emph{sequenctially compact} if any sequence \((v_n)\) in \(C\) has a convergent subsequence \((v_{n_j}) \to v \in C\).
\end{defi}

\begin{rmk}
  There is another (topological) definition of compactness using open covers. For metric spaces, in particular subspaces of normed spaces, these two are equivalent.
\end{rmk}

\begin{eg}\leavevmode
  \begin{enumerate}
  \item \(\R\) is not compact as \((n)\) has no convergent subsequence.
  \item \((0,1]\) is not compact as \((1/n) \to 0\) but \(0 \notin A\).
  \end{enumerate}
\end{eg}

\begin{thm}[Heine-Borel]
  \(A \subseteq \R^m\) is compact if and only if \(A\) is closed and bounded.
\end{thm}

\begin{proof}
  \(\Leftarrow\): Suppose \(A\) is closed and bounded. Given a sequence \((v_n)\) with \(v_n\in A\), must find a convergent subsequence. Since \(A\) is bounded, \((v_n)\) is bounded so by Bolzano-Weistrass there is a convergent subsequence \((v_{n_j})\to v \in \R^m\). As \(A\) is closed and \(v_{n_j}\in A\), \((v_{n_j})\to v\) implies that \(v \in A\).

  \(\Rightarrow\): If \(A\) is not closed or not bounded, we will find a sequence \((v_n)\) with \(v_n \in A\) with no convergent subsequence:
  \begin{itemize}
  \item if \(A\) is not closed, there is a sequence \((v_n)\to v\) with \(v_n \in A\) but \(v \notin A\). Suppose \((v_{n_j}) \to w\) is a convergent subsequence. Then by Exercise \((v_{n_j})\to v\). By uniqueness of limits \(v = w \notin A\).
  \item if \(A\) is not bounded, then for each \(n > 0\) there exists \(v_n \in A\) with \(\norm{v_n} \geq n\). Consider the sequence \((v_n)\). Suppose there is a subsequence \((v_{n_j})\to v\). Then we can find \(J\) such that \(\norm{v_{n_j}-v} < 1\) for all \(j\geq J\). Pick \(K\geq \max(J,\norm v + 1)\). Then for \(j\geq K\),
    \[
      \norm{v_{n_j}} \leq \norm{v_{n_j}-v} + \norm v \leq 1+ \norm v \leq K \leq n_j
    \]
    since \(n_j \geq j \geq K\). So \(\norm{v_{n_j}} < n_j\) for \(j\geq K\), contradiction.
  \end{itemize}
\end{proof}

\begin{eg}
  \((V, \norm \cdot) = (C[0,1], \norm \cdot_\infty)\). Consider \(f_n(x) = \begin{cases} 1-nx & x\in[0,1/n] \\ 0 & x\geq 1/n \end{cases}\). Note if \(f(x) = \begin{cases} 1 & x=0 \\ 0 & x>0 \end{cases}\) then \((f_n) \to f\) pointwise. Claim \((f_n)\) has no convergent subsequence with respect to \(\norm \cdot_\infty\).

  \begin{proof}
    Suppose \((f_{n_j})\to g\) uniformly. Then \((f_{n_j})\to g\) pointwise. But we know from Exercise \((f_{n_j})\to f\) pointwise so \(f=g\). But \(f_n\) is continuous so \(g\) is continuous. Contradition.
  \end{proof}

  \begin{note}
    \(f_n\in \cl B_1(0) \subseteq V\).
  \end{note}
\end{eg}

\begin{cor}
  \(\cl B_1(0)\) is closed and bounded in \((C[0,1], \norm \cdot_\infty)\) but is not compact.
\end{cor}

\begin{prop}[Continuous image of compact set]
  Suppose \(C \subseteq V\) is compact and \(f:C\to W\) is continuous then \(f(C)\) is also compact.
\end{prop}

\begin{proof}
  Suppose \((w_n)\) is a sequence in \(f(C)\). Pick \(v_n \in C\) with \(f(v_n) = w_n\). \(C\) is compact so \((v_n)\) has a convergent subsequence \((v_{n_j})\to v \in C\). \(f\) is continuous so
  \[
    (w_{n_j}) = (f(v_{n_j})) \to f(v) \in f(C).
  \]
\end{proof}

\begin{application}[Maximum Value Theorem]
 \begin{lem}
  If \(\emptyset \neq A \subseteq \R \) is compact then \(\sup A \in A\).
\end{lem}

\begin{proof}
  By Heine-Borel, \(A\) is closed and bounded so \(\alpha = \sup A\in \R\). For each \(n>0\), exists \(a_n \in A\) such that \(\alpha-1/n \leq a_n \leq \alpha\). Then \((a_n)\to \alpha\). Since \(A\) is closed and \(a_n \in A\), \(\alpha\in A\) as well.
\end{proof}
 
\begin{thm}
  Suppose \(f:C\to \R\)  is continuous and \(C\) is compact and nonempty. Then exists \(v\in C\) such that \(f(v) \geq f(v')\) for all \(v' \in C\).
\end{thm}

\begin{proof}
  \(f(C)\) is compact and nonempty by the Proposition, so Lemma implies that \(\alpha = \sup f(C)\) exists and \(\alpha\in f(C)\). Pick \(v\in C\) with \(f(v) = \alpha\). If \(v'\in C\) then \(f(v') \in f(C)\) so \(f(v') \leq \alpha = f(v)\).
\end{proof}

\begin{cor}
  Let \(f\) and \(C\) be as above. Then there exists \(v_-\in C\) with \(f(v_-) \leq f(v')\) for all \(v' \in C\).
\end{cor}

\begin{proof}
  Apply Theorem to \(-f\).
\end{proof}
\end{application}

\begin{application}[Norms on \(\R^n\)]
  Let \(\norm \cdot\) be some norm on \(\R^m\).
  
  \begin{lem}
    The map \(\id: (\R^m, \norm \cdot_1) \to (\R^m, \norm \cdot)\) is continuous.
  \end{lem}

  \begin{proof}
    By the criterion in Proposition~\ref{prop:continuity of id} it suffices to show that there is a constant \(C\) such that \(\norm v \leq C \norm v_1\) for all \(v\in \R^m\). Let \(v = (v_1,\dots,v_m) = \sum_{i=1}^{m}v_ie_i \) where \(e_i\) is the standard basis vector. Take \(C = \max_{1\leq i\leq m}\norm{e_i}\). Then
    \[
      \norm v \leq \sum_{i=1}^{m}\norm{v_ie_i} = \sum_{i=1}^{m}|v_i| \norm{e_i} \leq C \sum_{i=1}^{m}|v_i| = C \norm v_1
    \]
  \end{proof}

  \begin{cor}
    The map \(f:(\R^m, \norm \cdot_1) \to (\R, |\cdot|)\) given by \(f(v) = \norm v\) is continuous.
  \end{cor}

  \begin{proof}
    \(f = g\compose \id\) where \(g\) is the continuous map from \(\R^m\) to \(\R\) in Example~\ref{eg:continuity}.
  \end{proof}
\end{application}

Recall that two norms \(\norm \cdot\) and \(\norm \cdot'\) on \(V\) are Lipschitz equivalent if there exists \(C\) such that
\[
  C\norm v \leq \norm v' \leq C \norm v
\]
for all \(v\in V\).

\begin{rmk}
  This is trivially true if \(v=0\) so suffices to check for \(v\neq0\).
\end{rmk}

\begin{thm}
  If \(\norm \cdot\) is a norm on \(\R^m\) then it is Lipschitz equivalent to \(\norm \cdot_1\).
\end{thm}

\begin{proof}
  Let \(S = \{v\in \R^m: \norm v_1 = 1 \}\). Claim \(S\) is compact with respect to \(\norm \cdot_1\): \(S\) is clearly bounded. Consider \(g:(\R^m, \norm \cdot_1) \to (\R, |\cdot|), v\mapsto \norm v_1\). \(g\) is continuous and \(S = g^{-1}(\{1\})\). As \(\{1\} \subset \R\) is closed \(S\) is closed. So by Heine-Borel \(S\) is compact.

  By Corollary \(f:(S, \norm \cdot_1) \to (\R, |\cdot|)\) given by \(f(v) = \norm v\) is continuous. By the Maximum Value Theorem there exists \(v_\pm\in S\) with
  \[
    f(v_-) \leq f(v') \leq f(v_+)
  \]
  for all \(v' \in S\). Let \(C_\pm = f(v_\pm)\). Notice that
  \[
    v_\pm\in S \Rightarrow \norm{v_\pm}_1 = 1 \Rightarrow v_\pm \neq0 \Rightarrow C_-=\norm{v_\pm} \neq 0.
  \]
  Let \(C = \max(C_+,1/C_-)\). Then
  \[
    \frac{1}{C} \leq C_- \leq f(v) = \norm{v} \leq C_+ \leq C
  \]
  for all \(v \in S\).

  Finally, if \(v \in \R^m\setminus\{0\}\) then \(v/\norm v_1 \in S\) so
  \[
    \frac{1}{C} \leq \norm*{\frac{v}{\norm v_1}} \leq C
  \]
  and Lipschitz equivalence condition follows.
\end{proof}

\begin{cor}
  Any two norms on \(\R^m\) are Lipschitz equivalent.
\end{cor}

\begin{proof}
  Lipschitz equivalence is an equivalence relation.
\end{proof}

\subsection{Completeness}

Let \(V\) be a normed vector space.

\begin{defi}[Cauchy sequence]
  A sequence \((v_n)\) in \(V\) is \emph{Cauchy} if for any \(\varepsilon>0\), there exists \(N\) such that for all \(n,m\geq N\), \(\norm{v_n-v_m} < \varepsilon\).
\end{defi}

\begin{eg}\leavevmode
  \begin{enumerate}
  \item If \((v_n)\to v\) then \((v_n)\) is Cauchy.
    \begin{proof}
      Given \(\varepsilon>0\), pick \(N\) such that \(\norm{v_n-v} < \varepsilon/2\) for \(n\geq N\). Then for \(n,m \geq N\),
      \[
        \norm{v_n-v_m} \leq \norm{v_n-v}+ \norm{v-v_m} < \varepsilon.
      \]
    \end{proof}
  \item If \(s_n = \sum_{i=1}^{n}1/i \), then \((s_n)\) is not Cauchy since for any fixed \(N\), \(\sum_{i=N}^{m}1/i \to \infty \) as \(m\to \infty\).
  \end{enumerate}
\end{eg}

Informally, a Cauchy sequence \emph{wants} to converge: given \(\varepsilon>0\), pick \(N\) such that \(\norm{v_n-v_m}< \varepsilon\) for all \(n,m\geq N\). Then \(\norm{v_n-v_N} < \varepsilon\) for all \(n\geq N\) so \(v_n \in B_\varepsilon(v_N)\).

But there may \emph{not} be anything for it to converge to!

\begin{eg}
  \(V = C[0,1]\) with \(\norm \cdot_1\). Let \(f_n(x) =
  \begin{cases}
    0 & x\in[0,1/2] \\
    n(x-1/2) & x\in[1/2,1/2+1/n] \\
    1 & x\geq 1/2+1/n
  \end{cases}
  \) Then \((f_n)\) is Cauchy but does not converge to any \(f\in C[0,1]\).
\end{eg}
\end{document}