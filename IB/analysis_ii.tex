\documentclass[a4paper]{article}

\def\ntitle{Analysis II}
\def\ndate{Michaelmas, 2017 -- 2018}

\ifx \nauthor\undefined
  \def\nauthor{Qiangru Kuang}
\else
\fi

\ifx \ntitle\undefined
  \def\ntitle{Template}
\else
\fi

\ifx \nauthoremail\undefined
  \def\nauthoremail{qk206@cam.ac.uk}
\else
\fi

\ifx \ndate\undefined
  \def\ndate{\today}
\else
\fi

\title{\ntitle}
\author{\nauthor}
\date{\ndate}

%\usepackage{microtype}
\usepackage{mathtools}
\usepackage{amsthm}
\usepackage{stmaryrd}%symbols used so far: \mapsfrom
\usepackage{empheq}
\usepackage{amssymb}
\let\mathbbalt\mathbb
\let\pitchforkold\pitchfork
\usepackage{unicode-math}
\let\mathbb\mathbbalt%reset to original \mathbb
\let\pitchfork\pitchforkold

\usepackage{imakeidx}
\makeindex[intoc]

%to address the problem that Latin modern doesn't have unicode support for setminus
%https://tex.stackexchange.com/a/55205/26707
\AtBeginDocument{\renewcommand*{\setminus}{\mathbin{\backslash}}}
\AtBeginDocument{\renewcommand*{\models}{\vDash}}%for \vDash is same size as \vdash but orginal \models is larger
\AtBeginDocument{\let\Re\relax}
\AtBeginDocument{\let\Im\relax}
\AtBeginDocument{\DeclareMathOperator{\Re}{Re}}
\AtBeginDocument{\DeclareMathOperator{\Im}{Im}}
\AtBeginDocument{\let\div\relax}
\AtBeginDocument{\DeclareMathOperator{\div}{div}}

\usepackage{tikz}
\usetikzlibrary{automata,positioning}
\usepackage{pgfplots}
%some preset styles
\pgfplotsset{compat=1.15}
\pgfplotsset{centre/.append style={axis x line=middle, axis y line=middle, xlabel={$x$}, ylabel={$y$}, axis equal}}
\usepackage{tikz-cd}
\usepackage{graphicx}
\usepackage{newunicodechar}

\usepackage{fancyhdr}

\fancypagestyle{mypagestyle}{
    \fancyhf{}
    \lhead{\emph{\nouppercase{\leftmark}}}
    \rhead{}
    \cfoot{\thepage}
}
\pagestyle{mypagestyle}

\usepackage{titlesec}
\newcommand{\sectionbreak}{\clearpage} % clear page after each section
\usepackage[perpage]{footmisc}
\usepackage{blindtext}

%\reallywidehat
%https://tex.stackexchange.com/a/101136/26707
\usepackage{scalerel,stackengine}
\stackMath
\newcommand\reallywidehat[1]{%
\savestack{\tmpbox}{\stretchto{%
  \scaleto{%
    \scalerel*[\widthof{\ensuremath{#1}}]{\kern-.6pt\bigwedge\kern-.6pt}%
    {\rule[-\textheight/2]{1ex}{\textheight}}%WIDTH-LIMITED BIG WEDGE
  }{\textheight}% 
}{0.5ex}}%
\stackon[1pt]{#1}{\tmpbox}%
}

%\usepackage{braket}
\usepackage{thmtools}%restate theorem
\usepackage{hyperref}

% https://en.wikibooks.org/wiki/LaTeX/Hyperlinks
\hypersetup{
    %bookmarks=true,
    unicode=true,
    pdftitle={\ntitle},
    pdfauthor={\nauthor},
    pdfsubject={Mathematics},
    pdfcreator={\nauthor},
    pdfproducer={\nauthor},
    pdfkeywords={math maths \ntitle},
    colorlinks=true,
    linkcolor={red!50!black},
    citecolor={blue!50!black},
    urlcolor={blue!80!black}
}

\usepackage{cleveref}



% TODO: mdframed often gives bad breaks that cause empty lines. Would like to switch to tcolorbox.
% The current workaround is to set innerbottommargin=0pt.

%\usepackage[theorems]{tcolorbox}





\usepackage[framemethod=tikz]{mdframed}
\mdfdefinestyle{leftbar}{
  %nobreak=true, %dirty hack
  linewidth=1.5pt,
  linecolor=gray,
  hidealllines=true,
  leftline=true,
  leftmargin=0pt,
  innerleftmargin=5pt,
  innerrightmargin=10pt,
  innertopmargin=-5pt,
  % innerbottommargin=5pt, % original
  innerbottommargin=0pt, % temporary hack 
}
%\newmdtheoremenv[style=leftbar]{theorem}{Theorem}[section]
%\newmdtheoremenv[style=leftbar]{proposition}[theorem]{proposition}
%\newmdtheoremenv[style=leftbar]{lemma}[theorem]{Lemma}
%\newmdtheoremenv[style=leftbar]{corollary}[theorem]{corollary}

\newtheorem{theorem}{Theorem}[section]
\newtheorem{proposition}[theorem]{Proposition}
\newtheorem{lemma}[theorem]{Lemma}
\newtheorem{corollary}[theorem]{Corollary}
\newtheorem{axiom}[theorem]{Axiom}
\newtheorem*{axiom*}{Axiom}

\surroundwithmdframed[style=leftbar]{theorem}
\surroundwithmdframed[style=leftbar]{proposition}
\surroundwithmdframed[style=leftbar]{lemma}
\surroundwithmdframed[style=leftbar]{corollary}
\surroundwithmdframed[style=leftbar]{axiom}
\surroundwithmdframed[style=leftbar]{axiom*}

\theoremstyle{definition}

\newtheorem*{definition}{Definition}
\surroundwithmdframed[style=leftbar]{definition}

\newtheorem*{slogan}{Slogan}
\newtheorem*{eg}{Example}
\newtheorem*{ex}{Exercise}
\newtheorem*{remark}{Remark}
\newtheorem*{notation}{Notation}
\newtheorem*{convention}{Convention}
\newtheorem*{assumption}{Assumption}
\newtheorem*{question}{Question}
\newtheorem*{answer}{Answer}
\newtheorem*{note}{Note}
\newtheorem*{application}{Application}

%operator macros

%basic
\DeclareMathOperator{\lcm}{lcm}

%matrix
\DeclareMathOperator{\tr}{tr}
\DeclareMathOperator{\Tr}{Tr}
\DeclareMathOperator{\adj}{adj}

%algebra
\DeclareMathOperator{\Hom}{Hom}
\DeclareMathOperator{\End}{End}
\DeclareMathOperator{\id}{id}
\DeclareMathOperator{\im}{im}
\DeclarePairedDelimiter{\generation}{\langle}{\rangle}

%groups
\DeclareMathOperator{\sym}{Sym}
\DeclareMathOperator{\sgn}{sgn}
\DeclareMathOperator{\inn}{Inn}
\DeclareMathOperator{\aut}{Aut}
\DeclareMathOperator{\GL}{GL}
\DeclareMathOperator{\SL}{SL}
\DeclareMathOperator{\PGL}{PGL}
\DeclareMathOperator{\PSL}{PSL}
\DeclareMathOperator{\SU}{SU}
\DeclareMathOperator{\UU}{U}
\DeclareMathOperator{\SO}{SO}
\DeclareMathOperator{\OO}{O}
\DeclareMathOperator{\PSU}{PSU}

%hyperbolic
\DeclareMathOperator{\sech}{sech}

%field, galois heory
\DeclareMathOperator{\ch}{ch}
\DeclareMathOperator{\gal}{Gal}
\DeclareMathOperator{\emb}{Emb}



%ceiling and floor
%https://tex.stackexchange.com/a/118217/26707
\DeclarePairedDelimiter\ceil{\lceil}{\rceil}
\DeclarePairedDelimiter\floor{\lfloor}{\rfloor}


\DeclarePairedDelimiter{\innerproduct}{\langle}{\rangle}

%\DeclarePairedDelimiterX{\norm}[1]{\lVert}{\rVert}{#1}
\DeclarePairedDelimiter{\norm}{\lVert}{\rVert}



%Dirac notation
%TODO: rewrite for variable number of arguments
\DeclarePairedDelimiterX{\braket}[2]{\langle}{\rangle}{#1 \delimsize\vert #2}
\DeclarePairedDelimiterX{\braketthree}[3]{\langle}{\rangle}{#1 \delimsize\vert #2 \delimsize\vert #3}

\DeclarePairedDelimiter{\bra}{\langle}{\rvert}
\DeclarePairedDelimiter{\ket}{\lvert}{\rangle}




%macros

%general

%divide, not divide
\newcommand*{\divides}{\mid}
\newcommand*{\ndivides}{\nmid}
%vector, i.e. mathbf
%https://tex.stackexchange.com/a/45746/26707
\newcommand*{\V}[1]{{\ensuremath{\symbf{#1}}}}
%closure
\newcommand*{\cl}[1]{\overline{#1}}
%conjugate
\newcommand*{\conj}[1]{\overline{#1}}
%set complement
\newcommand*{\stcomp}[1]{\overline{#1}}
\newcommand*{\compose}{\circ}
\newcommand*{\nto}{\nrightarrow}
\newcommand*{\p}{\partial}
%embed
\newcommand*{\embed}{\hookrightarrow}
%surjection
\newcommand*{\surj}{\twoheadrightarrow}
%power set
\newcommand*{\powerset}{\mathcal{P}}

%matrix
\newcommand*{\matrixring}{\mathcal{M}}

%groups
\newcommand*{\normal}{\trianglelefteq}
%rings
\newcommand*{\ideal}{\trianglelefteq}

%fields
\renewcommand*{\C}{{\mathbb{C}}}
\newcommand*{\R}{{\mathbb{R}}}
\newcommand*{\Q}{{\mathbb{Q}}}
\newcommand*{\Z}{{\mathbb{Z}}}
\newcommand*{\N}{{\mathbb{N}}}
\newcommand*{\F}{{\mathbb{F}}}
%not really but I think this belongs here
\newcommand*{\A}{{\mathbb{A}}}

%asymptotic
\newcommand*{\bigO}{O}
\newcommand*{\smallo}{o}

%probability
\newcommand*{\prob}{\mathbb{P}}
\newcommand*{\E}{\mathbb{E}}

%vector calculus
\newcommand*{\gradient}{\V \nabla}
\newcommand*{\divergence}{\gradient \cdot}
\newcommand*{\curl}{\gradient \cdot}

%logic
\newcommand*{\yields}{\vdash}
\newcommand*{\nyields}{\nvdash}

%differential geometry
\renewcommand*{\H}{\mathbb{H}}
\newcommand*{\transversal}{\pitchfork}
\renewcommand{\d}{\mathrm{d}} % exterior derivative

%number theory
\newcommand*{\legendre}[2]{\genfrac{(}{)}{}{}{#1}{#2}}%Legendre symbol


\theoremstyle{definition}
\newtheorem*{joke}{Joke}

\begin{document}
\maketitle

\tableofcontents

\setcounter{section}{-1}

\section{Introduction}

In Analysis I, the primary space we are interested in is \(\R\) and we studied notions such as continuity, convergence, differentiation, integration and solving equaiton through, for example, Intermediate Value Theorem. In Analysis II, we moved to the study general function space.

\begin{table}[htbp]
  \centering
  \begin{tabular}{|c|c|c|}
    \hline
    & $\mathbb{R}^m$ & Function space \\ \hline
    Continuity \& convergence & $\checkmark$ & $\checkmark$ \\ \hline
    Differentiation & $\checkmark$ & Calculus of variations \\ \hline
    Integration & Probability and measure & ??? (ask physicists) \\ \hline
    Solving equations & inverse function theorem & existence of solutions for ODEs \\ \hline
  \end{tabular}
  \caption{Comparison of Euclidean space and function space}
\end{table}

\section{Normed Vector Spaces}

A motivating example: if $(a_n)$ is a sequence of real numbers, then $(a_n)\to a$ if
\[
  \forall \varepsilon>0,\exists N s.t. \forall n>N, |a_n-a|<\varepsilon.
\]
Now if I replace $\mathbb{R}$ by a real vector space $V$, what do I replace $|\cdot|$ with?

\begin{defi}
  If $V$ is a real vector space, a \emph{norm} on $V$ is a function $\|\cdot\|:V\to\mathbb{R}$ satisfying
  \begin{enumerate}
  \item $\forall \V v \in V, \|\V v\| \geq 0$ with equality if and only if $\V v =\V 0$.
  \item $\forall \V v,\forall \lambda \in \R \|\lambda \V v\| = |\lambda| \|\V v\|$
    \item $\forall \V v,\V w\in V, \|\V v+\V w\| \leq \|\V v\| + \|\V w\|$ (triangle inequality).
  \end{enumerate}
\end{defi}

\begin{eg}\leavevmode
  \begin{enumerate}
  \item $V= \mathbb{R}^m, \V v = (v_1,\ldots,v_m)$,
    \begin{enumerate}
    \item $\|\V v\| = (\sum_{i=1}^m v_i^2)^{1/2}$, the Euclidean norm,
    \item $\|\V v\|_\infty = \max |v_i|$, the max norm,
      \item $\|\V v\|_1 = \sum_{i=1}^m |v_i|$.
    \end{enumerate}
  \item $V=C[0,1]$,
    \begin{enumerate}
    \item $\|f\|_\infty=\max_{x\in[0,1]} |f(x)|$,
    \item $\|f\|_2=(\int_0^1 f(x)^2 dx)^{1/2}$, which comes from $\langle f,g\rangle = \int_0^1f(x)g(x) dx$,
      \item $\|f\|_1=\int_0^1|f(x)| dx$, the $L^1$ norm.
    \end{enumerate}
  \end{enumerate}
\end{eg}

\subsection{Convergence}

\begin{defi}
  Suppose $(V, \|\cdot\|)$ is a normed vector space and $(\V v_n)$ is a sequence of elements of $V$. We say $(\V v_n)$ converges to $\V v\in V$, denoted $(\V v_n)\to \V v$, if $\forall\varepsilon>0,\exists N s.t. \forall n>N, \|\V v_n-\V v\|<\varepsilon$. Equivalently, $(\V v_n)\to \V v$ if $(\|\V v_n-\V v\|)\to 0$.
\end{defi}

\begin{ex}
  Suppose $V=\mathbb{R}^m, (\V v_n) = (v_{n,1},\ldots,v_{n,m})$. Then $(\V v_n)\to \V v$ w.r.t. $\|\cdot\|_\infty$ means
  \begin{align*}
    & (\max_{1\leq i \leq m}|v_{n,i}-v_i|) \to 0 \\
    \Longleftrightarrow & (|v_{n,i}-v_i|)\to 0 \text{ for all } 1\leq i\leq m \\
    \Longleftrightarrow & (v_{n,i})\to v_i \text{ for all } 1\leq i\leq m
  \end{align*}

   The convergence w.r.t. $\|\cdot\|_1$ means
   \begin{align*}
     & (\sum_{i=1}^m |v_{n,i}-v_i|)\to 0 \\
     \Longleftrightarrow & (|v_{n,i}-v_i)\to 0 \text{ for all } 1\leq i\leq m \\ 
    \Longleftrightarrow & (v_{n,i})\to v_i \text{ for all } 1\leq i\leq m.
  \end{align*}
\end{ex}

\begin{rmk}
  Two different norms, same notion of convergence.
\end{rmk}

\begin{convention}
  If I say $(\V v_n)\to \V v$ where $(\V v_n)$ is a sequence in $\mathbb{R}^m$ without specifying the norm, I mean w.r.t. $\|\cdot\|_1,\|\cdot\|_2,\|\cdot\|_\infty$. (all give the same notion for convergence).
\end{convention}

\begin{eg}\label{eg:fdown}
  $V=C[0,1]$,
  \[
    f_n(x) =
    \begin{cases}
      1-nx & x\in[0,1/n] \\
      0 & x\geq 1/n
    \end{cases}
  \]
  Then $\|f_n\|_1=\int_0^1|f_n(x)| dx=\frac{1}{2n}\to 0$ so $(f_n)\to 0$ w.r.t. $\|\cdot\|_1$.

  But $\|f_n\|_\infty=1$ so $(\|f_n\|_\infty) \nto 0$ i.e. $(f_n) \nto 0$ w.r.t. $\|\cdot\|_\infty$.
\end{eg}

\begin{rmk}
  Two different norms, two different notions of convergence.
\end{rmk}

\subsection{Continuity}

\begin{defi}
  Suppose $V$ and $W$ are normed vector spaces. We say a function $f:V\to W$ is \emph{continuous} if
  \[
    (f(\V v_n))\to f(\V v) \text{ in } W \text{ whenever } (\V v_n)\to \V v \text{ in } V.
  \]
  
\end{defi}

\begin{eg}\leavevmode
  \begin{enumerate}
  \item $f:V\to \mathbb{R}^m, f(\V v) = (f_1(\V v), \ldots, f_m(\V v))$ is continuous if and only if $f_i:V\to \mathbb{R}$ is continuous for all $1\leq i \leq m$.
  \item $\rho_i:\mathbb{R}^m\to \mathbb{R}, \rho_i(\V v) =v_i$ is continuous.
  \item $F:C[0,1]\to\mathbb{R}, F(f)=f(0)$,
    \begin{enumerate}
    \item If $(f_n)$ is the sequence from example~\ref{eg:fdown}, then $F(f_n)=1$. Now $(f_n)\to \V 0$ w.r.t. $\|\cdot\|_1$. But $(F(f_n))\nrightarrow 0=F(\V 0)$. So $F$ is not continuous w.r.t. $\|\cdot\|_1$.
      \item If $(g_n)\to g$ w.r.t. $\|\cdot\|_\infty$, then $(\max|g_n(x)-g(x)|)\to 0$ so $(|g_n(0)-g(0))|\to 0$, $(|F(g_n)-F(g)|)\to 0$, so $F(g_n)\to F(g)$.
    \end{enumerate}
    $F$ is continuous w.r.t. $\|\cdot\|_\infty$ but not w.r.t. $\|\cdot\|_1$.
  \item If $f:V_1\to V_2$ and $g:V_2\to V_3$ are continuous then $g\compose f: V_1\to V_3$ are continuous. Proof: if $(\V v_n) \to (\V v)$ in $V_1$ , then as $f$ is continuous, $(f(\V v_n)) \to (f(\V v))$, then as $g$ is continuous, $(g(f(\V v_n))) \to (g(f(\V v)))$ in $V_3$.
  \item $\|\cdot\|: V \to \R$ is continuous. Proof: if $(\V v_n)\to \V v$, then $(\|\V v_n - \V v\|) \to 0$. Now
    \[
      0 \leq |\|\V v_n\| - \|\V v\|| \leq \|\V v_n - \V v\|
    \]
    by Lemma~\ref{lem:alternate triangle}. So $(|\|\V v_n - \V v\||) \to 0$ by squeeze rule, i.e. $\|\V v_n\| \to \|\V v\|$.
  \end{enumerate}
\end{eg}

\begin{lem}
  \label{lem:alternate triangle}
  $\|\V v- \V w\| \geq |\|\V v\| - \|\V w\||$ for all $\V v, \V w \in V$.
\end{lem}

\begin{proof}
  By triangle inequality, $\|\V v- \V w\| - \|\V w\| \geq \|\V v \|$ so $\|\V v - \V w\| \geq \|\V w - \V v\| \geq \|\V w - \V v\|$.
\end{proof}

More generally, if $X\subset V$ is a subset, we say $f:X\to W$ is continuous if
\[
  (f(\V x_n)) \to f(\V x)
\]
in $W$ whenever $(\V x_n) \to \V x$ in $V$ for $\V x$ and all $\V x_n \in X$.

\begin{eg}
  $f:\R \setminus \{0\} \to \R, x \mapsto \frac{1}{x}$ is continuous.
\end{eg}

\subsection{Open and Closed Subsets}

Let $(V, \|\cdot\|)$ be a normed vector space.

\begin{defi}
  If $\V v_0 \in V$ and $r \in \R$,
  \[
    B_r(\V v_0) = \{\V v\in V: \|\V v - \V v_0\| < r \}
  \]
  is the \emph{open ball} of radius $r$ centred at $\V v_0$, and
  \[
    \cl B_r(\V v_0) = \{\V v\in V: \|\V v - \V v_0\| < r \}
  \]
  is the \emph{closed ball} of radius $r$ centred at $\V v_0$.
\end{defi}

\begin{eg}\leavevmode
  \begin{enumerate}
  \item $(V, \|\cdot\|) = (\R, |\cdot|)$, then
    \begin{align*}
      B_r(a) &= (a-r, a+r) \\
      \cl B_r(a) &= [a-r, a+r] 
    \end{align*}
  \item $V=\R^2$, then $\cl B_1(\V 0)$ with respect to \texttt{to be filled in}.
\item $V = \R^3, \|\cdot\|_2$ is the ``three-dimensional ball''.
\item $(V, \|\cdot\|) = (C[0,1], \|\cdot\|_\infty)$,
  \[
    \cl B_r(f) = \{g\in C[0,1]: f(x)-r \leq g(x) \leq f(x) + r \: \forall x \in [0,1]\}.
  \]
  \end{enumerate}
\end{eg}

\begin{prop}[Alternate characterisation of Continuity]
  $f:V\to W$ is continuous if and only if
  \begin{equation}
    \label{eqn:alternate continuity}
    \forall \V v_0 \in V, \forall \varepsilon>0, \exists \delta>0 s.t. \|\V v- \V v_0\| \leq \delta \Rightarrow \|f(\V v) - f(\V v_0)\| \tag{$\ast$}
    \end{equation}
  i.e.
  \[
    f(B_\delta(\V v_0)) \subset B_\varepsilon(f(\V v_0)).
    \]
\end{prop}

\begin{proof}
  Suppose \eqref{eqn:alternate continuity} holds. Given $(v_n)\to v$, must show $(f(v_n))\to f(v)$. Given $\varepsilon>0$, pick $\delta$ such that $(f(B_\delta(v))\subseteq B_\epsilon(f(v))$. Since $(v_n)\to (v)$, exists $N$ such that whenever $n>N$, $\|v_n - v\| < \delta$, i.e. $v_n \in B_\delta(v)$, so $f(v_n) \in B_\epsilon(f(v))$. In other words, whenever $n>N$, $\|f(v_n)-f(v)\| <\varepsilon$.

  Suppose \eqref{eqn:alternate continuity} does not hold. Then exists some $v \in V$ and $\varepsilon>0$ such that there is no $\delta>0$ with $f(B_\delta(v)) \subseteq B_\varepsilon(f(v))$. In particular $f(B_{1/n}(v)) \nsubseteq B_\varepsilon(f(v))$ for all $n$. Pick $v_n \in B_{1/n}(v)$ with $f(v_n) \notin B_\varepsilon(f(v))$. Then $(v_n)\to v$, but $(f(v_n)) \nto f(v)$, since $\|f(v_n) - f(v) \| \geq \varepsilon$ for all $n$. $f$ is not continuous.
\end{proof}

\begin{defi}
  $U \subseteq V$ is an \emph{open subset} of $V$ if for every $u \in U$ there is some $\varepsilon>0$ with $B_\varepsilon(u) \subseteq U$.
\end{defi}

\begin{prop}
  If $f:V\to W$ is continuous and $U\subseteq W$ is open then
  \[
    f^{-1}(U) = \{v \in V: f(v) \in V\}
  \]
  is an open subset of $V$.
\end{prop}

\begin{proof}
  If $v\in f^{-1}(U)$, then $f(v) \in U$. $U$ is open in $W$ so exists $\varepsilon>0$ such that $B_\varepsilon(f(v)) \subseteq U$. $F$ is continuous so exist $\delta>0$ such that $f(B_\delta(v)) \subseteq B_\varepsilon(f(v)) \subseteq U$. So $B_\delta(v) \subseteq f^{-1}(U)$. $f^{-1}(U)$ is open.
\end{proof}

\begin{rmk}
  The converse statement is also true: if for any $U \subseteq W$ open $f^{-1}(U)$ open in $V$, then $f$ is continuous.
\end{rmk}

\begin{eg}\leavevmode
  \begin{enumerate}
  \item $(0,1)$ is open in $\R$.
    \item The function $h(v) = \|v-v_0\|$ is continuous: $h(v) = g \compose f(v)$ where $f(v) = v-v_0$, and $g(v)=\|v\|$. so $B_r(r) = h^{-1}((-r,r))$ is open in $V$.
  \end{enumerate}
\end{eg}

\begin{defi}
  $C\subseteq V$ is a \emph{closed subset} of $V$ if $V\setminus C$ is open in $V$.
\end{defi}

\begin{cor}
  If $f:V\to W$ is continuous and $C\subseteq W$ is closed, then $f^{-1}(C)$ is closed in $V$.
\end{cor}

\begin{proof}
  \[
    f^{-1}(W\setminus C) = V\setminus f^{-1}(C)
  \]
  so if $C\subseteq W$ is closed, $W\setminus C$ is open, so $f^{-1}(W\setminus C) = V\setminus f^{-1}(C)$ is open. Thus $f^{-1}(C)\subseteq V$ is closed.
\end{proof}

\begin{eg}\leavevmode
  \begin{enumerate}
  \item $[a,b]$ is closed in $\R$.
  \item $h(v) = \|v-v_0\|, \cl B_r(v_0) = h^{-1}([0,r])$.
  \item $V, \emptyset$ are both open and closed in $V$.
    \item $\Q \subseteq \R$ is neither open nor closed.
  \end{enumerate}
\end{eg}

\begin{prop}
  \(C \subseteq V\) is closed if and only if for every sequence \((v_n) \to v\) with all \(v_n \in C\), \(v\in C\).
\end{prop}

\begin{proof}
  Suppose \(C\) is closed and \((v_n)\to v \notin C\). Then \(v\in V\setminus C\) which is open, so \(\exists \varepsilon>0\) with \(B_\varepsilon(v) \subseteq V\setminus C\), i.e. \(B_\varepsilon(v) \cap C =\emptyset\). \((v_n)\to v\) so \(\exists N\) such that \(v_n\in B_\varepsilon(v)\) for all \(n>N\). Thus \(v_n \notin C\) for all \(n>N\). In other word, if \((v_n)\to v\), all but finitely many of \(v_n \notin C\).

  Conversely, suppose \(C\) is not closed. Then \(V\setminus C\) is not open so \(\exists c \in V\setminus C\) such that there is no \(\varepsilon>0\) with \(B_\varepsilon(v) \subseteq V\setminus C\). In other words, \(B_\varepsilon(v) \cap V \neq \emptyset\) for all \(\varepsilon>0\). Pick \(v_n\in B_{1/n}(v)\cap C\) for all \(n>0\). Then \(\|v_n-v\|< 1/n\) so \((v_n)\to v\). All \(v_n \in C\) but \(v\in V\setminus C\).
\end{proof}

\begin{eg}
  The set \(X=\{f\in C[0,1]: \forall x, f(x)>0\}\) is not closed with respect to \(\|\cdot\|_1\) or \(\|\cdot\|_\infty\) since \(f_n(x) = \frac{1}{n} \in X\), \((f_n) \to 0\) with respect to either norm but \(0 \notin X\).
\end{eg}

For future use, suppose \(U_\alpha \subseteq V, \alpha\in A\) are open subsets of \(V\). Given \(U = \bigcup_{\alpha\in A}U_\alpha\) and \(f: U \to W\),

\begin{prop}
  If \(f|_{U_\alpha}:U_\alpha \to W\) is continuous for all \(\alpha \in A\), then \(f: U \to W\) is continuous.
\end{prop}

\begin{note}
  The hypothesis that \(U_\alpha\) is open is important. For example, let \(f:\R\to\R, f(x) = 1 \text{ if } x\in\Q, f(x)=0\) otherwise, then \(f|_\Q\) and \(f|_{\R\setminus\Q}\) are both continuous but \(f\) is not.
\end{note}

\begin{proof}
  Suppose \(v_n, v\in U\) with \((v_n)\to v\). Must show \((f(v_n))\to f(v)\). \(v\in U\) so \(v\in U_\alpha\) for some \(\alpha\). \(U_\alpha\) is open so \(\exists \varepsilon >0\) with \(B_\varepsilon(v) \subseteq U_\alpha\), \((v_n)\to v\) so \(\exists N\) with \(v_n \in B_\varepsilon(v)\) for all \(n>N\). Let \(u_i=v_{N+1}\), then \(u_i\in U_\alpha\) and \((u_i)\to v\). Since \(f|_{U_\alpha}\) is continuous, \((f(u_i))\to f(v)\) which implies that \((f(v_n)) \to f(v)\).
\end{proof}

\subsection{Lipschitz Equivalence}

Recall from the introduction of norms that \(\|\cdot\|_1, \|\cdot\|_2, \|\cdot\|_\infty\) on \(\R^n\) all induce the same notion of convergence. We want to generalise this idea.

Suppose \(\|\cdot\|\) and \(\|\cdot\|'\) are two norms on \(V\). Consider
\begin{align*}
  \id_V:(V,\|\cdot\|) &\to (V,\|\cdot\|') \\
  v &\mapsto v
\end{align*}

\begin{prop}
  \(\id_V\) as above is continuous if and only if \(\exists C \in \R\) with \(\|v\|' \leq C \|v\|\) for all \(v \in V\).
\end{prop}

\begin{proof}
  Suppose \(\|v\|' \leq C \|v\|\) for all \(v\). To show \(\id_V\) is continuous, must show \((v_n)\to v\) with respect to \(\|\cdot\|'\) whenever \((v_n)\to v\) with respect to \(\|\cdot\|\).

  If \((v_n)\to v\) with respect ot \(\|\cdot\|\), then \((\|v_n-v\|)\to 0\) so \((C\|v_n-v\|)\to 0\). We know
  \[
    0 \leq \|v_n-v\|' \leq C \|v_n-v\|,
  \]
  so by squeeze rule \(\|v_n-v\|\to 0\). Thus \((v_n)\to v\) with respect to \(\|\cdot\|'\).

  Conversely, suppose \(\id_V\) is continuous. There exist \(\delta>0\) such that
  \[
    \id_V(B_\delta(0, \|\cdot\|) \subseteq B_1(0, \|\cdot\|')
  \]
  Given \(v \in V, v \neq 0\), exists \(K\in\R\) with \(\|Kv\| = \delta/2\) (take \(K=\frac{\delta}{2 \|v\|}\)). Then \(Kv \in B_\delta(0,\|\cdot\|) \Rightarrow Kv\in B_1(0,\|\cdot\|')\), i.e. \(\|Kv\| = \delta/2, \|Kv\|' < 1\), so \(\|Kv\|' \leq \frac{2}{\delta}\|Kv\| \Rightarrow K \|v\|' \leq \frac{2}{\delta}K \|v\|\). Let \(C = \frac{2}{\delta}\).
\end{proof}

\begin{joke}
  \texttt{the joke about a mathematician going for a firefighter interview... Well you should know it by now if you are a mathematician.}
\end{joke}

\begin{defi}
  If \(\|\cdot\|\) and \(\|\cdot\|'\) are two norms on \(V\), they are said to be \emph{Lipschitz equivalent} if
  \begin{align*}
    & \exists C>0 s.t. \forall v \in V, \frac{1}{C}\|v\| \leq \|v\|' \leq C \|v\| \\
    \Longleftrightarrow & \exists C_1,C_2 s.t. \|v\|' \leq C_1 \|v\|, \|v\| \leq C_2 \|v\|' \\
    \Longleftrightarrow & \id_V:(V,\|\cdot\|)\to (V,\|\cdot\|') \text{ and } \id_V:(V,\|\cdot\|')\to (V,\|\cdot\|) \text{are both continuous.}
  \end{align*}
\end{defi}

\begin{cor}
  If \((V, \|\cdot\|)\) and \(V, \|\cdot\|')\) are Lipschitz equivalent, then
  \begin{enumerate}
  \item \((v_n)\to v\) with respect to \(\|\cdot\|\) if and only if with respect to \(\|\cdot\|'\),
  \item \(f:V\to W\) is continuous with respect to \(\|\cdot\|\) if and only if with respect to \(\|\cdot\|'\),
  \item \(F:W\to V\) is continuous with respect to \(\|\cdot\|\) if and only if with respect to \(\|\cdot\|'\).
  \end{enumerate}
\end{cor}

\begin{proof}
  Example proof: if \(f:(V, \|\cdot\|)\to W\) is continuous, \(f:(V,\|\cdot\|')\to W = f:(V, \|\cdot\|)\to W \compose \id_V:(V, \|\cdot\|) \to (V, \|\cdot\|')\).
\end{proof}

\begin{eg}\leavevmode
  \begin{enumerate}
  \item \(V=\R^n\), \(\|v\|_\infty \leq \|v\|_2 \leq \|v\|_1 \leq n \|v\|_\infty\) so all three are Lipschitz equivalent.
  \item \(V=C[0,1]\), \(\id_V:(V, \|\cdot\|_\infty)\to (V,\|\cdot\|_1)\) is continuous but \(\id_V:(V, \|\cdot\|_1)\to (V,\|\cdot\|_\infty)\) is not so not Lipschitz equivalent.
  \end{enumerate}
\end{eg}

\section{Uniform Convergence}

\subsection{Notions of Convergence}

Suppose \(A\subseteq \R, f, f_n: A\to \R\). We say \(f\) is \emph{continuous} if given \(x\in A\) and \(\varepsilon>0\), \(\exists \delta>0\) such that \(|f(x)-f(y)| < \varepsilon\) whenever \(y\in A\) and \(|x-y| < \delta\). We say \(f\) is \emph{bounded} if exists \(M\) such that \(|f(x)| \leq M\) for all \(x\in A\). Define
\begin{itemize}
\item \(C(A) = \{f: A\to \R: f \text{ is continuous}\}\),
\item \(B(A) = \{f:A\to \R: f \text{ is bounded}\}\),
\end{itemize}
which are both vector space.

\begin{eg}
  \(C[0,1] \subseteq B[0,1]\) by Maximum Value Theorem. \(g(x) = \frac{1}{x} \in C(0,1]\) so \(C(0,1] \nsubseteq B(0,1]\).
\end{eg}

\begin{defi}[Pointwise Convergence]
  \((f_n) \to f\) \emph{pointwise} if
  \[
    (f_n(x)) \to f(x) \text{ for all } x\in \R.
    \]
\end{defi}

\begin{defi}
  The \emph{uniform norm} on \(B(A)\) is given by
  \[
    \|f\|_\infty = \sup_{x\in A} |f(x)|.
  \]
\end{defi}

\begin{defi}[Uniform Convergence]
  If \(f, f_n: A\to \R\), we say \(f(x) \to f\) \emph{uniformly on \(A\)} if \((f_n-f) \in B(A)\) for all \(n\) and \((\|f_n-f\|_\infty)\to0\).
\end{defi}
    In other words,
    \begin{itemize}
    \item \((f_n)\to f\) pointwise means: you give me \(x\in A\) and \(\varepsilon >0\), I have to find \(N\) such that \(|f_n(x) - f(x)| < \varepsilon\) whenever \(n> N\). \emph{This \(N\) only has to work for that particular value of \(x\).}
    \item \((f_n)\to f\) uniformly means: you give me \(\varepsilon>0\), I have to find \(N\) such that \(|f_n(x) - f(x)|<\varepsilon\) for all \(x\in A\) and \(n>N\). \emph{Same \(N\) works for all \(x\in A\).}
\end{itemize}

\begin{ex}
  If \((f_n)\to f \) uniformly, then \((f_n)\to f\) pointwise. The converse is false.
\end{ex}

\begin{eg}\leavevmode
  \begin{itemize}
  \item Suppose \(A=\R,f_n(x)=x+\frac{1}{n},f(x)=x\). Then \(f_n(x)-f(x)=\frac{1}{n}\) so \((f_n)\to f\) uniformly.
  \item Let \(g_n(x) = (x+\frac{1}{n})^2, g(x) = x^2\). Then \((g_n)\to g\) pointwise but \(g_n(x) - g(x) = \frac{2x}{n} + \frac{1}{n^2}\) is not even bounded. So \((g_n)\nto g\) uniformly on \(\R\).
  \end{itemize}
\end{eg}

\begin{thm}
  Suppose \(f_n \in C(A)\) for all \(n\) and \((f_n)\to f\) uniformly on \(A\). Then \(f\in C(A)\) as well.
\end{thm}

\begin{slogan}
  The uniform limit of continuous functions is continuous.
\end{slogan}

\begin{proof}
  Suppose \(f_n\) are continuous and \((f_n)\to f\) uniformly. Given \(x\in A\) and \(\varepsilon>0\), must find \(\delta>0\) such that \(|f(x)-f(y)| < \varepsilon\) whenever \(y\in A\) and \(|x-y|<\delta\).

  Since \((f_n)\to f\) uniformly, there exists \(N\) such that \(|f_n(x)-f(x)|< \varepsilon/4\) for all \(x\in A\) and \(n\geq N\). Since \(f_N\) is continuous, exists \(\delta>0\) such that \(|f_N(x)-f_n(y)|<\varepsilon/2\) whenever \(y\in A\) and \(|x-y|<\delta\). Then if \(|x-y| < \delta\),
  \begin{align*}
    |f(x)-f(y)| &\leq |f(x)-f_N(x)| + |f_N(x)-f_N(y)| + |f_N(y)-f(y)| \\
    &\leq \varepsilon/4 + \varepsilon/2 + \varepsilon/4 \\
    &= \varepsilon
  \end{align*}
\end{proof}

\begin{eg}
  Take \(A=[0,1]\),
  \begin{itemize}
  \item \(f_n(x)=x^n, f(x) = 1\text{ if } x=1, f(x)=0 \text{ if } x\neq 1\). Then \((f_n)\to f\) pointwise on \([0,1]\) but \(f_n\in C[0,1], f\notin C[0,1]\) so \((f_n)\nto f\) uniformly on \([0,1]\).
  \item \(g_n(x) = x^n(1-x), g(x) =0\). Then \((g_n)\to g\) uniformly.
\begin{proof}
  given \(\varepsilon>0, 1-\varepsilon<1\) so \((1-\varepsilon)^n\to 0\). Pick \(N\) such that \((1-\varepsilon)^n < \varepsilon\) for all \(n> N\). Then \(|f_n(x)| = |(1-x)x^n| \leq 1\cdot(1-\varepsilon)^n < \varepsilon\) for \(x\in[0,1-\varepsilon]\) and \(|f_n(x)| = |(1-x)x^n| < \varepsilon\cdot1^n=\varepsilon\) for \(x\in (1-\varepsilon,1]\). Thus \(|f_n(x)| < \varepsilon\) for all \(x\in[0,1]\).
\end{proof}
  \end{itemize}
\end{eg}

\begin{rmk}
  Everything I have said so far works fine for \(A\subseteq V, f: A\to W\), where \(V, W\) are normed vector spaces.
\end{rmk}

\begin{joke}
  A mathematician named Cliff measured his room for painting. His wife went off to the paint store and told the counter how much paint she needed. The counter said: ``Thats a lot of paint. Are you sure you want that much?'' To which the wife answered: ``Well my husband is a mathematician. I'm sure he gets the numbers correct.''

  She arrived back home with really a lot of paint. Cliff moved all the paint in the house and suddenly said:

  ``Oh, damn! I measured the volumn instead of the area!''
  \end{joke}

Recall that if \(f\in C[a,b]\) then \(\|f\|_1=\int_a^b|f(x)|dx\). 
\begin{defi}
  \(f_n\) converges \emph{in measure} to \(f\) if \((f_n)\to f\) with respect to \(\|\cdot\|_1\), 
\end{defi}

\begin{lem}
  If \((f_n)\in C[a,b]\) and \((f_n)\to f\) uniformly then \((f_n)\to f\) in measure.
\end{lem}

\begin{proof}
  Given \(\varepsilon>0\), pick \(N\) such that \(|f_n(x)-f(x)|<\varepsilon/2(b-a)\) for all \(x\in [a,b]\). Then
  \begin{align*}
    \|f_n-f\| &= \int_a^b |f_n(x)-f(x)| dx \\
              &\leq \int_a^b\varepsilon/2(b-a) dx \\
              &= (\varepsilon/2(b-a))(b-a) \\
              &= \varepsilon/2.
\end{align*}
\end{proof}

Equivalently, the map \(\id: (C[a,b], \|\cdot\|_\infty)\to (C[a,b], \|\cdot\|_1)\) is continuous.

\begin{eg}
  Let \(A=[0,1]\),
  \begin{enumerate}
  \item \(f(x) =
      \begin{cases}
        nx & x\in [0,1/n] \\ 2-nx & x\in [1/n,2/n] \\0 & x\geq2/n
      \end{cases}
  \)
  Then \((f_n)\to 0\) pointwise and in measure but not uniformly.
\item \(g_n(x) =
    \begin{cases}
      n^2 x & x\in [0,1/n]\\
      2n-n^2 x& x\in[1/n, 2/n] \\
      0 & x \geq 2/n
    \end{cases}
  \)
  Then \((g_n)\to 0\) pointwise but \((g_n)\nto 0\) in measure or uniformly.
  \end{enumerate}
\end{eg}
\end{document}
