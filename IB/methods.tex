\documentclass[a4paper]{article}

\def\ntitle{Methods}
\def\ndate{Michaelmas, 2017 -- 2018}

\ifx \nauthor\undefined
  \def\nauthor{Qiangru Kuang}
\else
\fi

\ifx \ntitle\undefined
  \def\ntitle{Template}
\else
\fi

\ifx \nauthoremail\undefined
  \def\nauthoremail{qk206@cam.ac.uk}
\else
\fi

\ifx \ndate\undefined
  \def\ndate{\today}
\else
\fi

\title{\ntitle}
\author{\nauthor}
\date{\ndate}

%\usepackage{microtype}
\usepackage{mathtools}
\usepackage{amsthm}
\usepackage{stmaryrd}%symbols used so far: \mapsfrom
\usepackage{empheq}
\usepackage{amssymb}
\let\mathbbalt\mathbb
\let\pitchforkold\pitchfork
\usepackage{unicode-math}
\let\mathbb\mathbbalt%reset to original \mathbb
\let\pitchfork\pitchforkold

\usepackage{imakeidx}
\makeindex[intoc]

%to address the problem that Latin modern doesn't have unicode support for setminus
%https://tex.stackexchange.com/a/55205/26707
\AtBeginDocument{\renewcommand*{\setminus}{\mathbin{\backslash}}}
\AtBeginDocument{\renewcommand*{\models}{\vDash}}%for \vDash is same size as \vdash but orginal \models is larger
\AtBeginDocument{\let\Re\relax}
\AtBeginDocument{\let\Im\relax}
\AtBeginDocument{\DeclareMathOperator{\Re}{Re}}
\AtBeginDocument{\DeclareMathOperator{\Im}{Im}}
\AtBeginDocument{\let\div\relax}
\AtBeginDocument{\DeclareMathOperator{\div}{div}}

\usepackage{tikz}
\usetikzlibrary{automata,positioning}
\usepackage{pgfplots}
%some preset styles
\pgfplotsset{compat=1.15}
\pgfplotsset{centre/.append style={axis x line=middle, axis y line=middle, xlabel={$x$}, ylabel={$y$}, axis equal}}
\usepackage{tikz-cd}
\usepackage{graphicx}
\usepackage{newunicodechar}

\usepackage{fancyhdr}

\fancypagestyle{mypagestyle}{
    \fancyhf{}
    \lhead{\emph{\nouppercase{\leftmark}}}
    \rhead{}
    \cfoot{\thepage}
}
\pagestyle{mypagestyle}

\usepackage{titlesec}
\newcommand{\sectionbreak}{\clearpage} % clear page after each section
\usepackage[perpage]{footmisc}
\usepackage{blindtext}

%\reallywidehat
%https://tex.stackexchange.com/a/101136/26707
\usepackage{scalerel,stackengine}
\stackMath
\newcommand\reallywidehat[1]{%
\savestack{\tmpbox}{\stretchto{%
  \scaleto{%
    \scalerel*[\widthof{\ensuremath{#1}}]{\kern-.6pt\bigwedge\kern-.6pt}%
    {\rule[-\textheight/2]{1ex}{\textheight}}%WIDTH-LIMITED BIG WEDGE
  }{\textheight}% 
}{0.5ex}}%
\stackon[1pt]{#1}{\tmpbox}%
}

%\usepackage{braket}
\usepackage{thmtools}%restate theorem
\usepackage{hyperref}

% https://en.wikibooks.org/wiki/LaTeX/Hyperlinks
\hypersetup{
    %bookmarks=true,
    unicode=true,
    pdftitle={\ntitle},
    pdfauthor={\nauthor},
    pdfsubject={Mathematics},
    pdfcreator={\nauthor},
    pdfproducer={\nauthor},
    pdfkeywords={math maths \ntitle},
    colorlinks=true,
    linkcolor={red!50!black},
    citecolor={blue!50!black},
    urlcolor={blue!80!black}
}

\usepackage{cleveref}



% TODO: mdframed often gives bad breaks that cause empty lines. Would like to switch to tcolorbox.
% The current workaround is to set innerbottommargin=0pt.

%\usepackage[theorems]{tcolorbox}





\usepackage[framemethod=tikz]{mdframed}
\mdfdefinestyle{leftbar}{
  %nobreak=true, %dirty hack
  linewidth=1.5pt,
  linecolor=gray,
  hidealllines=true,
  leftline=true,
  leftmargin=0pt,
  innerleftmargin=5pt,
  innerrightmargin=10pt,
  innertopmargin=-5pt,
  % innerbottommargin=5pt, % original
  innerbottommargin=0pt, % temporary hack 
}
%\newmdtheoremenv[style=leftbar]{theorem}{Theorem}[section]
%\newmdtheoremenv[style=leftbar]{proposition}[theorem]{proposition}
%\newmdtheoremenv[style=leftbar]{lemma}[theorem]{Lemma}
%\newmdtheoremenv[style=leftbar]{corollary}[theorem]{corollary}

\newtheorem{theorem}{Theorem}[section]
\newtheorem{proposition}[theorem]{Proposition}
\newtheorem{lemma}[theorem]{Lemma}
\newtheorem{corollary}[theorem]{Corollary}
\newtheorem{axiom}[theorem]{Axiom}
\newtheorem*{axiom*}{Axiom}

\surroundwithmdframed[style=leftbar]{theorem}
\surroundwithmdframed[style=leftbar]{proposition}
\surroundwithmdframed[style=leftbar]{lemma}
\surroundwithmdframed[style=leftbar]{corollary}
\surroundwithmdframed[style=leftbar]{axiom}
\surroundwithmdframed[style=leftbar]{axiom*}

\theoremstyle{definition}

\newtheorem*{definition}{Definition}
\surroundwithmdframed[style=leftbar]{definition}

\newtheorem*{slogan}{Slogan}
\newtheorem*{eg}{Example}
\newtheorem*{ex}{Exercise}
\newtheorem*{remark}{Remark}
\newtheorem*{notation}{Notation}
\newtheorem*{convention}{Convention}
\newtheorem*{assumption}{Assumption}
\newtheorem*{question}{Question}
\newtheorem*{answer}{Answer}
\newtheorem*{note}{Note}
\newtheorem*{application}{Application}

%operator macros

%basic
\DeclareMathOperator{\lcm}{lcm}

%matrix
\DeclareMathOperator{\tr}{tr}
\DeclareMathOperator{\Tr}{Tr}
\DeclareMathOperator{\adj}{adj}

%algebra
\DeclareMathOperator{\Hom}{Hom}
\DeclareMathOperator{\End}{End}
\DeclareMathOperator{\id}{id}
\DeclareMathOperator{\im}{im}
\DeclarePairedDelimiter{\generation}{\langle}{\rangle}

%groups
\DeclareMathOperator{\sym}{Sym}
\DeclareMathOperator{\sgn}{sgn}
\DeclareMathOperator{\inn}{Inn}
\DeclareMathOperator{\aut}{Aut}
\DeclareMathOperator{\GL}{GL}
\DeclareMathOperator{\SL}{SL}
\DeclareMathOperator{\PGL}{PGL}
\DeclareMathOperator{\PSL}{PSL}
\DeclareMathOperator{\SU}{SU}
\DeclareMathOperator{\UU}{U}
\DeclareMathOperator{\SO}{SO}
\DeclareMathOperator{\OO}{O}
\DeclareMathOperator{\PSU}{PSU}

%hyperbolic
\DeclareMathOperator{\sech}{sech}

%field, galois heory
\DeclareMathOperator{\ch}{ch}
\DeclareMathOperator{\gal}{Gal}
\DeclareMathOperator{\emb}{Emb}



%ceiling and floor
%https://tex.stackexchange.com/a/118217/26707
\DeclarePairedDelimiter\ceil{\lceil}{\rceil}
\DeclarePairedDelimiter\floor{\lfloor}{\rfloor}


\DeclarePairedDelimiter{\innerproduct}{\langle}{\rangle}

%\DeclarePairedDelimiterX{\norm}[1]{\lVert}{\rVert}{#1}
\DeclarePairedDelimiter{\norm}{\lVert}{\rVert}



%Dirac notation
%TODO: rewrite for variable number of arguments
\DeclarePairedDelimiterX{\braket}[2]{\langle}{\rangle}{#1 \delimsize\vert #2}
\DeclarePairedDelimiterX{\braketthree}[3]{\langle}{\rangle}{#1 \delimsize\vert #2 \delimsize\vert #3}

\DeclarePairedDelimiter{\bra}{\langle}{\rvert}
\DeclarePairedDelimiter{\ket}{\lvert}{\rangle}




%macros

%general

%divide, not divide
\newcommand*{\divides}{\mid}
\newcommand*{\ndivides}{\nmid}
%vector, i.e. mathbf
%https://tex.stackexchange.com/a/45746/26707
\newcommand*{\V}[1]{{\ensuremath{\symbf{#1}}}}
%closure
\newcommand*{\cl}[1]{\overline{#1}}
%conjugate
\newcommand*{\conj}[1]{\overline{#1}}
%set complement
\newcommand*{\stcomp}[1]{\overline{#1}}
\newcommand*{\compose}{\circ}
\newcommand*{\nto}{\nrightarrow}
\newcommand*{\p}{\partial}
%embed
\newcommand*{\embed}{\hookrightarrow}
%surjection
\newcommand*{\surj}{\twoheadrightarrow}
%power set
\newcommand*{\powerset}{\mathcal{P}}

%matrix
\newcommand*{\matrixring}{\mathcal{M}}

%groups
\newcommand*{\normal}{\trianglelefteq}
%rings
\newcommand*{\ideal}{\trianglelefteq}

%fields
\renewcommand*{\C}{{\mathbb{C}}}
\newcommand*{\R}{{\mathbb{R}}}
\newcommand*{\Q}{{\mathbb{Q}}}
\newcommand*{\Z}{{\mathbb{Z}}}
\newcommand*{\N}{{\mathbb{N}}}
\newcommand*{\F}{{\mathbb{F}}}
%not really but I think this belongs here
\newcommand*{\A}{{\mathbb{A}}}

%asymptotic
\newcommand*{\bigO}{O}
\newcommand*{\smallo}{o}

%probability
\newcommand*{\prob}{\mathbb{P}}
\newcommand*{\E}{\mathbb{E}}

%vector calculus
\newcommand*{\gradient}{\V \nabla}
\newcommand*{\divergence}{\gradient \cdot}
\newcommand*{\curl}{\gradient \cdot}

%logic
\newcommand*{\yields}{\vdash}
\newcommand*{\nyields}{\nvdash}

%differential geometry
\renewcommand*{\H}{\mathbb{H}}
\newcommand*{\transversal}{\pitchfork}
\renewcommand{\d}{\mathrm{d}} % exterior derivative

%number theory
\newcommand*{\legendre}[2]{\genfrac{(}{)}{}{}{#1}{#2}}%Legendre symbol


\newcommand*\p{\partial}
\renewcommand*\L{\mathcal{L}}

\begin{document}
\maketitle

\tableofcontents

\section*{Self-adjoint ODEs}

\section{Fourier Series}

\subsection{Periodic functions}

A function $f(t)$ is \emph{periodic} with period $T$ if $f(t+T) = f(t)$. Consider $A \sin\omega t$, where $A$ is the amplitude, $\omega$ is the frequency and $2\pi/\omega=T$ is the period. Sines and cosines have an orthogonality property:
\begin{align*}
  \cos(A\pm B) &= \cos A\cos B \mp \sin A\sin B \\
  \cos A \cos B &= \frac{1}{2}[\cos(A-B)+\cos(A+B)] \\
  \sin A \sin B &= \frac{1}{2}[\cos(A-B)-\cos(A+B)]
\end{align*}

Consider $\sin\frac{n\pi x}{L}, \sin\frac{m\pi x}{L}$ where $n,m$ are non-negative integers. These functions are periodic with period $2L$.
\begin{align*}
  ss_{m,n} &:= \int_0^{2L}\sin\frac{m\pi x}{L}\sin\frac{n\pi x}{L} dx \\
  &= \frac{1}{2}\int_0^{2L}\cos\frac{(m-n)\pi x}{L} - \cos\frac{(m+n)\pi x}{L} dx \\
          &= \frac{L}{2\pi}\Bigg[\frac{\sin\frac{(m-n)\pi x}{L}}{m-n} - \frac{\sin\frac{(m+n)\pi x}{L}}{m+n}\Bigg]_0^{2L}\\
           &=
             \begin{cases}
               0 & \text{if } m\neq n \\
               L & \text{if } m=n\neq 0 \\
               0 & \text{if } m=n=0
             \end{cases}
\end{align*}
Thus
\[
  ss_{m,n}=
  \begin{cases}
    L \delta_{mn} &\text{if } n\neq 0 \\
    0 &\text{if } m=0 \text{ or } n=0
  \end{cases}
\]
Similarly
\[
  cc_{m,n} := \int_0^{2L}\cos\frac{m\pi x}{L}\cos\frac{n\pi x}{L} dx =
  \begin{cases}
    2L &\text{if } m=n=0 \\
    L\delta_{mn} &\text{otherwise}
  \end{cases}
\]
Finally,
\begin{align*}
  cs_{mn} &:= \int_0^{2L}\cos\frac{m\pi x}{L}\sin\frac{n\pi x}{L} dx \\
          &= \frac{1}{2}\int_0^{2L}\frac{\sin (m+n)\pi x}{L} - \frac{\sin (m-n)\pi x}{L} dx \\
          &= 0
\end{align*}

By analogy with vectors (these integrals are \emph{inner products}), $\sin \frac{n\pi x}{L}$ and $\cos\frac{n\pi x}{L}$ are said to be \emph{orthogonal} on the interval $[0,2L]$. Actually they constitute an \emph{orthogonal basis}, i.e. it is possible to represent an arbitray (but sufficiently well-behaved) function in terms of an infinite series (Fourier series) formed as a sum of sines and cosines.

\subsection{Definition of Fouries Series}

Any ``well-behaved'' (to be defined later) periodic function $f(x)$ with period $2L$ can be written as a Fourier series:
\begin{equation}
  \label{eqn:fourier coefficients}
  \frac{f(x_+)+f(x_-)}{2} = \frac{1}{2}a_0 + \sum_{n=1}^\infty \Big(a_n\cos\frac{n\pi x}{L} +b_n\sin\frac{n\pi x}{L}\Big)
  \tag{$\ast$}
\end{equation}
where $a_n$ and $b_n$ are the \emph{Fourier coefficients} and $f(x_+)$ and $f(x_-)$ are the right limit approaching from above and the left limit approaching from below respectively.

\begin{enumerate}
\item If $f(x)$ is continuous at $x_c$, the LHS is just $f(x)$.
  \item if $f(x)$ has a \emph{bounded} discontinuity at $x_d$, i.e., $|f(x_d^-)-f(x_d^+)|$ is non-zero but finite, the Fourier series tends to the mean value of the two limits.
\end{enumerate}

\subsection{Coefficient Construction}

Multiply RHS of equation~\eqref{eqn:fourier coefficients} by $\sin \frac{m\pi x}{L}$, integrate over $[0, 2\pi]$. Assume we can exchange summation and integration,
\begin{align*}
  \int_0^{2L} f(x)\sin\frac{m\pi x}{L} dx &= \int_0^{2L} \Bigg[ \frac{1}{2}a_0 + \sum_{n=1}^\infty \Big(a_n\cos\frac{n\pi x}{L} +b_n\sin\frac{n\pi x}{L}\Big) \Bigg] \sin\frac{m\pi x}{L} dx \\
    &= 0 + \sum_{n=1}^\infty a_n \int_0^{2L} \cos\frac{n\pi x}{L} \sin\frac{m\pi x}{L} dx +  \sum_{n=1}^\infty b_n \int_0^{2L} \sin\frac{n\pi x}{L} \sin\frac{m\pi x}{L} dx\\
  &= 0 + 0 + L b_m
    \end{align*}
Thus
\[
  b_m = \frac{1}{L}\int_0^{2L} f(x) \sin\frac{m\pi x}{L} dx
\]

Similarly, multiply by $\cos \frac{m\pi x}{L}$ and integrate (include $m=0$), we get
\[
  \int_0^{2L} f(x)\cos\frac{m\pi x}{L} dx = \int_0^{2L} \Bigg[ \frac{1}{2}a_0 + \sum_{n=1}^\infty \Big(a_n\cos\frac{n\pi x}{L} +b_n\sin\frac{n\pi x}{L}\Big) \Bigg] \cos\frac{m\pi x}{L} dx
\]
The first term is non-zero only when $m=0$. Therefore
\[
  \frac{a_0}{2} \cdot 2L = \int_0^{2L} f(x) dx
\]
so
\[
\frac{a_0}{2} = \frac{1}{2L}\int_0^{2L} f(x) dx.
\]
The second term gives
\[
  a_m = \frac{1}{L} \int_0^{2L} f(x) \cos\frac{m\pi x}{L} dx.
\]

The range of integration is one period so it is also permissible to choose $-L$ and $L$ as the limit of integration. A particularly neat case is when $L=\pi$:
\begin{align*}
  a_m &= \frac{1}{\pi} \int_{-\pi}^\pi f(x) \cos mx dx, m\geq0 \\
  b_m &= \frac{1}{\pi} \int_{-\pi}^\pi f(x) \sin mx dx, m\geq1
\end{align*}

\subsection{Dirichlet Conditions}

If $f(x)$ is a periodic function with period $2L$ such that
\begin{enumerate}
\item it is absolutely integrable, i.e.$\int_0^{2L}|f(x)| dx$ is well-defined,
\item it has a finite number of extrema (i.e. maxima and minima) in $[0,2L]$,
  \item it has a finite number of \emph{bounded} discontinuities in $[0,2L]$,
\end{enumerate}
then the Fourier series converges to $f(x)$ for all points where $f(x)$ is continuous and at points $x_d$ where $f(x)$ is discontinuous, the series converges to the average value of the left and right limit, i.e. $(f(x_d^+)+f(x_d^-))/2$.

These conditions are satisfied if the function is of \emph{bounded variation}.

\begin{rmk}
  The Fourier series converges but not necessarily uniformly converges. This gives rises to some weird behaviour.
\end{rmk}

\subsubsection{Smoothness and Order of Fourier Coefficients}

If the $p$th derivative is the lowest derivative which is discontinuous somewhere (including the endpoints), then the Fourier coefficients are $O(n^{-(p+1)})$ as $n \to \infty$.

For example, if a function has a bouned discontinuity, the $0$th derivative is discontinuous: coefficients are of order $\frac{1}{n}$ as $n \to \infty$.

\begin{eg}\leavevmode
  \begin{enumerate}
  \item The sawtooth function $f(x) = x$ for $-L \leq x < L$.
    \begin{center}
      \begin{tikzpicture}
        \begin{axis}[
          xtick={-4,4},
          xticklabels={$-2L$,$2L$},
          ytick={-2,2},
          yticklabels={$-L$,$L$},
          axis lines=center,
          axis equal image,
          xmin=-8,
          xmax=8,
          ymin=-3,
          ymax=3,
          xlabel=$x$,
          ylabel=$f(x)$]
          \addplot[domain=-2:2,color=red]{x};
          \addplot[domain=2:6,color=red]{x-4};
          \addplot[domain=-6:-2,color=red]{x+4};
        \end{axis}
      \end{tikzpicture}
    \end{center}
    The function is odd so
\begin{align*}
  a_m &= \frac{1}{L}\int_{-L}^L x \cos\frac{m\pi x}{L} dx = 0 \\ 
  b_m  &= \frac{1}{L}\int_{-L}^L x \sin\frac{m\pi x}{L} dx \\
      &= \frac{1}{L}\Big[x \frac{L}{m\pi}(-\cos\frac{m\pi x}{L})\Big]_{-L}^L - \frac{1}{L}\frac{L}{m\pi} \int_{-L}^L -\cos\frac{m\pi x}{L} dx \\
      &= \frac{1}{m\pi}\Big( L(-\cos m\pi) - (-L)(-\cos m\pi) \Big) + \frac{1}{m\pi} \Big[ \frac{L}{m\pi} \sin\frac{m\pi x}{L} \Big]_{-L}^L \\
  &= \frac{2L}{m\pi}(-1)^{m+1}
\end{align*}

So
\[
  \frac{f(x_+)+f(x_-)}{2} = \frac{2L}{\pi}\Big[\sin\frac{\pi x}{L} - \frac{1}{2}\sin\frac{2\pi x}{L} + \cdots \Big].
\]
and we observe that
\begin{enumerate}
\item $f_N(x) = \sum_{n=1}^N b_n\sin\frac{n\pi x}{L} \to f(x)$ almost everywhere but the convergence is \emph{not} uniform.
\item Persistent overshooting at $x=L$: Gibbs phenomenon.
\item $f(L) = 0$, the average of right and left hand limit.
\item Coefficients are $O(\frac{1}{n})$ as $n\to \infty$.
\end{enumerate}

\item The integral of the sawtooth function, $f(x) = \frac{x^2}{2}$ for $-L \leq x < L$.
    \begin{center}
      \begin{tikzpicture}
        \begin{axis}[
          xtick={-4,4},
          xticklabels={$-2L$,$2L$},
          ytick={-2,2},
          yticklabels={$-L/2$,$L/2$},
          axis lines=center,
          axis equal image,
          xmin=-8,
          xmax=8,
          ymin=-3,
          ymax=3,
          xlabel=$x$,
          ylabel=$f(x)$]
          \addplot[domain=-2:2,color=red]{x^2/2};
          \addplot[domain=2:6,color=red]{(x-4)^2/2};
          \addplot[domain=-6:-2,color=red]{(x+4)^2/2};
        \end{axis}
      \end{tikzpicture}
    \end{center}
  Exercise: $f(x) = L^2 \big[ \frac{1}{6} + 2\sum_{n=1}^\infty \frac{(-1)^n}{(n\pi)^2} \cos\frac{n\pi x}{L} \big]$.

  Note at $x=0$,
  \begin{align*}
    & 0 = L^2 \Big[\frac{1}{6} + 2\sum_{n=1}^\infty \frac{(-1)^n}{(n\pi)^2} \Big] \\
    \Rightarrow & \frac{\pi^2}{12} = \sum_{n=1}^\infty\frac{(-1)^{n+1}}{n^2}
  \end{align*}
  \end{enumerate}
\end{eg}

\section{Properties of Fourier Series}

\subsection{Integration \& Differentiation}

\subsubsection{Integration\protect\footnote{Don't panic: it's always OK.}}

Fourier series \emph{can} be integrated term by term. Suppose $f(x)$ with period $2L$ has a Fourier series (so it satisfies the Direichlet condition):
\[
  \frac{f(x_+)+f(x_-)}{2} = \frac{1}{2}a_0 + \sum_{n=1}^\infty \Big(a_n\cos\frac{n\pi x}{L} +b_n\sin\frac{n\pi x}{L}\Big).
\]
Consider
\begin{align*}
  F(x) & = \int_{-L}^x f(u) du \\
       &= \frac{a_0(x+L)}{2} + \sum_{n=1}^\infty \frac{a_n L}{n\pi}\sin\frac{n\pi x}{L} + \sum_{n=1}^\infty \frac{b_n L}{n\pi} \Big[(-1)^n - \cos\frac{n\pi x}{L} \Big] \\
       &= \frac{a_0 L}{2} + L \sum_{n=1}^\infty (-1)^n \frac{b_n}{n\pi} - L \sum_{n=1}^\infty \frac{b_n}{n\pi} \cos\frac{n\pi x}{L} + L \sum_{n=1}^\infty \frac{a_n-(-1)^n a_0}{n\pi}\sin\frac{n\pi x}{L}
\end{align*}

If $a_n, b_n$ are Fourier coefficients, then series involving $\frac{a_n}{n}, \frac{b_n}{n}$ (multiplied by sine or cosine) must also converge, so they are part of a Fourier series. Fourier series of $f(x)$ exists so $b_n$ is at least of $O(\frac{1}{n})$ as $n\to\infty$. Thus $\frac{b_n}{n}$ is at least of order $O(\frac{1}{n^2})$ as $n\to \infty$ and so by comparison test with $\sum \frac{M}{n^2}$ the second term converges, so $F(x)$ has a Fourier series.

\begin{note}
  Integration smoothes.
  The proof relies on discontinuity being bounded ($f(x)$ satisfies the Dirichlet condition).
\end{note}

\subsubsection{Differentiation\protect\footnote{Do panic, it doesn't always work!}}

Let $f(x)$ be a periodic function with period $2$ such that
\[
  f(x) =
  \begin{cases}
    1 & 0 < x < 1\\
    -1 & -1 < x < 0 
  \end{cases}
\]
This is an odd function so
\begin{align*}
  a_m &= 0 \\
  b_m &= -\int_{-1}^0 \sin m\pi x dx + \int_{-1}^0 \sin m\pi x dx \\
  &= \frac{\cos m\pi x}{m\pi} \Big|_{-1}^0 - \frac{\cos m\pi x}{m\pi} \Big|_0^{-1} \\
  &= \frac{1}{m\pi} \big(1- (-1)^m - (-1)^m + 1 \big) \\
  & =
  \begin{cases}
    \frac{4}{m\pi} & m \text{ is odd} \\
    0 & m \text{ is even}
  \end{cases}
\end{align*}
Thus
\[
  \frac{f(x_+)+f(x_-)}{2} = \frac{4}{\pi} \sum_{n=1}^\infty \frac{\sin(2n-1)\pi x}{2n-1}.
  \]
Apply differentiation rules
\[
  f'(x) = 4 \sum_{n=1}^\infty \cos(2n-1)\pi x
\]
This is clearly divergent even though $f'(x) = 0$ for all $x \neq 0$. The extra factor of $2n-1$ is the problem, which is related to the discontinuity. $f'(x)$ \emph{does not} satisfy the Dirichlet condition.

\begin{note}
  Intuitively, this behaviour can be explained by noticing that Dirac delta function is the derivative of Heaviside function.
\end{note}

\section{Differentiation Under Circumstances}

Differentiation of Fourier series under certain circumstances.

\begin{eg}
  Assume \(f(x)\) is continuous and is extended as a \(2L\)-periodic function, piecewise continuously differentiable on \((-L,L)\). Let \(g(x)=\frac{df}{dx}\). \(g(x)\) satisfies the Dirichlet condition as it has at worst a finite number of bounded discontinuities.
  \begin{align*}
    f(x) &= \frac{a_0}{2} + \sum_{n=1}^{\infty} a_n \cos \frac{n\pi x}{L}+ b_n \sin \frac{n\pi x}{L} \\
    \frac{g(x_+)+g(x_-)}{2} &=\frac{A_0}{2} + \sum_{n=1}^{\infty} A_n \cos \frac{n\pi x}{L} + B_n \sin \frac{n\pi x}{L}
  \end{align*}
  Then
  \begin{align*}
    A_0 &= \frac{1}{L}\int_{0}^{2L}g(x) dx \\
        &= \frac{f(2L)-f(0)}{L} \\
        &= 0
  \end{align*}
  by periodicity.
  \begin{align*}
    A_n &=\frac{1}{L} \int_{0}^{2L}\frac{df}{dx}\cos \frac{n\pi x}{L}dx \\
        &= \frac{1}{L}\Big[f(x)\cos \frac{n\pi x}{L} \Big] \Big|_0^{2L} + \frac{n\pi}{L^2} \int_{0}^{2L} f(x)\sin \frac{n\pi x}{L} dx \\
        &= 0 + \frac{n\pi b_n}{L}
      \end{align*}
      Similarly, \(B_n=-\frac{n\pi a_n}{L}\).

      This reduces differentiation to multiplication by \(\pm \frac{n\pi}{L}\).
      \end{eg}

\subsection{Altenate Representation: Complex Form}

Recall
\begin{align*}
  \cos \frac{n\pi x }{L} &= \frac{1}{2} (e^{i\frac{n\pi x}{L}} + e^{-i\frac{n\pi x}{L}}) \\
  \sin &= \frac{1}{2i} (e^{i\frac{n\pi x}{L}} - e^{-i\frac{n\pi x}{L}})
\end{align*}
so
\begin{align*}
  \frac{f(x_+)+f(x_-)}{2} &= \frac{a_0}{2} + \sum_{n=1}^{\infty} \frac{a_n}{2} (e^{i \frac{n\pi x}{L}} + e^{-i \frac{n \pi x}{L}}) - \sum_{n=1}^{\infty} \frac{b_n}{2} (e^{i \frac{n\pi x}{L}} - e^{-i \frac{n \pi x}{L}})\\
  &= \frac{a_0}{2} + \sum_{n=1}^{\infty} \big( \frac{a_n-i b_n}{2} e^{i \frac{n\pi x}{L}} \big) + \sum_{n=1}^{\infty} \big( \frac{a_n+i b_n}{2} e^{-i \frac{n\pi x}{L}} \big) \\
    &= \sum_{n=-\infty}^{\infty} c_n e^{i\frac{n\pi x}{L}}
\end{align*}
with
\begin{align*}
  c_0 &= \frac{a_0}{2}, \\
  c_n &= \frac{a_n-ib_n}{2}, n>0, \\
  c_n &= \frac{a_n+ib_n}{2}, n<0.
\end{align*}

Note that \(c_n^* = c_{-n}\). It can be easily shown that complex exponentials are orthogonal:
\begin{align*}
  \int_{0}^{2L} e^{i\frac{n\pi x}{L}} e^{-i\frac{m\pi x}{L}}dx &= \int_{0}^{2L}\cos \frac{(n-m)\pi x}{L}dx + i \int_{0 }^{2L}\sin \frac{(n-m)\pi x}{L} dx \\
  &= 2L \delta_{n,m} + 0
\end{align*}
so
\[
  c_m = \frac{1}{2L} \int_{0}^{2L}f(x)e^{-i\frac{m\pi x}{L}} dx = \frac{1}{2L} \int_{0}^{2L} \Big( \sum_{n=-\infty}^{\infty} c_n e^{i\frac{n\pi x}{L}} \Big) e^{i\frac{-m\pi x}{L}} dx.
\]

Now assume \(g(x) = \frac{df}{dx} = \sum_{n=-\infty}^{\infty} c_n e^{i\frac{n\pi x}{L}}\). Then
\begin{align*}
  c_n &= \frac{1}{2L}\int_{0}^{2L} \frac{df}{dx }e^{-\frac{in\pi x}{L}}dx \\
    &= \frac{1}{2L} \big[ f(x) e^{-i\frac{n\pi x}{L}} \big]_0^{2L} + \frac{in\pi}{2L^2} \int_{0}^{2L}f(x) e^{i\frac{-n\pi x}{L}} dx \\
    &= \frac{in\pi}{L}c_n
\end{align*}
by periodicity.



\subsection{Half-range Series}

Consider a function defined \emph{only} on \(0 \leq x \leq L\). There are two possible ways to extend this function to a \(2L\)-periodic function that can be represented as a Fourier series.

\subsubsection{Odd function: Fourier Sine Series}

\(f(x)\) can be extended as an \emph{odd} function \(f(x)=-f(-x)\) on \(-L \leq x \leq L\) and then extended as a \(2L\)-periodic function. In this case \(a_n=0\) we can define the \emph{Fourier sine series}:
\[
  \frac{f(x_+)+f(x_-)}{2} = \sum_{n=1}^{\infty}b_n \sin \frac{n\pi x}{L}
\]
where
\[
  b_n = \frac{2}{L} \int_{0}^{L} f(x) \sin \frac{n\pi x}{L} dx.
\]
Note the range of integration.

\begin{eg}[Sawtooth function]
  
\end{eg}

\subsubsection{Even function: Fourier Cosine Series}

\(f(x)\) can also be extended as an even function on \(-L\leq x\leq L\), i.e. \(f(x) = f(-x)\) and then extended as a \(2L\)-periodic function. \(b_n=0\) for all \(n\). The Fourier cosine series is
\[
  \frac{f(x)+f(x)}{2} = \frac{a_0}{2} + \sum_{n-1}^{\infty}a_n \cos \frac{n\pi x}{L}
\]
where
\[
  a_n = \frac{2}{L}\int_{0}^{L}f(x)\cos \frac{n\pi x}{L}.
\]

\subsection{Parseval's Theorem}

``Energy'' of a periodic signal is often of interest, i.e. \(E = \int_{0}^{2L}f^2(x) dx \). Consider the general case
\[
  f(x) = \sum_{n=-\infty}^{\infty}c_n e^{\frac{in\pi x}{L}}, g(x) = \sum_{m=-\infty}^{\infty}d_m e^{\frac{im\pi x}{L}};
\]
\begin{align*}
  \int_{0}^{2L} f(x) g(x) dx &= \sum_{n=-\infty}^{\infty} \sum_{m=-\infty}^{\infty} c_n d_m \int_{0}^{2L} \exp \Big [\frac{i\pi x}{L} (n+m) \Big] dx \\
  &= \sum_{n=-\infty}^{\infty} \sum_{m=\infty}^{\infty} c_n d_m (2L \delta_{n,-m}) \\
  &= 2L \sum_{n=-\infty}^{\infty} c_n d_{-n} \\
  &= 2L \sum_{n=-\infty}^{\infty} c_n d_n*
  \end{align*}
so if \(g=f\),
\[
  \int_{0}^{2L} f^2(x) dx = 2L \sum_{n=-\infty}^{\infty}|c_n|^2 = L \Big[ \frac{a_0^2}{2} + \sum_{n=1}^{\infty}(a_n^2+b_n^2) \Big]
\]

\begin{eg}[Sawtooth function]
  Remember \(f(x)=x\) for \(-L\leq x\leq L\).
  \[
    b_n = \frac{2L}{m\pi}(-1)^{m+1}
  \]
  so
  \[
    \int_{-L}^{L} x^2 dx = \frac{2L^3}{3} = L \sum_{m=1}^{\infty}\frac{4L^2}{m^2\pi^2}
  \]
  so
  \[
    \sum_{m=1}^{\infty}\frac{1}{m^2} = \frac{\pi^2}{6}.
  \]
\end{eg}

\begin{ex}
  From the Fourier series of \(x^2/2\) show that
  \[
    \sum_{m=1}^{\infty}\frac{1}{m^4} = \frac{\pi^4}{90}.
  \]
\end{ex}

\section{Sturm-Liouville Theory}

\subsection{Second Order ODEs}

Consider a general second order ordinary partical differential equation
\[
  \L y(x) = \alpha(x) \frac{d^2y}{dx^2} + \beta(x) \frac{dy}{dx} + \gamma(x) y = f(x).
\]

\(\alpha,\beta,\gamma\) are continuous, with \(\alpha\) non-zero except perhaps at a finite number of isolated points. \(f(x)\) is bounded, defined on \(a\leq x\leq b\) (\(a\) or \(b\) may be infinity).

The \emph{homogeneous} equation \(\mathcal{L}y = 0\) has two linearly independent solutions \(y_1,y_2\) and the complementary function is \[
  y_c=Ay_1+By_2.
\]
\emph{Inhomogeneous} or \emph{forced} equation \(\L y = f(x)\) where \(f\) is the forcing, has a particular integral \(y_p(x)\). The general solution is \(y=y_c(x)+y_p(x)\) where \(A,B\) are determined in a problem by applying condition.
\end{document}
