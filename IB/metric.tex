\documentclass[a4paper]{article}

\def\ntitle{Metric \& Topological Spaces}
\ifx \nauthor\undefined
  \def\nauthor{Qiangru Kuang}
\else
\fi

\ifx \ntitle\undefined
  \def\ntitle{Template}
\else
\fi

\ifx \nauthoremail\undefined
  \def\nauthoremail{qk206@cam.ac.uk}
\else
\fi

\ifx \ndate\undefined
  \def\ndate{\today}
\else
\fi

\title{\ntitle}
\author{\nauthor}
\date{\ndate}

%\usepackage{microtype}
\usepackage{mathtools}
\usepackage{amsthm}
\usepackage{stmaryrd}%symbols used so far: \mapsfrom
\usepackage{empheq}
\usepackage{amssymb}
\let\mathbbalt\mathbb
\let\pitchforkold\pitchfork
\usepackage{unicode-math}
\let\mathbb\mathbbalt%reset to original \mathbb
\let\pitchfork\pitchforkold

\usepackage{imakeidx}
\makeindex[intoc]

%to address the problem that Latin modern doesn't have unicode support for setminus
%https://tex.stackexchange.com/a/55205/26707
\AtBeginDocument{\renewcommand*{\setminus}{\mathbin{\backslash}}}
\AtBeginDocument{\renewcommand*{\models}{\vDash}}%for \vDash is same size as \vdash but orginal \models is larger
\AtBeginDocument{\let\Re\relax}
\AtBeginDocument{\let\Im\relax}
\AtBeginDocument{\DeclareMathOperator{\Re}{Re}}
\AtBeginDocument{\DeclareMathOperator{\Im}{Im}}
\AtBeginDocument{\let\div\relax}
\AtBeginDocument{\DeclareMathOperator{\div}{div}}

\usepackage{tikz}
\usetikzlibrary{automata,positioning}
\usepackage{pgfplots}
%some preset styles
\pgfplotsset{compat=1.15}
\pgfplotsset{centre/.append style={axis x line=middle, axis y line=middle, xlabel={$x$}, ylabel={$y$}, axis equal}}
\usepackage{tikz-cd}
\usepackage{graphicx}
\usepackage{newunicodechar}

\usepackage{fancyhdr}

\fancypagestyle{mypagestyle}{
    \fancyhf{}
    \lhead{\emph{\nouppercase{\leftmark}}}
    \rhead{}
    \cfoot{\thepage}
}
\pagestyle{mypagestyle}

\usepackage{titlesec}
\newcommand{\sectionbreak}{\clearpage} % clear page after each section
\usepackage[perpage]{footmisc}
\usepackage{blindtext}

%\reallywidehat
%https://tex.stackexchange.com/a/101136/26707
\usepackage{scalerel,stackengine}
\stackMath
\newcommand\reallywidehat[1]{%
\savestack{\tmpbox}{\stretchto{%
  \scaleto{%
    \scalerel*[\widthof{\ensuremath{#1}}]{\kern-.6pt\bigwedge\kern-.6pt}%
    {\rule[-\textheight/2]{1ex}{\textheight}}%WIDTH-LIMITED BIG WEDGE
  }{\textheight}% 
}{0.5ex}}%
\stackon[1pt]{#1}{\tmpbox}%
}

%\usepackage{braket}
\usepackage{thmtools}%restate theorem
\usepackage{hyperref}

% https://en.wikibooks.org/wiki/LaTeX/Hyperlinks
\hypersetup{
    %bookmarks=true,
    unicode=true,
    pdftitle={\ntitle},
    pdfauthor={\nauthor},
    pdfsubject={Mathematics},
    pdfcreator={\nauthor},
    pdfproducer={\nauthor},
    pdfkeywords={math maths \ntitle},
    colorlinks=true,
    linkcolor={red!50!black},
    citecolor={blue!50!black},
    urlcolor={blue!80!black}
}

\usepackage{cleveref}



% TODO: mdframed often gives bad breaks that cause empty lines. Would like to switch to tcolorbox.
% The current workaround is to set innerbottommargin=0pt.

%\usepackage[theorems]{tcolorbox}





\usepackage[framemethod=tikz]{mdframed}
\mdfdefinestyle{leftbar}{
  %nobreak=true, %dirty hack
  linewidth=1.5pt,
  linecolor=gray,
  hidealllines=true,
  leftline=true,
  leftmargin=0pt,
  innerleftmargin=5pt,
  innerrightmargin=10pt,
  innertopmargin=-5pt,
  % innerbottommargin=5pt, % original
  innerbottommargin=0pt, % temporary hack 
}
%\newmdtheoremenv[style=leftbar]{theorem}{Theorem}[section]
%\newmdtheoremenv[style=leftbar]{proposition}[theorem]{proposition}
%\newmdtheoremenv[style=leftbar]{lemma}[theorem]{Lemma}
%\newmdtheoremenv[style=leftbar]{corollary}[theorem]{corollary}

\newtheorem{theorem}{Theorem}[section]
\newtheorem{proposition}[theorem]{Proposition}
\newtheorem{lemma}[theorem]{Lemma}
\newtheorem{corollary}[theorem]{Corollary}
\newtheorem{axiom}[theorem]{Axiom}
\newtheorem*{axiom*}{Axiom}

\surroundwithmdframed[style=leftbar]{theorem}
\surroundwithmdframed[style=leftbar]{proposition}
\surroundwithmdframed[style=leftbar]{lemma}
\surroundwithmdframed[style=leftbar]{corollary}
\surroundwithmdframed[style=leftbar]{axiom}
\surroundwithmdframed[style=leftbar]{axiom*}

\theoremstyle{definition}

\newtheorem*{definition}{Definition}
\surroundwithmdframed[style=leftbar]{definition}

\newtheorem*{slogan}{Slogan}
\newtheorem*{eg}{Example}
\newtheorem*{ex}{Exercise}
\newtheorem*{remark}{Remark}
\newtheorem*{notation}{Notation}
\newtheorem*{convention}{Convention}
\newtheorem*{assumption}{Assumption}
\newtheorem*{question}{Question}
\newtheorem*{answer}{Answer}
\newtheorem*{note}{Note}
\newtheorem*{application}{Application}

%operator macros

%basic
\DeclareMathOperator{\lcm}{lcm}

%matrix
\DeclareMathOperator{\tr}{tr}
\DeclareMathOperator{\Tr}{Tr}
\DeclareMathOperator{\adj}{adj}

%algebra
\DeclareMathOperator{\Hom}{Hom}
\DeclareMathOperator{\End}{End}
\DeclareMathOperator{\id}{id}
\DeclareMathOperator{\im}{im}
\DeclarePairedDelimiter{\generation}{\langle}{\rangle}

%groups
\DeclareMathOperator{\sym}{Sym}
\DeclareMathOperator{\sgn}{sgn}
\DeclareMathOperator{\inn}{Inn}
\DeclareMathOperator{\aut}{Aut}
\DeclareMathOperator{\GL}{GL}
\DeclareMathOperator{\SL}{SL}
\DeclareMathOperator{\PGL}{PGL}
\DeclareMathOperator{\PSL}{PSL}
\DeclareMathOperator{\SU}{SU}
\DeclareMathOperator{\UU}{U}
\DeclareMathOperator{\SO}{SO}
\DeclareMathOperator{\OO}{O}
\DeclareMathOperator{\PSU}{PSU}

%hyperbolic
\DeclareMathOperator{\sech}{sech}

%field, galois heory
\DeclareMathOperator{\ch}{ch}
\DeclareMathOperator{\gal}{Gal}
\DeclareMathOperator{\emb}{Emb}



%ceiling and floor
%https://tex.stackexchange.com/a/118217/26707
\DeclarePairedDelimiter\ceil{\lceil}{\rceil}
\DeclarePairedDelimiter\floor{\lfloor}{\rfloor}


\DeclarePairedDelimiter{\innerproduct}{\langle}{\rangle}

%\DeclarePairedDelimiterX{\norm}[1]{\lVert}{\rVert}{#1}
\DeclarePairedDelimiter{\norm}{\lVert}{\rVert}



%Dirac notation
%TODO: rewrite for variable number of arguments
\DeclarePairedDelimiterX{\braket}[2]{\langle}{\rangle}{#1 \delimsize\vert #2}
\DeclarePairedDelimiterX{\braketthree}[3]{\langle}{\rangle}{#1 \delimsize\vert #2 \delimsize\vert #3}

\DeclarePairedDelimiter{\bra}{\langle}{\rvert}
\DeclarePairedDelimiter{\ket}{\lvert}{\rangle}




%macros

%general

%divide, not divide
\newcommand*{\divides}{\mid}
\newcommand*{\ndivides}{\nmid}
%vector, i.e. mathbf
%https://tex.stackexchange.com/a/45746/26707
\newcommand*{\V}[1]{{\ensuremath{\symbf{#1}}}}
%closure
\newcommand*{\cl}[1]{\overline{#1}}
%conjugate
\newcommand*{\conj}[1]{\overline{#1}}
%set complement
\newcommand*{\stcomp}[1]{\overline{#1}}
\newcommand*{\compose}{\circ}
\newcommand*{\nto}{\nrightarrow}
\newcommand*{\p}{\partial}
%embed
\newcommand*{\embed}{\hookrightarrow}
%surjection
\newcommand*{\surj}{\twoheadrightarrow}
%power set
\newcommand*{\powerset}{\mathcal{P}}

%matrix
\newcommand*{\matrixring}{\mathcal{M}}

%groups
\newcommand*{\normal}{\trianglelefteq}
%rings
\newcommand*{\ideal}{\trianglelefteq}

%fields
\renewcommand*{\C}{{\mathbb{C}}}
\newcommand*{\R}{{\mathbb{R}}}
\newcommand*{\Q}{{\mathbb{Q}}}
\newcommand*{\Z}{{\mathbb{Z}}}
\newcommand*{\N}{{\mathbb{N}}}
\newcommand*{\F}{{\mathbb{F}}}
%not really but I think this belongs here
\newcommand*{\A}{{\mathbb{A}}}

%asymptotic
\newcommand*{\bigO}{O}
\newcommand*{\smallo}{o}

%probability
\newcommand*{\prob}{\mathbb{P}}
\newcommand*{\E}{\mathbb{E}}

%vector calculus
\newcommand*{\gradient}{\V \nabla}
\newcommand*{\divergence}{\gradient \cdot}
\newcommand*{\curl}{\gradient \cdot}

%logic
\newcommand*{\yields}{\vdash}
\newcommand*{\nyields}{\nvdash}

%differential geometry
\renewcommand*{\H}{\mathbb{H}}
\newcommand*{\transversal}{\pitchfork}
\renewcommand{\d}{\mathrm{d}} % exterior derivative

%number theory
\newcommand*{\legendre}[2]{\genfrac{(}{)}{}{}{#1}{#2}}%Legendre symbol


\begin{document}
\maketitle
\tableofcontents

\section{Metric Spaces}

\subsection{Metric}

Euclidean space $\mathbb{R}^n$ equipped with standart Euclidean inner-product $(·,·)$, given $\V x, \V y ∈ \mathbb{R}^n$ with coordinates $x_i, y_i$ respectively,

\[
(\V x, \V y) = \V x · \V y = \sum_{i=1}^{n}x_i y_i
\]

Euclidean norm $||\V x|| = (\V x, \V x)^{1/2}$ which is the length of $\V x$

Euclidean distance:
\[
d_2: \mathbb{R}^n × \mathbb{R}^n → \mathbb{R}, \quad d_2(\V x, \V y) = ||\V x - \V y|| = (\sum_{i=1}^n (x_i - y_i))^{1/2}
\]

This is an example of a metric.

\begin{defi}
	A metric space $(X, d)$ consists of a set $X$ and a function $d: X × X → \mathbb{R}$ s.t.

	\begin{enumerate}
		\item $d(P, Q) ≥ 0$ with equality iff $P = Q$,
		\item $∀P, Q ∈ X, d(P, Q) = d(Q, P)$,
		\item $∀P, Q, R ∈ X, d(P, Q) + d(Q, R) ≥ d(P, R)$. (triangle inequality)
	\end{enumerate}
\end{defi}

(iii) says, for the Euclidean metrix, for any triangle (possibly degenerate) with vertices $P, Q, R$, sum of lengths of two sides is larger than the length of the third side.

\begin{prop}
	The Euclidean distance function $d_2$ on $\mathbb{R}^n$ is a metric.
\end{prop}

\begin{proof}
	(i) and (ii) are obvious. For (iii) we use Cauchy-schwarz inequality
	\[
		(\sum x_i y_i)^2 ≤ (\sum x_i²)(\sum y_i²)
	\]
	or in inner product notation
	\[
		(\V x, \V y)² ≤ ||\V x||²||\V y||²
	\]
	
	We may take $P = 0 ∈ \mathbb{R}^n$ to be the origin, $Q$ has position vector $\V x$ wrt $P$ and $R$ has position vector $\V y$ wrt $Q$, and has position vector $\V x + \V y$ wrt $P$ :

	\begin{align*}
		||\V x + \V y||² &= (\V x + \V y, \V x + \V y) \\
                         &= ||\V x+\V y, \V x+\V y|| \\
                         &= ||\V x||^2 + 2(\V x,\V y) + ||\V y||^2 \\
                         &\leq ||\V x||^2 + 2||\V x|| ||\V y|| + ||\V y||^2 \\
                         &= (||\V x|| + ||\V y||)^2
	\end{align*}
\end{proof}

\begin{lem}[Cauchy-Schwarz]
  \[
    \forall \V x,\V y \in \mathbb{R}^n, (\V x,\V y)^2 \leq ||\V x||^2 \times ||\V y||^2
  \]
\end{lem}

\begin{proof}
  The quadratic polynomial on real variable $\lambda$
  \[
    (\lambda \V x + \V y, \lambda \V x + \V y) = ||\V x||^2 \lambda^2 + 2(\V x,\V y)\lambda + ||\V y||^2
  \]
  is positive semi-definite. Thus by considering the determinant the result follows.
\end{proof}

More example on metric spaces:
\begin{eg}
  \begin{enumerate}
  \item $X = \mathbb{R}^n, d_1(\V x,\V y) = \sum_{i=1}^n |x_i-y_i|$ and $d_\infty(\V x,\V y) = \max_i |x_i-y_i|$ are both metrics.
  \item for any set $X$, define the \emph{discrete metric}
    \[
      d(x, y) =
      \begin{cases}
        1 & \text{if } x \neq y \\
        0 & \text{if } x = y
      \end{cases}
    \]
  \item $X= C[0,1]$, the set of real continuous functions on $[0,1]$. We can define $d_1, d_2,\ldots d_\infty$ by $d_1(f,g) = \int_0^1|f-g|, d_2(f,g) = (\int_0^1(f-g)^2)^{1/2} \cdot d_\infty(f,g) = \sup|f-g|$.
  \item British Rail metric: consider $\mathbb{R}^n$ with Euclidean metric, and let $0$ denote the origin. Define ew metric on $\mathbb{R}^n$ by
    \[
      \rho(P,Q) =
      \begin{cases}
        d(P,0) + d(0,Q) &\text{if } P\neq Q \\
        0 &\text{if } P=Q
      \end{cases}
    \]
  \end{enumerate}
\end{eg}

Some metrics in fact satisfy a stronger triangle inequality:

\begin{defi}
  A metric space $(X,d)$ is called an \emph{ultra-metric} if $d$ satisfies
  \[
    \forall P,Q,R \in X, d(P,R) \leq \max\{d(P,Q), d(Q,R)\}
  \]
  
\end{defi}

\begin{ex}
  $X=\mathbb{Z}$, $p$ prime. The \emph{$p$-adic metric} is defined by
  \[
    d(m,n) =
    \begin{cases}
      0 & m = n \\
      \frac{1}{p^r} & m \neq n \: \text{where } r = \max\{s\in\mathbb{N}: p^s|(m-n)\}
    \end{cases}
  \]
  Claim $d$ is an ultra-metric.
\end{ex}

This extends to a $p$-adic metric on $\mathbb{Q}$. For $x\neq y \in \mathbb{Q}$, write $x-y = p^r \frac{m}{n}, r\in\mathbb{Z}$, $m, n$ coprime ot $p$ and define $d(x,y) = \frac{1}{p^r}$.

\begin{ex}
  The sequence $a_n = 1+p+p^2+\cdot p^n$ is a convergent sequence in $(\mathbb{Q},d_p)$ with limit $a = \frac{1}{1-p}$ since $p^n | (a_n-a) = \frac{p^n}{p-1}$ for all $n$.
\end{ex}

\begin{defi}
  Two metrics $\rho_1, \rho_2$ on a set $X$ are \emph{Lipshitz equivalent} if
  \[
    \exists 0 < \lambda_1 \leq \lambda_2 \in \mathbb{R} s.t. \lambda_1 \rho_1 \leq \rho_2 \leq \lambda_2\rho2
  \]
  
\end{defi}

This is an equivalence relation.

\begin{rmk}
  For metrics $d_1, d_2, d_\infty$ on $\mathbb{R}^n$, one can show that
  \[
    d_1 \geq d_2 \geq d_\infty \geq \frac{d_2}{\sqrt n} \geq \frac{d_1}{n}
  \]
  and so are Lipshitz equivalent.
\end{rmk}

\begin{prop}
  $d_1,d_\infty$ one $C[0,1]$ are not Lipshitz equivalent.
\end{prop}

\begin{proof}
  For $n \geq 2$, let $f_n \in C[0,1]$ be given by the function as shown. Thus $d_1(f_n,0) =$ area of triangle $\to 0$ as while $d_\infty(f_n,0) \to \sqrt n = \infty$ as $n \to \infty$.
\end{proof}

\begin{center}
\begin{tikzpicture}
  \draw[->] (-.5,0) -- (3,0);
  \draw[->] (0,-.5) -- (0,3);
  \draw (0,0) -- (1,2) -- (2,0);
  \draw (1,0) node[below] {$\frac{1}{n}$} (2,0) node[below] {$\frac{2}{n}$} (0,2) node[left] {$\sqrt n$};
\end{tikzpicture}
\end{center}

\begin{ex}
  Show that $d_2(f_n,0) = \sqrt{\frac{2}{3}}$ for all $n$ is not equivalent to either.
\end{ex}

\subsection{Open balls and open sets}

Let $(X,d)$ be a metric space, $P\in X, \delta > 0$. The \emph{open ball}
\[
  B_d(P,\delta) = \{Q\in X: d(P,Q) < \delta\}
\]
Often we omit the meric and write $B(P,\delta)$ or $B_\delta(P)$.

\begin{defi}
  A subset $U \subseteq X$ of a metric space $(X,d)$ is called \emph{open} if $\forall P \in U, \exists \delta>0 s.t. B(P,\delta)\subseteq U$, i.e. an open subset is just a union of open balls.

  A subset $F\subseteq X$ is closed if $X\setminus F$ is open.
\end{defi}

Open balls are open and closed balls are closed.

\begin{lem}\leavevmode
  \begin{enumerate}
  \item $\emptyset, X \subseteq X$ are open subsets of $(X,d)$.
  \item If $\{U_i\}_{i\in I}$ are open sets then so is $\bigcup_{i\in I} U_i$.
    \item If $U_1, U_2$ are open then so is $U_1 \cap U_2$.
  \end{enumerate}
\end{lem}

\begin{defi}
  Give $P$ a point of $(X,d)$, an \emph{open neighbourhood (nbhd)} of $P$ is an open set $U \ni P$.
\end{defi}

\subsection{Limits and continuity}

Suppose $(x_n)$ is a sequence of points in a metric space $(X,d)$. We say $x_n \to x\in X$ (converges to limit $x$) if $d(x_n,x) \to 0$ as $n \to \infty$. Equivalently, $\forall \varepsilon>0, \exists N \forall n\geq N x_n \in B(x, \varepsilon)$.

\begin{rmk}
  $x_n \to x$ in $(X,d)$ iff for any open neighbourhood $U \ni x \exists N \forall n\geq N x_n \in U$. 
\end{rmk}
\end{document}
