\documentclass[a4paper]{article}

\def\ntitle{Principle of Quantum Mechanics Summary}

\ifx \nauthor\undefined
  \def\nauthor{Qiangru Kuang}
\else
\fi

\ifx \ntitle\undefined
  \def\ntitle{Template}
\else
\fi

\ifx \nauthoremail\undefined
  \def\nauthoremail{qk206@cam.ac.uk}
\else
\fi

\ifx \ndate\undefined
  \def\ndate{\today}
\else
\fi

\title{\ntitle}
\author{\nauthor}
\date{\ndate}

%\usepackage{microtype}
\usepackage{mathtools}
\usepackage{amsthm}
\usepackage{stmaryrd}%symbols used so far: \mapsfrom
\usepackage{empheq}
\usepackage{amssymb}
\let\mathbbalt\mathbb
\let\pitchforkold\pitchfork
\usepackage{unicode-math}
\let\mathbb\mathbbalt%reset to original \mathbb
\let\pitchfork\pitchforkold

\usepackage{imakeidx}
\makeindex[intoc]

%to address the problem that Latin modern doesn't have unicode support for setminus
%https://tex.stackexchange.com/a/55205/26707
\AtBeginDocument{\renewcommand*{\setminus}{\mathbin{\backslash}}}
\AtBeginDocument{\renewcommand*{\models}{\vDash}}%for \vDash is same size as \vdash but orginal \models is larger
\AtBeginDocument{\let\Re\relax}
\AtBeginDocument{\let\Im\relax}
\AtBeginDocument{\DeclareMathOperator{\Re}{Re}}
\AtBeginDocument{\DeclareMathOperator{\Im}{Im}}
\AtBeginDocument{\let\div\relax}
\AtBeginDocument{\DeclareMathOperator{\div}{div}}

\usepackage{tikz}
\usetikzlibrary{automata,positioning}
\usepackage{pgfplots}
%some preset styles
\pgfplotsset{compat=1.15}
\pgfplotsset{centre/.append style={axis x line=middle, axis y line=middle, xlabel={$x$}, ylabel={$y$}, axis equal}}
\usepackage{tikz-cd}
\usepackage{graphicx}
\usepackage{newunicodechar}

\usepackage{fancyhdr}

\fancypagestyle{mypagestyle}{
    \fancyhf{}
    \lhead{\emph{\nouppercase{\leftmark}}}
    \rhead{}
    \cfoot{\thepage}
}
\pagestyle{mypagestyle}

\usepackage{titlesec}
\newcommand{\sectionbreak}{\clearpage} % clear page after each section
\usepackage[perpage]{footmisc}
\usepackage{blindtext}

%\reallywidehat
%https://tex.stackexchange.com/a/101136/26707
\usepackage{scalerel,stackengine}
\stackMath
\newcommand\reallywidehat[1]{%
\savestack{\tmpbox}{\stretchto{%
  \scaleto{%
    \scalerel*[\widthof{\ensuremath{#1}}]{\kern-.6pt\bigwedge\kern-.6pt}%
    {\rule[-\textheight/2]{1ex}{\textheight}}%WIDTH-LIMITED BIG WEDGE
  }{\textheight}% 
}{0.5ex}}%
\stackon[1pt]{#1}{\tmpbox}%
}

%\usepackage{braket}
\usepackage{thmtools}%restate theorem
\usepackage{hyperref}

% https://en.wikibooks.org/wiki/LaTeX/Hyperlinks
\hypersetup{
    %bookmarks=true,
    unicode=true,
    pdftitle={\ntitle},
    pdfauthor={\nauthor},
    pdfsubject={Mathematics},
    pdfcreator={\nauthor},
    pdfproducer={\nauthor},
    pdfkeywords={math maths \ntitle},
    colorlinks=true,
    linkcolor={red!50!black},
    citecolor={blue!50!black},
    urlcolor={blue!80!black}
}

\usepackage{cleveref}



% TODO: mdframed often gives bad breaks that cause empty lines. Would like to switch to tcolorbox.
% The current workaround is to set innerbottommargin=0pt.

%\usepackage[theorems]{tcolorbox}





\usepackage[framemethod=tikz]{mdframed}
\mdfdefinestyle{leftbar}{
  %nobreak=true, %dirty hack
  linewidth=1.5pt,
  linecolor=gray,
  hidealllines=true,
  leftline=true,
  leftmargin=0pt,
  innerleftmargin=5pt,
  innerrightmargin=10pt,
  innertopmargin=-5pt,
  % innerbottommargin=5pt, % original
  innerbottommargin=0pt, % temporary hack 
}
%\newmdtheoremenv[style=leftbar]{theorem}{Theorem}[section]
%\newmdtheoremenv[style=leftbar]{proposition}[theorem]{proposition}
%\newmdtheoremenv[style=leftbar]{lemma}[theorem]{Lemma}
%\newmdtheoremenv[style=leftbar]{corollary}[theorem]{corollary}

\newtheorem{theorem}{Theorem}[section]
\newtheorem{proposition}[theorem]{Proposition}
\newtheorem{lemma}[theorem]{Lemma}
\newtheorem{corollary}[theorem]{Corollary}
\newtheorem{axiom}[theorem]{Axiom}
\newtheorem*{axiom*}{Axiom}

\surroundwithmdframed[style=leftbar]{theorem}
\surroundwithmdframed[style=leftbar]{proposition}
\surroundwithmdframed[style=leftbar]{lemma}
\surroundwithmdframed[style=leftbar]{corollary}
\surroundwithmdframed[style=leftbar]{axiom}
\surroundwithmdframed[style=leftbar]{axiom*}

\theoremstyle{definition}

\newtheorem*{definition}{Definition}
\surroundwithmdframed[style=leftbar]{definition}

\newtheorem*{slogan}{Slogan}
\newtheorem*{eg}{Example}
\newtheorem*{ex}{Exercise}
\newtheorem*{remark}{Remark}
\newtheorem*{notation}{Notation}
\newtheorem*{convention}{Convention}
\newtheorem*{assumption}{Assumption}
\newtheorem*{question}{Question}
\newtheorem*{answer}{Answer}
\newtheorem*{note}{Note}
\newtheorem*{application}{Application}

%operator macros

%basic
\DeclareMathOperator{\lcm}{lcm}

%matrix
\DeclareMathOperator{\tr}{tr}
\DeclareMathOperator{\Tr}{Tr}
\DeclareMathOperator{\adj}{adj}

%algebra
\DeclareMathOperator{\Hom}{Hom}
\DeclareMathOperator{\End}{End}
\DeclareMathOperator{\id}{id}
\DeclareMathOperator{\im}{im}
\DeclarePairedDelimiter{\generation}{\langle}{\rangle}

%groups
\DeclareMathOperator{\sym}{Sym}
\DeclareMathOperator{\sgn}{sgn}
\DeclareMathOperator{\inn}{Inn}
\DeclareMathOperator{\aut}{Aut}
\DeclareMathOperator{\GL}{GL}
\DeclareMathOperator{\SL}{SL}
\DeclareMathOperator{\PGL}{PGL}
\DeclareMathOperator{\PSL}{PSL}
\DeclareMathOperator{\SU}{SU}
\DeclareMathOperator{\UU}{U}
\DeclareMathOperator{\SO}{SO}
\DeclareMathOperator{\OO}{O}
\DeclareMathOperator{\PSU}{PSU}

%hyperbolic
\DeclareMathOperator{\sech}{sech}

%field, galois heory
\DeclareMathOperator{\ch}{ch}
\DeclareMathOperator{\gal}{Gal}
\DeclareMathOperator{\emb}{Emb}



%ceiling and floor
%https://tex.stackexchange.com/a/118217/26707
\DeclarePairedDelimiter\ceil{\lceil}{\rceil}
\DeclarePairedDelimiter\floor{\lfloor}{\rfloor}


\DeclarePairedDelimiter{\innerproduct}{\langle}{\rangle}

%\DeclarePairedDelimiterX{\norm}[1]{\lVert}{\rVert}{#1}
\DeclarePairedDelimiter{\norm}{\lVert}{\rVert}



%Dirac notation
%TODO: rewrite for variable number of arguments
\DeclarePairedDelimiterX{\braket}[2]{\langle}{\rangle}{#1 \delimsize\vert #2}
\DeclarePairedDelimiterX{\braketthree}[3]{\langle}{\rangle}{#1 \delimsize\vert #2 \delimsize\vert #3}

\DeclarePairedDelimiter{\bra}{\langle}{\rvert}
\DeclarePairedDelimiter{\ket}{\lvert}{\rangle}




%macros

%general

%divide, not divide
\newcommand*{\divides}{\mid}
\newcommand*{\ndivides}{\nmid}
%vector, i.e. mathbf
%https://tex.stackexchange.com/a/45746/26707
\newcommand*{\V}[1]{{\ensuremath{\symbf{#1}}}}
%closure
\newcommand*{\cl}[1]{\overline{#1}}
%conjugate
\newcommand*{\conj}[1]{\overline{#1}}
%set complement
\newcommand*{\stcomp}[1]{\overline{#1}}
\newcommand*{\compose}{\circ}
\newcommand*{\nto}{\nrightarrow}
\newcommand*{\p}{\partial}
%embed
\newcommand*{\embed}{\hookrightarrow}
%surjection
\newcommand*{\surj}{\twoheadrightarrow}
%power set
\newcommand*{\powerset}{\mathcal{P}}

%matrix
\newcommand*{\matrixring}{\mathcal{M}}

%groups
\newcommand*{\normal}{\trianglelefteq}
%rings
\newcommand*{\ideal}{\trianglelefteq}

%fields
\renewcommand*{\C}{{\mathbb{C}}}
\newcommand*{\R}{{\mathbb{R}}}
\newcommand*{\Q}{{\mathbb{Q}}}
\newcommand*{\Z}{{\mathbb{Z}}}
\newcommand*{\N}{{\mathbb{N}}}
\newcommand*{\F}{{\mathbb{F}}}
%not really but I think this belongs here
\newcommand*{\A}{{\mathbb{A}}}

%asymptotic
\newcommand*{\bigO}{O}
\newcommand*{\smallo}{o}

%probability
\newcommand*{\prob}{\mathbb{P}}
\newcommand*{\E}{\mathbb{E}}

%vector calculus
\newcommand*{\gradient}{\V \nabla}
\newcommand*{\divergence}{\gradient \cdot}
\newcommand*{\curl}{\gradient \cdot}

%logic
\newcommand*{\yields}{\vdash}
\newcommand*{\nyields}{\nvdash}

%differential geometry
\renewcommand*{\H}{\mathbb{H}}
\newcommand*{\transversal}{\pitchfork}
\renewcommand{\d}{\mathrm{d}} % exterior derivative

%number theory
\newcommand*{\legendre}[2]{\genfrac{(}{)}{}{}{#1}{#2}}%Legendre symbol


\renewcommand*{\H}{\mathcal{H}}

\newcommand*{\bk}{\braket}
\newcommand*{\bkt}{\braketthree}

\theoremstyle{definition}
\newtheorem*{postulate}{Postulate}


%stretch table row height
\renewcommand{\arraystretch}{1.5}



\usepackage{pdflscape}

\begin{document}
\maketitle

\section{Introduction}

\begin{defi}[Hilbert space]
  A complete inner product space.
\end{defi}

\begin{defi}[Linear Operator]
  A linear map \(A: \H\to \H\).
\end{defi}

\begin{defi}[(Continuous) dual space]
  \(\H^* = \mathcal{B}(\H,\C)\).
\end{defi}

Note \(\H \cong \H^*\).

\begin{notation}[Dirac notation]
  Write an element of \(\H\) as \(|\psi\rangle\), a \emph{ket} and an element of \(\H^*\) as \(\langle \phi|\), a \emph{bra}.
\end{notation}

Given an orthonormal basis \(\{|e_1\rangle,\dots,|e_n\rangle\}\) of \(\H\), any vector can be expressed as
\[
  |u\rangle = \sum_{i=1}^{n}u_i|e_i\rangle.
\]
The inner product is then
\[
  \langle v| u\rangle = \sum_{i,j=1}^{n}\conj v_j u_i \langle e_j|e_i\rangle = \sum_{i=0}^{n}\conj v_a u_a.
\]
If the basis elements are indexed by a continuous family then we simply change summation above to integration:
\[
  |\psi\rangle = \int_{ }^{ } \psi(a) |a\rangle da 
\]
where an basis element \(|a\rangle\) is normalised using Dirac \(\delta\)-function: \(\langle a'|a\rangle = \delta(a'-a)\).

For example, if we use the \emph{position basis} \(\{|x\rangle\}_{x\in\R}\) then a state can be expressed as
\[
  |\psi\rangle = \int_{\R}^{ } \psi(x')|x'\rangle dx'
\]
where
\[
  \psi(x) = \langle x| \psi\rangle 
\]
is the \emph{position space wavefunction}.

\begin{defi}[Commutator]
  Given two operators \(A\) and \(B\), the \emph{commutator} \([A,B] = AB- BA\) measures the degree to which they are incompatible.
\end{defi}

\begin{defi}
  \(|\psi\rangle \in \H\) is an \emph{eigenstate} of an operator \(A\) if \(A|\psi\rangle = a_\psi|\psi\rangle\). \(a_\psi\in\C\) is the \emph{eigenvalue}.

  The set of all eigenvalues of \(A\) is called the \emph{spectrum} of \(A\).

  The number of linearly independent eigenstates having the same eigenvalue is call the \emph{degeneracy} of the eigenvalue.
\end{defi}

\begin{defi}[Adjoint]
  The adjoint \(A^\dag\) of an operator \(A\) is such that
  \[
    \langle\phi|A^\dag|\psi\rangle = \conj{\langle\psi|A|\phi\rangle} 
  \] 
  for all \(|\phi\rangle, |\psi\rangle \in \H\).
\end{defi}

\begin{defi}[Self-adjoint operator]
  If \(Q = Q^\dag\) then \(Q\) is \emph{self-adjoint}, or \emph{Hermitian}.
\end{defi}

For a self-adjoint operator \(Q\), the set of eigenstates \(\{|n\rangle\}\) with eigenvales \(\{q_n\}\) form an orthonormal basis so we can write
\[
  Q = \sum_{n}^{ }q_n |n\rangle \langle n|.
\]
From this we can define functions of operators:
\[
  f(Q) := \sum_{n}^{ }f(q_n) |n\rangle \langle n|.
\]

\begin{postulate}[Postulates of Quantum Mechanics]\leavevmode
  \begin{enumerate}
  \item Measurement: if a state is prepared to be in some general state\
    \[
      \ket \psi = \sum_ac_a\ket \phi_a
    \]
    then the \emph{probability} that the measurement will yield an outcome corresponding to some state \(\ket \phi_b\) is
    \[
      \P(\ket \psi\to \ket\phi_b) = \frac{|\bk{\phi_b}{\psi}|^2}{\bk{\psi}{\psi}\bk{\phi_b}{\phi_b}} = |c_b|^2 \frac{\bk{\phi_b}{\phi_b}}{\bk{\psi}{\psi}}.
    \]
    Note that sum of probabilities of all outcomes is \(1\):
    \[
      \sum_b\P(\ket\psi \to \ket\phi_b) = \sum_b\frac{|c_b|^2\bk{\phi_b}{\phi_b}}{\bk{\psi}{\psi}} = \frac{\bk{\psi}{\psi}}{\bk{\psi}{\psi}} = 1.
    \]
    In the case where the states are \emph{normalised}, i.e. \(\bk{\psi}{\psi} = 1\) then this simplifies to
    \[
      \P(\ket\psi \to \ket\phi_b) = |\bk{\phi_b}{\psi}|^2
    \]
    where \(\bk{\phi_b}{\psi} \in\C\) is the \emph{probability amplitude}.

    The states are represented by elements of the projective Hilbert space \(\mathbb{P}\H\), i.e. two vectors in \(\H\) differing by a constant factor (known as phase factor) represent the same state.
  \item Observable: observable quantities are represented by self-adjoint operators. Upon measurement by an operator \(A\), a state is certain to return the definite value \(a\) if and only if it is an eigenstate of \(A\) with eigenvalue \(a\).
  \end{enumerate}
\end{postulate}

\begin{prop}[Generalised uncertainty relation]
  Given two Hermitian operators \(A\) and \(B\),
  \[
    (\Delta A)_\psi^2(\Delta B)_\psi^2 \geq \frac{1}{4}|\langle[A,B]\rangle_\psi|^2
  \]
\end{prop}

\begin{proof}
  For any Hermtian operators \(A\) and \(B\), for all \(\lambda \in \R\), by positive-definiteness of inner-product we have
  \[
    0 \leq \|(A+i\lambda B)|\psi\rangle\| = \langle A^2\rangle_\psi + \lambda\langle[A,B]\rangle_\psi + \lambda^2\langle B^2\rangle_\psi.
  \]
  Treat this as a quadratic in \(\lambda\) and notice the discriminant to get
  \[
    \langle A^2\rangle_\psi \langle B^2\rangle_\psi \geq \frac{1}{4}|\langle[A,B]\rangle_\psi|^2.
  \]
  Finally use the fact that \(A'=A-\langle A\rangle_\psi\) is also Hermitian and that \(\langle A'^2\rangle_\psi = (\Delta A)_\psi^2\).
\end{proof}

\section{Transformations and Symmetries}

A transformation is an operator \(U:\H\to \H, |\psi\rangle\mapsto |\psi'\rangle = U|\psi\rangle\).

\begin{prop}
  \(U\) as above is unitary, i.e. \(U^{-1} = U^\dag\).
\end{prop}

\begin{proof}
  After the tranformation we are expected to still find the state somewhere. Thus
  \[
    1 = \langle \psi|\psi\rangle = \langle \psi'|\psi' \rangle = \langle \psi |U^\dag U|\psi \rangle \text{ for all } |\psi\rangle \in \H.
  \]

  polarisation identity
\end{proof}

An operator gives a homomorphism from the group of all such transformations to the group of operators of all such transformations, i.e.\(U: g\mapsto U(g)\) and \(U(g_2)\compose U(g_1) = U(g_2\cdot g_2\).

\begin{defi}[Generator]
  Given a \emph{continuous} transformation \(U(\theta)\), write
  \[
    U(\delta\theta) = 1 - i \delta\theta T + \bigO(\delta\theta^2)\footnotemark
  \]
  \footnotetext{The \(-i\) is a historical convention.}
\end{defi}

By unitarity of \(U\), \(T=T^\dag\). A transformation can be seen as the limit of
\[
  U(\theta) = \lim_{N\to \infty}\Big( 1- i \frac{\theta}{N}T \Big)^N = e^{-i\theta T}.
\]
It follows that
\[
  i \frac{\partial |\psi\rangle}{\partial \theta} = T |\psi\rangle.
\]

Instead of view a transformation as ``changing'' the state, we can also see it as ``changing'' the operator by \emph{similarity transform}:
\[
  A \mapsto A' = U^\dag A U = U^{-1} A U.
\]
Note that
\[
  [A',B'] = U^{-1}[A,B]U
\]
and for a continuous transformation,
\[
  U^{-1}(\delta\theta) A U(\delta\theta) = A + i\delta\theta [T,A] + \bigO(\delta\theta^2)
\]
so the rate of change of states are given by the generator while the rate of change of operators is given by the commutator of the generator with the operator.

\subsection{Continuous Transformations}

Translation and rotation are summarised in Table~\ref{tab:continuous}. Here are some additional discussion.

\subsubsection{Translation}

Suppose \(\ket{\V x}\) is a position eigenstate with eigenvalue \(\V x\), i.e.
\[
  \V X\ket{\V x} = \V x\ket{\V x},
\]
representing a particle definitely located at \(\V x\). Then
\begin{align*}
  \V X U(\V a)\ket{\V x} &= ([\V X,U(\V a)]+U(\V a) \V X)\ket{\V x} \\
                         &= (U(\V a)\V a + U(\V a)\V X)\ket{\V x} \\
                         &= (\V x+ \V a)U(\V a)\ket{\V x}
\end{align*}
Thus \(U(\V a)\ket{\V x}\) is an eigenstate of \(\V X\) with eigenvalue \(\V x+ \V a\). Write
\[
  U(\V a)\ket{\V x} = c\ket{\V x+ \V a}
\]
since they represent the same state. Take inner product with \(\bra{\V x'}\),
\begin{align*}
  \bk{\V x'}{\V x} &= c \delta^3(\V x'-\V x-\V a) = c \bk{\V x'}{\V x+ \V a} = \bkt{\V x'}{U(\V a)}{\V x} \\
                  &= \big( U(\V a)^{-1}\ket{\V x'} \big)^\dag \ket{\V x} = \Big( \frac{1}{c}\ket{\V x'-\V a} \Big)^\dag \ket{\V x} = \frac{1}{\conj c}\delta^3(\V x'- \V a - \V x)
\end{align*}
so \(|c|^2=1\).

Similarly for a state in position basis
\[
  \ket{\psi} = \int_{\R^3}^{ } \bk{\V x}{\psi}\ket{\V x} d^3x = \int_{\R^3}^{ } \psi(\V x)\ket{\V x} d^3x,
\]
the wavefunction of the translated space \(\psi'(\V x)\) is

\begin{align*}
  \psi'(\V x) &= \bkt{\V x}{U(\V a)}{\psi} = \bkt*{\V x}{U(\V a) \int_{\R^3}^{} \psi(\V x') d^3x'}{\V x'} \\
              &= \big(U^{-1}(\V a)\ket{\V x} \big)^\dag \int_{\R^3}^{ } \psi(\V x') \ket{\V x} d^3 x' \\
              &= \bkt*{\V x- \V a}{\int_{\R^3}^{ } \psi(\V x') d^3x}{\V x'} \\
              &= \int_{\R^3}^{ } \psi(\V x')\bk{\V x - \V a}{\V x'} dx' \\
              &= \psi(\V x- \V a)
\end{align*}
where we used the unitarity \(U^{-1} = U^\dag\) in \(U^\dag(\V a)\ket{\V x} = \ket{\V x- \V a}\).

In particular, for an infinitesimal translation \(\delta\V a\), we have
\[
  \psi'(\V x) - \psi(\V x) = -\delta\V a\cdot \nabla \psi(\V x)
\]
as well as
\[
  \psi'(\V x) - \psi(\V x) = \bkt*{\V x}{-\frac{i}{\hbar}\delta\V \alpha\cdot \V P}{\psi} = -\frac{i}{\hbar}\delta\V a\cdot \bkt{\V x}{\V P}{\psi}
\]
so the momentum operator \(\V P\) acts in the position representation as
\[
  \bkt{\V x}{\V P}{\psi} = -i\hbar\nabla\psi(\V x).
\]

For example, for a state \(\ket{\V p}\) satisfying \(\V P \ket{\V p} = \V p \ket{\V p}\),
\[
  \psi_{\V p}(\V x- \V a) = \bk{\V x- \V a}{\V p} = \bkt{\V x}{U(\V a)}{\V p} = \bkt{\V x}{e^{-i\V a\cdot\V P/\hbar}}{\V p} =  e^{-i\V a\cdot \V p/\hbar}\bk{\V x}{\V p} = e^{-i\V a\cdot \V p/\hbar}\psi_{\V p}(\V x).
\]
Thus the position wavefunction for momentum eigenstates is
\[
  \psi_{\V p}(\V x) = \frac{1}{(2\pi \hbar)^{3/2}} e^{i\V p\cdot \V x/\hbar}.
\]

\subsubsection{Rotation}

A anticlockwise rotation around the axis \(\hat \V{\alpha}\) by an amount \(|\V \alpha\) is a linear transformation
\begin{align*}
  \V R(\V \alpha): \R^3 &\to \R^3 \\
  \V v &\mapsto \V v' = \V R(\V \alpha)\V v
\end{align*}
that obeys
\[
  \norm{\V v} = \norm{\V v'}, \, \det{\V R(\V \alpha)} = 1.
\]

For an infinitesimal rotation,
\[
  \V v' = \V v + \delta\V \alpha\times \V v + \bigO(\delta\V\alpha^2).
\]

\pagestyle{empty}
\begin{landscape}
  
\begin{table}[htbp]
  \centering
  \begin{tabular}{|c|p{8cm}|p{8cm}|}
    \hline
    & Translation & Rotation \\ \hline
    Governing equation & \(U^{-1}(\V a)\V X U(\V a) = \V X + \V a\) where \(\V X\) is position operator & \(U^{-1}(\V \alpha)\V V U(\V\alpha) = \V R(\V\alpha)\V V\) for any vector \(\V V\) \\ \hline
    Generator & \(U(\delta\V a) = 1- \frac{i}{\hbar}\delta\V a \cdot \V P + \bigO(\delta a^2)\) & \(U(\delta\V\alpha) = 1- \frac{i}{\hbar}\delta\V \alpha \cdot \V J + \bigO(\delta \alpha^2)\) \\ \hline
    Formula & \(U(\V a) = \exp(-\frac{i}{\hbar}\V a\cdot \V P)\) & \(U(\V \alpha) = \exp(-\frac{i}{\hbar}\V \alpha\cdot \V J)\) \\ \hline
    Commutator relation & \(\begin{array}{cc} & \frac{i}{\hbar}[\delta\V a\cdot\V P, \V X] = \delta\V a \\ \Rightarrow & [X_i,P_j] = i\hbar\delta_{i,j} \end{array}\) & \(\begin{array}{cc} & \frac{i}{\hbar}[\delta\V\alpha\cdot \V J, \V V] = \delta\V\alpha\times \V V \\ \Rightarrow & [J_i,V_j] = i\hbar \epsilon_{ijk}V_k \end{array}\) \\ \hline
    Transformation group & ??? & \(SO(3)\) \\ \hline
    & abelian & non-abelian \\ \hline
    & \(U(\V a)U(\V b) = U(\V b)U(\V a)\) & \([\V R(\delta\V \alpha), \V R(\delta\V \beta)]\V x = \V R(\delta\V \alpha \times \delta\V \beta)\V x\) \\
    & & applying homomorphism, \([U(\delta\V\alpha), U(\delta\V\beta)] = U(\delta\V\alpha \times \delta\V\beta)\) \\
    Property derived from abelianess & \([P_i,P_j] = 0\) & \([J_i,J_j] = i\hbar \epsilon_{ijk}V_k\) (can also be derived by viewing \(\V J\) as a vector) \\ \hline
  \end{tabular}
  \caption{Translation and rotation}
  \label{tab:continuous}
\end{table}


\end{landscape}
\pagestyle{plain}

\end{document}