\documentclass[a4paper]{article}

\def\ntitle{Optimisation}

\ifx \nauthor\undefined
  \def\nauthor{Qiangru Kuang}
\else
\fi

\ifx \ntitle\undefined
  \def\ntitle{Template}
\else
\fi

\ifx \nauthoremail\undefined
  \def\nauthoremail{qk206@cam.ac.uk}
\else
\fi

\ifx \ndate\undefined
  \def\ndate{\today}
\else
\fi

\title{\ntitle}
\author{\nauthor}
\date{\ndate}

%\usepackage{microtype}
\usepackage{mathtools}
\usepackage{amsthm}
\usepackage{stmaryrd}%symbols used so far: \mapsfrom
\usepackage{empheq}
\usepackage{amssymb}
\let\mathbbalt\mathbb
\let\pitchforkold\pitchfork
\usepackage{unicode-math}
\let\mathbb\mathbbalt%reset to original \mathbb
\let\pitchfork\pitchforkold

\usepackage{imakeidx}
\makeindex[intoc]

%to address the problem that Latin modern doesn't have unicode support for setminus
%https://tex.stackexchange.com/a/55205/26707
\AtBeginDocument{\renewcommand*{\setminus}{\mathbin{\backslash}}}
\AtBeginDocument{\renewcommand*{\models}{\vDash}}%for \vDash is same size as \vdash but orginal \models is larger
\AtBeginDocument{\let\Re\relax}
\AtBeginDocument{\let\Im\relax}
\AtBeginDocument{\DeclareMathOperator{\Re}{Re}}
\AtBeginDocument{\DeclareMathOperator{\Im}{Im}}
\AtBeginDocument{\let\div\relax}
\AtBeginDocument{\DeclareMathOperator{\div}{div}}

\usepackage{tikz}
\usetikzlibrary{automata,positioning}
\usepackage{pgfplots}
%some preset styles
\pgfplotsset{compat=1.15}
\pgfplotsset{centre/.append style={axis x line=middle, axis y line=middle, xlabel={$x$}, ylabel={$y$}, axis equal}}
\usepackage{tikz-cd}
\usepackage{graphicx}
\usepackage{newunicodechar}

\usepackage{fancyhdr}

\fancypagestyle{mypagestyle}{
    \fancyhf{}
    \lhead{\emph{\nouppercase{\leftmark}}}
    \rhead{}
    \cfoot{\thepage}
}
\pagestyle{mypagestyle}

\usepackage{titlesec}
\newcommand{\sectionbreak}{\clearpage} % clear page after each section
\usepackage[perpage]{footmisc}
\usepackage{blindtext}

%\reallywidehat
%https://tex.stackexchange.com/a/101136/26707
\usepackage{scalerel,stackengine}
\stackMath
\newcommand\reallywidehat[1]{%
\savestack{\tmpbox}{\stretchto{%
  \scaleto{%
    \scalerel*[\widthof{\ensuremath{#1}}]{\kern-.6pt\bigwedge\kern-.6pt}%
    {\rule[-\textheight/2]{1ex}{\textheight}}%WIDTH-LIMITED BIG WEDGE
  }{\textheight}% 
}{0.5ex}}%
\stackon[1pt]{#1}{\tmpbox}%
}

%\usepackage{braket}
\usepackage{thmtools}%restate theorem
\usepackage{hyperref}

% https://en.wikibooks.org/wiki/LaTeX/Hyperlinks
\hypersetup{
    %bookmarks=true,
    unicode=true,
    pdftitle={\ntitle},
    pdfauthor={\nauthor},
    pdfsubject={Mathematics},
    pdfcreator={\nauthor},
    pdfproducer={\nauthor},
    pdfkeywords={math maths \ntitle},
    colorlinks=true,
    linkcolor={red!50!black},
    citecolor={blue!50!black},
    urlcolor={blue!80!black}
}

\usepackage{cleveref}



% TODO: mdframed often gives bad breaks that cause empty lines. Would like to switch to tcolorbox.
% The current workaround is to set innerbottommargin=0pt.

%\usepackage[theorems]{tcolorbox}





\usepackage[framemethod=tikz]{mdframed}
\mdfdefinestyle{leftbar}{
  %nobreak=true, %dirty hack
  linewidth=1.5pt,
  linecolor=gray,
  hidealllines=true,
  leftline=true,
  leftmargin=0pt,
  innerleftmargin=5pt,
  innerrightmargin=10pt,
  innertopmargin=-5pt,
  % innerbottommargin=5pt, % original
  innerbottommargin=0pt, % temporary hack 
}
%\newmdtheoremenv[style=leftbar]{theorem}{Theorem}[section]
%\newmdtheoremenv[style=leftbar]{proposition}[theorem]{proposition}
%\newmdtheoremenv[style=leftbar]{lemma}[theorem]{Lemma}
%\newmdtheoremenv[style=leftbar]{corollary}[theorem]{corollary}

\newtheorem{theorem}{Theorem}[section]
\newtheorem{proposition}[theorem]{Proposition}
\newtheorem{lemma}[theorem]{Lemma}
\newtheorem{corollary}[theorem]{Corollary}
\newtheorem{axiom}[theorem]{Axiom}
\newtheorem*{axiom*}{Axiom}

\surroundwithmdframed[style=leftbar]{theorem}
\surroundwithmdframed[style=leftbar]{proposition}
\surroundwithmdframed[style=leftbar]{lemma}
\surroundwithmdframed[style=leftbar]{corollary}
\surroundwithmdframed[style=leftbar]{axiom}
\surroundwithmdframed[style=leftbar]{axiom*}

\theoremstyle{definition}

\newtheorem*{definition}{Definition}
\surroundwithmdframed[style=leftbar]{definition}

\newtheorem*{slogan}{Slogan}
\newtheorem*{eg}{Example}
\newtheorem*{ex}{Exercise}
\newtheorem*{remark}{Remark}
\newtheorem*{notation}{Notation}
\newtheorem*{convention}{Convention}
\newtheorem*{assumption}{Assumption}
\newtheorem*{question}{Question}
\newtheorem*{answer}{Answer}
\newtheorem*{note}{Note}
\newtheorem*{application}{Application}

%operator macros

%basic
\DeclareMathOperator{\lcm}{lcm}

%matrix
\DeclareMathOperator{\tr}{tr}
\DeclareMathOperator{\Tr}{Tr}
\DeclareMathOperator{\adj}{adj}

%algebra
\DeclareMathOperator{\Hom}{Hom}
\DeclareMathOperator{\End}{End}
\DeclareMathOperator{\id}{id}
\DeclareMathOperator{\im}{im}
\DeclarePairedDelimiter{\generation}{\langle}{\rangle}

%groups
\DeclareMathOperator{\sym}{Sym}
\DeclareMathOperator{\sgn}{sgn}
\DeclareMathOperator{\inn}{Inn}
\DeclareMathOperator{\aut}{Aut}
\DeclareMathOperator{\GL}{GL}
\DeclareMathOperator{\SL}{SL}
\DeclareMathOperator{\PGL}{PGL}
\DeclareMathOperator{\PSL}{PSL}
\DeclareMathOperator{\SU}{SU}
\DeclareMathOperator{\UU}{U}
\DeclareMathOperator{\SO}{SO}
\DeclareMathOperator{\OO}{O}
\DeclareMathOperator{\PSU}{PSU}

%hyperbolic
\DeclareMathOperator{\sech}{sech}

%field, galois heory
\DeclareMathOperator{\ch}{ch}
\DeclareMathOperator{\gal}{Gal}
\DeclareMathOperator{\emb}{Emb}



%ceiling and floor
%https://tex.stackexchange.com/a/118217/26707
\DeclarePairedDelimiter\ceil{\lceil}{\rceil}
\DeclarePairedDelimiter\floor{\lfloor}{\rfloor}


\DeclarePairedDelimiter{\innerproduct}{\langle}{\rangle}

%\DeclarePairedDelimiterX{\norm}[1]{\lVert}{\rVert}{#1}
\DeclarePairedDelimiter{\norm}{\lVert}{\rVert}



%Dirac notation
%TODO: rewrite for variable number of arguments
\DeclarePairedDelimiterX{\braket}[2]{\langle}{\rangle}{#1 \delimsize\vert #2}
\DeclarePairedDelimiterX{\braketthree}[3]{\langle}{\rangle}{#1 \delimsize\vert #2 \delimsize\vert #3}

\DeclarePairedDelimiter{\bra}{\langle}{\rvert}
\DeclarePairedDelimiter{\ket}{\lvert}{\rangle}




%macros

%general

%divide, not divide
\newcommand*{\divides}{\mid}
\newcommand*{\ndivides}{\nmid}
%vector, i.e. mathbf
%https://tex.stackexchange.com/a/45746/26707
\newcommand*{\V}[1]{{\ensuremath{\symbf{#1}}}}
%closure
\newcommand*{\cl}[1]{\overline{#1}}
%conjugate
\newcommand*{\conj}[1]{\overline{#1}}
%set complement
\newcommand*{\stcomp}[1]{\overline{#1}}
\newcommand*{\compose}{\circ}
\newcommand*{\nto}{\nrightarrow}
\newcommand*{\p}{\partial}
%embed
\newcommand*{\embed}{\hookrightarrow}
%surjection
\newcommand*{\surj}{\twoheadrightarrow}
%power set
\newcommand*{\powerset}{\mathcal{P}}

%matrix
\newcommand*{\matrixring}{\mathcal{M}}

%groups
\newcommand*{\normal}{\trianglelefteq}
%rings
\newcommand*{\ideal}{\trianglelefteq}

%fields
\renewcommand*{\C}{{\mathbb{C}}}
\newcommand*{\R}{{\mathbb{R}}}
\newcommand*{\Q}{{\mathbb{Q}}}
\newcommand*{\Z}{{\mathbb{Z}}}
\newcommand*{\N}{{\mathbb{N}}}
\newcommand*{\F}{{\mathbb{F}}}
%not really but I think this belongs here
\newcommand*{\A}{{\mathbb{A}}}

%asymptotic
\newcommand*{\bigO}{O}
\newcommand*{\smallo}{o}

%probability
\newcommand*{\prob}{\mathbb{P}}
\newcommand*{\E}{\mathbb{E}}

%vector calculus
\newcommand*{\gradient}{\V \nabla}
\newcommand*{\divergence}{\gradient \cdot}
\newcommand*{\curl}{\gradient \cdot}

%logic
\newcommand*{\yields}{\vdash}
\newcommand*{\nyields}{\nvdash}

%differential geometry
\renewcommand*{\H}{\mathbb{H}}
\newcommand*{\transversal}{\pitchfork}
\renewcommand{\d}{\mathrm{d}} % exterior derivative

%number theory
\newcommand*{\legendre}[2]{\genfrac{(}{)}{}{}{#1}{#2}}%Legendre symbol


\begin{document}

\maketitle
\tableofcontents

\section{Introduction}

The typical problem is of the form

\begin{center}
  minimise $f(x)$ subject to $g(x) = b,\: x \in X$,
\end{center}

where

\begin{itemize}
\item $f : \mathbb{R}^n → \mathbb{R}$ is the \emph{objective function},
\item $X \subseteq \mathbb{R}^n$ defines a \emph{regional constraint},
\item $g: \mathbb{R}^n → \mathbb{R}^m$ defines $m$ \emph{functional constraints},
\item $b \in \mathbb{R}^m$ is the \emph{right-hand side}.
\end{itemize}

We will also use the terminology

\begin{itemize}
\item a \emph{feasible solution} is any $x \in X$ s.t. $g(x) = b$,
\item an \emph{optimal solution} is a feasible solution $x^*$ s.t. $f(x^*) ≤ f(x)$ for all feasible $x$.

\end{itemize}

A problem with inequality constraints can be put into equality form by introducing \emph{slack variables}:

\begin{center}
  minimise $f(x)$ subject to $g(x) ≤ b,\: x \in X$

  minmise $f(x)$ subject to $g(x) + z = b,\: x \in X$.
\end{center}

\section{Lagrangian Methods}

Consider the problem

\begin{center}
  minimise $f(x)$ subject to $g(x) = b,\: x \in X$.
\end{center}

Introduce a new function $L : \mathbb{R}^n × \mathbb{R}^m → \mathbb{R},\: L(x, \lambda) = f(x) + \lambda^T (b - g(x))$, the \emph{Lagrangian} of the problem. The components $\lambda_i$ is called the \emph{Lagrangian multiplier} for the $i$-th functional constraint.

\subsection{Lagrangian Sufficiency}

\begin{thm}[Lagrangian Sufficiency Theorem]
  Let $x^*$ be feasible. Suppose there exists $\lambda^* \in \mathbb{R}^m$ s.t.
  \[
    L(x^*, \lambda^*) ≤ L(x, \lambda^*) \: \text{for all } x \in X
  \]

  then $x^*$ is optimal.

\end{thm}

\begin{proof}
  For any feasible $x$ and any $\lambda$ we have
  \[
    L(x, \lambda) = f(x) + \lambda^T (b - g(x)) = f(x)
  \]

  so
  \begin{align*}
    f(x^*) &= L(x^*, \lambda^*) \\
           &≤ L(x, \lambda^*) \: \text{for all} x \in X \: \text{by assumption } \\
           &= f(x) \: \text{for all feasible } x \in X
  \end{align*}
\end{proof}

\subsection{Steps}

Consider the problem to

\begin{center}
  minimise $f(x)$ subject to $g(x) = b,\: x \in X$
\end{center}

\begin{itemize}
\item[Step 1] $\Lambda := \{\displaystyle \lambda \in \mathbb{R}^n: \inf_{x \in X} L(x, \lambda) > -∞\}$
\item[Step 2] For each $\lambda \in \Lambda$ find the optimal solution to the unconstrained problem

\begin{center}
  minimise $L(x, \lambda)$ subject to $x \in X$
\end{center}

Let $x(\lambda)$ be the minimiser
\item[Step 3] Find a $\lambda^* \in \Lambda$ s.t. $x^* = x(\lambda^*)$ is feasible for the original problem, i.e. $g(x^*) = b$.
\end{itemize}

Note that $x^*$ is optimal by Lagrangian sufficiency theorem.

\subsection{Complementary Slackness}

Given an inequality constraint, we introduce the equivalent equality constraint problem with slack variables. The Lagrangian is now
\[
  L(x, z, \lambda) = f(x) + \lambda^T (b - g(x)) - \lambda^T z
\]

Note that for the infimum to exist, $\lambda ≤ 0$. The inequality constraint $g(x) ≤ b $ for the variable $x$ introduces a \emph{sign constraint} $\lambda ≤ 0$ for the Lagrange multiplier $\lambda$.

In addition, for $\lambda ≤ 0$ we have $\displaystyle \inf_{z ≥ 0} (-\lambda^T z) = 0$. Thus for each $\lambda \in \Lambda$, the optimal $z = z(\lambda)$ satisfis the \emph{compementary slacknes} condition $\lambda^T z = 0$. Hence $\lambda_i z_i = 0$ for all $i$.

\subsection{Lagrangian Necessity}

Note that for any Lagrange multiplier $\lambda$ we have

\begin{align*}
  \inf_{x \in X,\: g(x) = b} f(x) &= \inf_{x \in X,\: g(x) = b} (f(x) + \lambda^T (b - g(x))) \\
                                &= \inf_{x \in X,\: g(x) = b} L(x, \lambda) \\
                                &≥ \inf_{x \in X} L(x, \lambda)
\end{align*}

since $\{x \in X: g(x) = b\} ⊆ X$. We say the Lagrangian method works if there exists a Lagrange multiplier $\lambda^*$ s.t. there is an equality. To characterise when the Lagrangian method works, we need to define some terms.

\begin{defi}
  $\psi: \mathbb{R}^m → \mathbb{R}$ has a \emph{supporting hyperplane} at a point $b \in \mathbb{R}^m$ if there exist a $\lambda$ s.t.
  \[
    \psi(c) ≥ \psi(b) + \lambda^T (c-b)
  \]
  for all $c \in \mathbb{R}^m$.
\end{defi}

\begin{defi}
  The \emph{value function} $φ$ on $\mathbb{R}^n$ is defined by $\displaystyle\phi(c) = \inf_{x \in X,\: g(x) = c} f(x)$.
\end{defi}

\begin{thm}[Lagrangian Necessity Theorem]
  The Lagrangian method works for the problem iff the value function has a supporting hyperplane at $b$.
\end{thm}

\begin{proof}
  The Lagrangian method works iff there exists a $\lambda$ s.t.
  \[
   \phi(b) = \inf_{x \in X} (f(x) + \lambda^T (b - g(x)))
  \]

  The value function has a supporting hyperplane at $b$ iff there exists a $\lambda$ s.t.
  \[
   \phi(b) = \inf_{c \in \mathbb{R}^m} (\phi(c) + \lambda^T (b - c))
  \]

  Thus the equivalence of the two hypotheses is proven by noting the equality
  \begin{align*}
    \inf_{x \in X} (f(x) + \lambda^T (b - g(x))) &= \inf_{c \in \mathbb{R}^m} \underbrace{\inf_{x \in X,\: g(x) = c} (f(x)}_{\phi(c)} + \lambda^T \underbrace{(c - g(x))}_{= 0} + \lambda^T (b - c)) \\
                                         &= \inf_{c \in \mathbb{R}^m} (\phi(c) + \lambda^T (b - c))
  \end{align*}
\end{proof}

To check whether a function has a supporting hyperplane, we have to define a few terms.

\begin{defi}
  A subset $C \subseteq \mathbb{R}^n$ is \emph{convex} if
  \[
    x, y \in C \: \text{implies } \theta x + (1 - \theta)y \in C \: \text{for all } 0 \subseteq \theta \subseteq 1.
  \]
\end{defi}

\begin{defi}
  A function $\psi: \mathbb{R}^m \rightarrow \mathbb{R}^m$ is \emph{convex} if
  \[
    \psi(\theta x + (1 - \theta) y \leq \theta\psi(x) + (1-\theta) \psi(y) \: \text{for all } x, y \in \mathbb{R}^m \: \text{and } 0 \leq \theta \leq 1.
  \]
\end{defi}

\begin{cor}
  A fucntion $\psi : \mathbb{R}^m \rightarrow \mathbb{R}$ is convex iff the set
  \[
    C := \{(x, y) : \psi(x) \leq y \} \subseteq \mathbb{R}^{m + 1}
  \]
  is convex\footnote{The set $C$ defined above is call the \emph{epigraph} of $\psi$.}.
\end{cor}

\begin{thm}[Non-examinable]
  A function is convex iff it has a supporting hyperplane at each point.
\end{thm}

\begin{prop}
  If
  \begin{enumerate}
  \item the set $X$ is convex,
  \item the objective function $f$ is convex, and
  \item the functional constraint is
    \begin{itemize}
    \item either $g(x) = b$ and $g$ is linear, or
    \item $g(x) \leq b$ and $g$ is convex.
    \end{itemize}
  \end{enumerate}

  then $\psi$ is convex.
\end{prop}

\section{Dual Problem}

Consider the \emph{primal} problem

\begin{center}
  $P:$ minimise $f(x)$ subject to $g(x) = b,\: x \in X$.
\end{center}

As before, introduce the Lagrangian $L$ and the set of Lagrange multipliers $\Lambda$. Now define the \emph{dual objective function} $h: \Lambda \rightarrow \mathbb{R}$ by
\[
  h(\lambda) = \inf_{x \in X} L(x, \lambda).
\]

The \emph{dual} problem is defined to be

\begin{center}
  $D:$ maximise $h(\lambda)$ subject to $\lambda \in \Lambda$.
\end{center}

The set $\Lambda$ is the set of \emph{feasible solutions to the dual problem}.

\begin{thm}[Weak Duality]
  Let $x$ be feasible for $P$ and let $\lambda$ be feasible for $D$. Then
  \[
    h(\lambda) \leq f(x)
  \]

  and in particular
  \[
    \sup_{\lambda \in \Lambda} \leq \inf_{x \in X,\: g(x) = b} f(x).
  \]
\end{thm}

\begin{proof}
  Let $x$ and $f$ be feasible for their respective problems, then
  \begin{align*}
    h(\lambda) &= \inf\{L(x', \lambda): x' \in X \} \\
               &\leq L(x, \lambda) \: \text{for all } x \in X \\
               &= f(x) \: \text{for all feasible } x
  \end{align*}
\end{proof}

The difference
\[
  \inf_{x \in X,\: g(x) = b} f(x) - \sup_{\lambda \in \Lambda} h(\lambda)
\]

is called the \emph{duality gap}. Weak duality says that the duality gap is non-negative while in the case where the conditions of Lagrangian necessity are met, the duality gap is zero. This is called strong duality.

\begin{eg}[Linear Programming]
  Consider the primal problem
  \begin{center}
    $P:$ maximise $c^T x$ subject to $Ax \leq b,\: x \geq 0$
  \end{center}
  where $A$ is a $m \times n$ matrix, $b \in \mathbb{R}^m$, and $c \in \mathbb{R}^n$.

  The dual problem is found as follows:
  \begin{enumerate}
  \item Introduce slack variables
    \begin{center}
      $P:$ maximise $c^T x$ subject to $Ax + z = b, \: x \geq 0, \: z \geq 0$
    \end{center}
  \item The Lagrangian is
    \[
      L(x, z, \lambda) = b^T \lambda + (c - A^T \lambda)^T x - \lambda ^T z.
    \]
  \item The set of feasible solutions to the dual problem is
    \begin{align*}
      \Lambda &= \{\lambda \in \mathbb{R}^m: \sup_{x \geq 0,\: z \geq 0} L(x, z, \lambda) < \infty \} \\
              &= \{\lambda \in \mathbb{R}^m: A^T \lambda \geq c,\: \lambda \geq 0\}.
    \end{align*}
  \item The dual objective funtion
    \[
      \sup_{x \geq 0, z \geq 0} L(x, z, \lambda) = b^T \lambda \: \text{for } \lambda \in \Lambda.
    \]

    The dual problem is then

    \begin{center}
      $D:$ minimise $b^T \lambda$ subject to $A^T \lambda \geq c,\: \lambda \geq 0$.
    \end{center}
  \end{enumerate}
\end{eg}

We can verify that the dual of the dual is the original problem.

\begin{thm}[Fundamental Theorem of Linear Programming]
  Consider the problem
  
  \begin{center}
    $P:$ maximise $c^T x$ subject to $Ax \leq b,\: x \geq 0$.
  \end{center}
  
  A vector $x^* \in \mathbb{R}^n$ is optimal for $P$ iff there exists a vector $\lambda^* \in \mathbb{R}^m$ s.t.

  \begin{itemize}
  \item $A x^* \leq b,\: x^* \geq 0$ \: (primal feasibility)
  \item $A^T \lambda^* \geq c,\: \lambda^* \geq 0$ \: (dual feasibility)
  \item $(\lambda^*)^T (b - A x^*) = 0 = (x^*)^T (c - A^T \lambda ^*)$ \: (complementary slackness)
  \end{itemize}
\end{thm}

In this case, the value of the primal problem $c^T x^* = b^T \lambda^*$ equals the value of the dual problem.

\subsection{Extreme Points and Basic Feasible Solutions}

Suppose $\psi$ is a convex function, then give $x, y \in X$ and $0 \leq \theta \leq 1$,

\begin{align*}
  \psi(\theta x + (1 - \theta) y) & \leq \theta \psi(x) + (1 - \theta)\psi(y) \\
                                  &\leq \max\{\psi(x), \psi(y)\}
\end{align*}

That is to say the maximimum of $\psi$ on any segment occurs at one of the end points. Thus to find the maximum of $\psi$ over $X$ we over have to consider points of $X$ that do not lie on a line segment contained in $X$.

\begin{defi}
  Let $C \subseteq \mathbb{R}^n$ be a convex set. A point $x \in C$ is an \emph{extreme point} if

  \[
    x = \theta y + (1 - \theta) z
  \]
  for $y, z \in X$ and $0 \leq \theta \leq 1$ imples $x = y = z$.
\end{defi}

\begin{defi}
  The \emph{standard form} of a linear programme is

  \begin{center}
    maximise $c^T x$ subject to $Ax = b,\: x\geq 0$
  \end{center}
  where $c \in \mathbb{R}^n,\: b \in \mathbb{R}^m$ and $A$ is a $m \times n$ matrix. The set $C := \{x \in \mathbb{R}^n: Ax = b\}$ is the set of feasible solutions to the problem.
\end{defi}

\begin{prop}
  The set $C$ is convex.
\end{prop}

\begin{proof}
  Suppose $x, y \in C$. Then $Ax = b,\: x \geq 0$, and $Ay = b,\: y \geq 0$. Fix $\theta \in [0, 1]$ and let $z = \theta x + (1 - \theta)y$. Then

  \begin{align*}
    Az &= \theta Ax + (1 - \theta) Ay = b \\
    z_i &= \theta x_i + (1 - \theta) y_i \geq 0 \: \text{for all } i
  \end{align*}

  Hence $z \in C$.
\end{proof}

A solution $x \in \mathbb{R}^n$ of the equation $Ax = b$ is called \emph{basic} if at least $n - m$ entries of $x$ are zero. If $x$ is a basic solution and $x \geq 0$ then $x$ is called a \emph{basic feasible solution}, abbreviated \emph{b.f.s}.

\begin{thm}
  Let $x$ be a point in $C$ with the property that have at least $m+1$ indices $i$ s.t. $x_i>0$. Then $x$ is not a extreme point of $C$.
\end{thm}

\begin{proof}
  \texttt{to be filled in}
\end{proof}

\begin{thm}
  Suppose that every set of $m$ columns of $A$ is linearly independent. Let $x$ be a point in $C$ with the property that at most $m$ indices $i$ are s.t. $z_i>0$. Then $x$ is an extreme point of $C$.
\end{thm}

\begin{proof}
  \texttt{to be filled in}
\end{proof}

\subsection{Simplex Algorithm}

We assume $A$ is a $m\times n$ matrix and $n > m$ and that every set of $m$ columns of $A$ is linearly independent. From discussion above it suffices to consider the extreme points of $C$. Fix $B \subset \{1, \ldots ,n\}$ with $|B|=m$, and let $N=\{1,\ldots, n\} \setminus B$. If $B=\{i_1,\ldots,i_m\}$, let $A_B=(A_{i_1}\:\ldots\:A_{i_m})$ bethe $m\times m$ matrix formed by takning the colums of $A$ indexed by $i \in B$. By assumption $A_B$ is invertible. Define $x_B$ and $c_B$ similarly. Similar for the set $N$.

Using this notation, the equation is
\[
  A_B x_B + A_N x_N = b
\]

Setting $x_N=0$ yields $x_B=A_B^{-1}b$ so by rearranging the coordinates we may write the basic point as $x=\binom{x_B}{x_N}\binom{A_B^{-1}b}{0}$. To check this $x$ is feasible, we need $A_B^{-1}b \geq 0$. If it is, compute the objective function $c^Tx=c_B^Tx_B$.

To check the optimality, we use the fundamental theorem. For the b.f.s. $x=\binom{A_B^{-1}b}{0}$, we associate to it a Lagrange multiplier $\lambda$ by complementary slackness

\begin{align*}
  0 &= (c - A^T\lambda)^Tx \\
    &= (c_B - A_B^T\lambda)^Tx_B
\end{align*}

Assuming \emph{non-degeneracy}, we take $\lambda=(A_B^T)^{-1}c_B$. By construction $x$ satisfies primal feasibility and $\lambda$ satisfies complementary slackness. Thus $x$ is optimal iff $\lambda$ satisfies dual feasibility. Thus if $A^T\lambda\geq c$ then we know we have found an optimal solution.

\section{Two-person Zero-sum Game}

Suppose Player I and Player II are competing. I has $m$ choices of strategies, labelled $i \in \{1,\ldots,m\}$ while II has $n$, labelled $j \in \{1,\ldots,n\}$. Zero-sum means that if I chooses strategy $i$ and II chooses strategy $j$ then

\begin{itemize}
\item I is paid $a_{i,j}$
\item II is paid $-a_{i,j}$
\end{itemize}

So the net payment is zero. The matrix $A=(a_{i,j})_{i,j}$ is called the \emph{payoff} matrix of the game
\end{document}
