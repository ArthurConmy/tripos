\documentclass[a4paper]{article}

\def\ntitle{IB Quantum Mechanics in $5$ Minutes}

\ifx \nauthor\undefined
  \def\nauthor{Qiangru Kuang}
\else
\fi

\ifx \ntitle\undefined
  \def\ntitle{Template}
\else
\fi

\ifx \nauthoremail\undefined
  \def\nauthoremail{qk206@cam.ac.uk}
\else
\fi

\ifx \ndate\undefined
  \def\ndate{\today}
\else
\fi

\title{\ntitle}
\author{\nauthor}
\date{\ndate}

%\usepackage{microtype}
\usepackage{mathtools}
\usepackage{amsthm}
\usepackage{stmaryrd}%symbols used so far: \mapsfrom
\usepackage{empheq}
\usepackage{amssymb}
\let\mathbbalt\mathbb
\let\pitchforkold\pitchfork
\usepackage{unicode-math}
\let\mathbb\mathbbalt%reset to original \mathbb
\let\pitchfork\pitchforkold

\usepackage{imakeidx}
\makeindex[intoc]

%to address the problem that Latin modern doesn't have unicode support for setminus
%https://tex.stackexchange.com/a/55205/26707
\AtBeginDocument{\renewcommand*{\setminus}{\mathbin{\backslash}}}
\AtBeginDocument{\renewcommand*{\models}{\vDash}}%for \vDash is same size as \vdash but orginal \models is larger
\AtBeginDocument{\let\Re\relax}
\AtBeginDocument{\let\Im\relax}
\AtBeginDocument{\DeclareMathOperator{\Re}{Re}}
\AtBeginDocument{\DeclareMathOperator{\Im}{Im}}
\AtBeginDocument{\let\div\relax}
\AtBeginDocument{\DeclareMathOperator{\div}{div}}

\usepackage{tikz}
\usetikzlibrary{automata,positioning}
\usepackage{pgfplots}
%some preset styles
\pgfplotsset{compat=1.15}
\pgfplotsset{centre/.append style={axis x line=middle, axis y line=middle, xlabel={$x$}, ylabel={$y$}, axis equal}}
\usepackage{tikz-cd}
\usepackage{graphicx}
\usepackage{newunicodechar}

\usepackage{fancyhdr}

\fancypagestyle{mypagestyle}{
    \fancyhf{}
    \lhead{\emph{\nouppercase{\leftmark}}}
    \rhead{}
    \cfoot{\thepage}
}
\pagestyle{mypagestyle}

\usepackage{titlesec}
\newcommand{\sectionbreak}{\clearpage} % clear page after each section
\usepackage[perpage]{footmisc}
\usepackage{blindtext}

%\reallywidehat
%https://tex.stackexchange.com/a/101136/26707
\usepackage{scalerel,stackengine}
\stackMath
\newcommand\reallywidehat[1]{%
\savestack{\tmpbox}{\stretchto{%
  \scaleto{%
    \scalerel*[\widthof{\ensuremath{#1}}]{\kern-.6pt\bigwedge\kern-.6pt}%
    {\rule[-\textheight/2]{1ex}{\textheight}}%WIDTH-LIMITED BIG WEDGE
  }{\textheight}% 
}{0.5ex}}%
\stackon[1pt]{#1}{\tmpbox}%
}

%\usepackage{braket}
\usepackage{thmtools}%restate theorem
\usepackage{hyperref}

% https://en.wikibooks.org/wiki/LaTeX/Hyperlinks
\hypersetup{
    %bookmarks=true,
    unicode=true,
    pdftitle={\ntitle},
    pdfauthor={\nauthor},
    pdfsubject={Mathematics},
    pdfcreator={\nauthor},
    pdfproducer={\nauthor},
    pdfkeywords={math maths \ntitle},
    colorlinks=true,
    linkcolor={red!50!black},
    citecolor={blue!50!black},
    urlcolor={blue!80!black}
}

\usepackage{cleveref}



% TODO: mdframed often gives bad breaks that cause empty lines. Would like to switch to tcolorbox.
% The current workaround is to set innerbottommargin=0pt.

%\usepackage[theorems]{tcolorbox}





\usepackage[framemethod=tikz]{mdframed}
\mdfdefinestyle{leftbar}{
  %nobreak=true, %dirty hack
  linewidth=1.5pt,
  linecolor=gray,
  hidealllines=true,
  leftline=true,
  leftmargin=0pt,
  innerleftmargin=5pt,
  innerrightmargin=10pt,
  innertopmargin=-5pt,
  % innerbottommargin=5pt, % original
  innerbottommargin=0pt, % temporary hack 
}
%\newmdtheoremenv[style=leftbar]{theorem}{Theorem}[section]
%\newmdtheoremenv[style=leftbar]{proposition}[theorem]{proposition}
%\newmdtheoremenv[style=leftbar]{lemma}[theorem]{Lemma}
%\newmdtheoremenv[style=leftbar]{corollary}[theorem]{corollary}

\newtheorem{theorem}{Theorem}[section]
\newtheorem{proposition}[theorem]{Proposition}
\newtheorem{lemma}[theorem]{Lemma}
\newtheorem{corollary}[theorem]{Corollary}
\newtheorem{axiom}[theorem]{Axiom}
\newtheorem*{axiom*}{Axiom}

\surroundwithmdframed[style=leftbar]{theorem}
\surroundwithmdframed[style=leftbar]{proposition}
\surroundwithmdframed[style=leftbar]{lemma}
\surroundwithmdframed[style=leftbar]{corollary}
\surroundwithmdframed[style=leftbar]{axiom}
\surroundwithmdframed[style=leftbar]{axiom*}

\theoremstyle{definition}

\newtheorem*{definition}{Definition}
\surroundwithmdframed[style=leftbar]{definition}

\newtheorem*{slogan}{Slogan}
\newtheorem*{eg}{Example}
\newtheorem*{ex}{Exercise}
\newtheorem*{remark}{Remark}
\newtheorem*{notation}{Notation}
\newtheorem*{convention}{Convention}
\newtheorem*{assumption}{Assumption}
\newtheorem*{question}{Question}
\newtheorem*{answer}{Answer}
\newtheorem*{note}{Note}
\newtheorem*{application}{Application}

%operator macros

%basic
\DeclareMathOperator{\lcm}{lcm}

%matrix
\DeclareMathOperator{\tr}{tr}
\DeclareMathOperator{\Tr}{Tr}
\DeclareMathOperator{\adj}{adj}

%algebra
\DeclareMathOperator{\Hom}{Hom}
\DeclareMathOperator{\End}{End}
\DeclareMathOperator{\id}{id}
\DeclareMathOperator{\im}{im}
\DeclarePairedDelimiter{\generation}{\langle}{\rangle}

%groups
\DeclareMathOperator{\sym}{Sym}
\DeclareMathOperator{\sgn}{sgn}
\DeclareMathOperator{\inn}{Inn}
\DeclareMathOperator{\aut}{Aut}
\DeclareMathOperator{\GL}{GL}
\DeclareMathOperator{\SL}{SL}
\DeclareMathOperator{\PGL}{PGL}
\DeclareMathOperator{\PSL}{PSL}
\DeclareMathOperator{\SU}{SU}
\DeclareMathOperator{\UU}{U}
\DeclareMathOperator{\SO}{SO}
\DeclareMathOperator{\OO}{O}
\DeclareMathOperator{\PSU}{PSU}

%hyperbolic
\DeclareMathOperator{\sech}{sech}

%field, galois heory
\DeclareMathOperator{\ch}{ch}
\DeclareMathOperator{\gal}{Gal}
\DeclareMathOperator{\emb}{Emb}



%ceiling and floor
%https://tex.stackexchange.com/a/118217/26707
\DeclarePairedDelimiter\ceil{\lceil}{\rceil}
\DeclarePairedDelimiter\floor{\lfloor}{\rfloor}


\DeclarePairedDelimiter{\innerproduct}{\langle}{\rangle}

%\DeclarePairedDelimiterX{\norm}[1]{\lVert}{\rVert}{#1}
\DeclarePairedDelimiter{\norm}{\lVert}{\rVert}



%Dirac notation
%TODO: rewrite for variable number of arguments
\DeclarePairedDelimiterX{\braket}[2]{\langle}{\rangle}{#1 \delimsize\vert #2}
\DeclarePairedDelimiterX{\braketthree}[3]{\langle}{\rangle}{#1 \delimsize\vert #2 \delimsize\vert #3}

\DeclarePairedDelimiter{\bra}{\langle}{\rvert}
\DeclarePairedDelimiter{\ket}{\lvert}{\rangle}




%macros

%general

%divide, not divide
\newcommand*{\divides}{\mid}
\newcommand*{\ndivides}{\nmid}
%vector, i.e. mathbf
%https://tex.stackexchange.com/a/45746/26707
\newcommand*{\V}[1]{{\ensuremath{\symbf{#1}}}}
%closure
\newcommand*{\cl}[1]{\overline{#1}}
%conjugate
\newcommand*{\conj}[1]{\overline{#1}}
%set complement
\newcommand*{\stcomp}[1]{\overline{#1}}
\newcommand*{\compose}{\circ}
\newcommand*{\nto}{\nrightarrow}
\newcommand*{\p}{\partial}
%embed
\newcommand*{\embed}{\hookrightarrow}
%surjection
\newcommand*{\surj}{\twoheadrightarrow}
%power set
\newcommand*{\powerset}{\mathcal{P}}

%matrix
\newcommand*{\matrixring}{\mathcal{M}}

%groups
\newcommand*{\normal}{\trianglelefteq}
%rings
\newcommand*{\ideal}{\trianglelefteq}

%fields
\renewcommand*{\C}{{\mathbb{C}}}
\newcommand*{\R}{{\mathbb{R}}}
\newcommand*{\Q}{{\mathbb{Q}}}
\newcommand*{\Z}{{\mathbb{Z}}}
\newcommand*{\N}{{\mathbb{N}}}
\newcommand*{\F}{{\mathbb{F}}}
%not really but I think this belongs here
\newcommand*{\A}{{\mathbb{A}}}

%asymptotic
\newcommand*{\bigO}{O}
\newcommand*{\smallo}{o}

%probability
\newcommand*{\prob}{\mathbb{P}}
\newcommand*{\E}{\mathbb{E}}

%vector calculus
\newcommand*{\gradient}{\V \nabla}
\newcommand*{\divergence}{\gradient \cdot}
\newcommand*{\curl}{\gradient \cdot}

%logic
\newcommand*{\yields}{\vdash}
\newcommand*{\nyields}{\nvdash}

%differential geometry
\renewcommand*{\H}{\mathbb{H}}
\newcommand*{\transversal}{\pitchfork}
\renewcommand{\d}{\mathrm{d}} % exterior derivative

%number theory
\newcommand*{\legendre}[2]{\genfrac{(}{)}{}{}{#1}{#2}}%Legendre symbol


\theoremstyle{definition}
\newtheorem*{postulate}{Postulate}

\begin{document}
\maketitle

The IB Quantum Mechanics course is a first course in quantum mechainics, which lays foundation for the subject and investigates some simple one-dimensional phenomenon via the Schr\"odinger picture.

This document is as a summary of the major results covered in this course.

\begin{defi}[Wavefunction]
  A \emph{wavefunction} is a function that describes the quantum state of a system.
\end{defi}

\begin{defi}[Inner product]
  Given \(\psi(x)\) and \(\phi(x)\), two normalisable wavefunction at some fixed time, the \emph{innder product} is
  \[
(\psi, \sigma) = \int_{-\infty}^{\infty} \psi(x)^*\phi(x) dx.
  \]
\end{defi}

\begin{defi}[Norm]
  The \emph{norm} of a wavefunction \(\psi(x)\) is
  \[
    \|\psi\|^2 = (\psi,\psi) = \int_{-\infty}^{\infty} |\psi(x)|^2 dx. 
  \]
\end{defi}
\begin{defi}[Operator/Observable]
  An operator acting on wavefunctions.
\end{defi}

\begin{itemize}
\item position: \(\hat x = x\),
\item momentum: \(\hat p = p\),
\item energy/Hamiltonian: \(H = -\frac{\hat p^2}{2m} + V(\hat x)\).
\end{itemize}

\begin{defi}[Expectation value]
  For any normalised wavefunction \(\psi(x)\) and operator \(Q\), the \emph{expectation value} is
  \[
  \langle Q\rangle_\psi = (\psi, Q\phi)
\]

\end{defi}

\begin{defi}
  A state \(\psi \neq 0\) is an \emph{eigenstate} or \emph{eigenfunction} of an operator or observable \(Q\) with \emph{eigenvalue} \(q\) if
  \[
Q\psi = q\psi.
  \]
\end{defi}


\begin{postulate}[P1]
  \label{postulate:1}
  A measurement of position gives a result with probability density $|\psi(x)|^2$. In other words, $|\psi(x)|^2 \delta x$ is the probability that the particle is found between $x$ and $x+\delta x$. Equivalently, $\int_a^b |\psi(x)|^2 dx$ is the probability that the particle is found in $[a,b]$.
\end{postulate}

\begin{postulate}[P2]
  \label{postulate:2}
  For any observable, $\langle Q\rangle_\psi$ is the mean result (expected value) if $Q$ is measured many times (as times $N\to\infty$) with particle in state $\psi$ before each measurement.
\end{postulate}

\begin{postulate}[P3]
  If \(Q\) is measured when the particle is an eigenstate \(\psi\) as above, then the results is the eigenvalue \(q\) with probability $1$.
\end{postulate}

\begin{defi}
The \emph{time-independent Schr\"odinger equation} for a particle of mass \(m\) in a potential \(CV(x)\) is the energy eigenvalue equation
\[
H\psi = -\frac{\hbar^2}{2m} \psi'' + V(x)\psi = E \psi
\]
\end{defi}
which describes the state at a particular time.

\begin{defi}
  The \emph{time-dependent Schr\"odinger equation} for a time-dependent wavefunction \(\Psi(x,t)\) is
  \[
    i\hbar \Psi = H\Psi.
  \]
\end{defi}

For this course we use \emph{separation of variables} and consider a particular class of solutions \(\Psi(x,t) = T(t)\psi(x)\) where \(\Psi(x,0) = \psi(x)\). The solution to \(H\Psi = E\Psi\) is 
\[
  \Psi(x,t) = \psi(x) \exp(-\frac{iEt}{\hbar}).
\]

\begin{defi}[Probability current]
  \[
    j(x,t) = -\frac{i\hbar}{2m}\Big( \Psi^* \frac{d\Psi}{dx} - \frac{\Psi^*}{dx}\Psi \Big)
  \]
\end{defi}

\begin{prop}
  The probability density \(P(x,t) = |\Psi(x,t)|^2\) satisfies the conservation equation
  \[
    \frac{\p P}{\p t} = - \frac{\p j}{\p x}.
  \]
\end{prop}

\begin{eg}\leavevmode
\begin{itemize}
\item Infinite potential wall
\item Potential well
\item Harmonic oscillator
\end{itemize}
\end{eg}

\begin{defi}[Uncertainty]
  The \emph{uncertainty} in position \((\Delta x)_\psi\) is
  \[
(\Delta x)_\psi^2 = \langle (\hat x - \langle \hat x \rangle_\psi)^2\rangle_\psi = \langle \hat x^2 \rangle_\psi - \langle \hat x\rangle_\psi^2.
  \]
\end{defi}

\begin{thm}[Ehrenfest's Theorem]
  \begin{align*}
    \frac{d}{dx} \langle \hat x\rangle_\Psi &= \frac{1}{m} \langle \hat p\rangle_\Psi \\
    \frac{d}{dt}\langle \hat p\rangle_\Psi &= - \langle V'(\hat x)\rangle_\Psi
  \end{align*}
\end{thm}

\begin{proof}
  \begin{align*}
    \frac{d}{dx}\langle\hat x\rangle_\Psi &= (\dot\Psi, \hat x \Psi) + (\Psi,\hat x\dot\Psi) \\
                                          &= (\frac{1}{i\hbar}H\Psi,\hat x\Psi) + (\Psi, \hat x \frac{1}{i\hbar}H\Psi) \\
                                          &= -\frac{1}{i\hbar}(\Psi,H\hat x\Psi) + \frac{1}{i\hbar}(\Psi, \hat x H \Psi) \\
                                          &= \frac{1}{i\hbar} \langle\Psi|[\hat x, H]|\Psi\rangle \\
                                          &= \frac{1}{i\hbar} \langle\Psi| \frac{i\hbar\hat p}{m}|\Psi\rangle
  \end{align*}
  The second relation is similar.
\end{proof}

\begin{thm}[Heisenbery's Uncertainty Principle]
  Let \(\psi(x)\) be a normalised state, then
  \[
(\delta x)_\psi (\delta p)_\psi \geq \frac{\hbar}{2}.
  \]
\end{thm}

\begin{eg}[Uncertainty for Gaussian wavepacket]
  Consider the normalised Gaussian wavepacket
  \[
    \psi(x) = \Big( \frac{1}{\alpha\pi} \Big)^{1/4} e^{-x^2/2\alpha}
  \]
  for which we have
  \begin{align*}
    \langle\hat x\rangle_\psi &= \langle\hat p\rangle_\psi = 0 \\
    \langle\hat x^2\rangle_\psi &= \frac{\alpha}{2} \\
    \langle\hat p^2\rangle_\psi &= \frac{\hbar^2}{2\alpha}
  \end{align*}
  so the lower bound is achieved.
\end{eg}

\begin{defi}[Commutator]
  Let \(Q, S\) be operators. The \emph{commutator} is \([Q,S] = QS-SQ\).
\end{defi}

\begin{eg}
  \begin{align*}
    [\hat x,\hat p] &= i\hbar \\
    [\hat x, H] &= \frac{i\hbar}{m}\hat p \\
    [\hat p, H] &= -i\hbar V'(\hat x)
  \end{align*}
\end{eg}

\begin{proof}[Proof of Uncertainty Principle]
  Define two new operators
  \[
X = \hat x - \langle\hat x\rangle_\psi, \: P = \hat p - \langle\hat p\rangle_\psi.
  \]
  Note that \([X,P] = [\hat x,\hat p] = i\hbar\). So
 \begin{align*}
   (\delta x)_\psi (\delta p)_\psi &= \|X x\|\cdot\|P x\| \\
                                   &\geq |(X\psi, P\psi)| \\
                                   &\geq |\im (X\psi, p\psi)| \\
                                   &\geq \frac{1}{2} |(X\psi, P\psi)-(P\psi,X\psi)| \\
                                   &= \frac{1}{2} |(\psi,XP\psi)-(\psi,PX\psi)| \\
                                   &= \frac{1}{2} |(\psi,[X,P]\psi)| \\
                                   &= \frac{\hbar}{2}
 \end{align*}
\end{proof}

\begin{defi}[Wavepacket]
  A wavepacket is a wavefunction that is localised in space.
\end{defi}

\begin{defi}[Gaussian wavepacket]
  A \emph{Guassian wavepacket} is
  \[
    \Psi(x,t) = \Big( \frac{1}{\alpha\pi} \Big)^{1/4}\frac{1}{\sqrt{\gamma(t)}} \exp \Big( -\frac{x^2}{2\gamma(t)} \Big)
  \]
  for some \(\gamma(t)\).
\end{defi}
It is a solution to time-dependent Schr\"odinger equation for \(\gamma(t) = \alpha+\frac{i\hbar}{m}t\).

A related solution to the equation is the moving particle
\[
  \Psi(x,t) = \Psi_0(x-ct,t) \exp \Big(-\frac{im}{\hbar}u \Big) \exp \Big(-\frac{imu^2}{\hbar^2}t \Big)
\]
which has probability density \( P(x,t) = P_0(x-ut,t) \) and momentum expectation value \(\langle \hat p\rangle_{\Psi_u} = mu\).

\end{document}