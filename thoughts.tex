\documentclass[a4paper]{article}

\def\ntitle{Thoughts During Tripos}

\ifx \nauthor\undefined
  \def\nauthor{Qiangru Kuang}
\else
\fi

\ifx \ntitle\undefined
  \def\ntitle{Template}
\else
\fi

\ifx \nauthoremail\undefined
  \def\nauthoremail{qk206@cam.ac.uk}
\else
\fi

\ifx \ndate\undefined
  \def\ndate{\today}
\else
\fi

\title{\ntitle}
\author{\nauthor}
\date{\ndate}

%\usepackage{microtype}
\usepackage{mathtools}
\usepackage{amsthm}
\usepackage{stmaryrd}%symbols used so far: \mapsfrom
\usepackage{empheq}
\usepackage{amssymb}
\let\mathbbalt\mathbb
\let\pitchforkold\pitchfork
\usepackage{unicode-math}
\let\mathbb\mathbbalt%reset to original \mathbb
\let\pitchfork\pitchforkold

\usepackage{imakeidx}
\makeindex[intoc]

%to address the problem that Latin modern doesn't have unicode support for setminus
%https://tex.stackexchange.com/a/55205/26707
\AtBeginDocument{\renewcommand*{\setminus}{\mathbin{\backslash}}}
\AtBeginDocument{\renewcommand*{\models}{\vDash}}%for \vDash is same size as \vdash but orginal \models is larger
\AtBeginDocument{\let\Re\relax}
\AtBeginDocument{\let\Im\relax}
\AtBeginDocument{\DeclareMathOperator{\Re}{Re}}
\AtBeginDocument{\DeclareMathOperator{\Im}{Im}}
\AtBeginDocument{\let\div\relax}
\AtBeginDocument{\DeclareMathOperator{\div}{div}}

\usepackage{tikz}
\usetikzlibrary{automata,positioning}
\usepackage{pgfplots}
%some preset styles
\pgfplotsset{compat=1.15}
\pgfplotsset{centre/.append style={axis x line=middle, axis y line=middle, xlabel={$x$}, ylabel={$y$}, axis equal}}
\usepackage{tikz-cd}
\usepackage{graphicx}
\usepackage{newunicodechar}

\usepackage{fancyhdr}

\fancypagestyle{mypagestyle}{
    \fancyhf{}
    \lhead{\emph{\nouppercase{\leftmark}}}
    \rhead{}
    \cfoot{\thepage}
}
\pagestyle{mypagestyle}

\usepackage{titlesec}
\newcommand{\sectionbreak}{\clearpage} % clear page after each section
\usepackage[perpage]{footmisc}
\usepackage{blindtext}

%\reallywidehat
%https://tex.stackexchange.com/a/101136/26707
\usepackage{scalerel,stackengine}
\stackMath
\newcommand\reallywidehat[1]{%
\savestack{\tmpbox}{\stretchto{%
  \scaleto{%
    \scalerel*[\widthof{\ensuremath{#1}}]{\kern-.6pt\bigwedge\kern-.6pt}%
    {\rule[-\textheight/2]{1ex}{\textheight}}%WIDTH-LIMITED BIG WEDGE
  }{\textheight}% 
}{0.5ex}}%
\stackon[1pt]{#1}{\tmpbox}%
}

%\usepackage{braket}
\usepackage{thmtools}%restate theorem
\usepackage{hyperref}

% https://en.wikibooks.org/wiki/LaTeX/Hyperlinks
\hypersetup{
    %bookmarks=true,
    unicode=true,
    pdftitle={\ntitle},
    pdfauthor={\nauthor},
    pdfsubject={Mathematics},
    pdfcreator={\nauthor},
    pdfproducer={\nauthor},
    pdfkeywords={math maths \ntitle},
    colorlinks=true,
    linkcolor={red!50!black},
    citecolor={blue!50!black},
    urlcolor={blue!80!black}
}

\usepackage{cleveref}



% TODO: mdframed often gives bad breaks that cause empty lines. Would like to switch to tcolorbox.
% The current workaround is to set innerbottommargin=0pt.

%\usepackage[theorems]{tcolorbox}





\usepackage[framemethod=tikz]{mdframed}
\mdfdefinestyle{leftbar}{
  %nobreak=true, %dirty hack
  linewidth=1.5pt,
  linecolor=gray,
  hidealllines=true,
  leftline=true,
  leftmargin=0pt,
  innerleftmargin=5pt,
  innerrightmargin=10pt,
  innertopmargin=-5pt,
  % innerbottommargin=5pt, % original
  innerbottommargin=0pt, % temporary hack 
}
%\newmdtheoremenv[style=leftbar]{theorem}{Theorem}[section]
%\newmdtheoremenv[style=leftbar]{proposition}[theorem]{proposition}
%\newmdtheoremenv[style=leftbar]{lemma}[theorem]{Lemma}
%\newmdtheoremenv[style=leftbar]{corollary}[theorem]{corollary}

\newtheorem{theorem}{Theorem}[section]
\newtheorem{proposition}[theorem]{Proposition}
\newtheorem{lemma}[theorem]{Lemma}
\newtheorem{corollary}[theorem]{Corollary}
\newtheorem{axiom}[theorem]{Axiom}
\newtheorem*{axiom*}{Axiom}

\surroundwithmdframed[style=leftbar]{theorem}
\surroundwithmdframed[style=leftbar]{proposition}
\surroundwithmdframed[style=leftbar]{lemma}
\surroundwithmdframed[style=leftbar]{corollary}
\surroundwithmdframed[style=leftbar]{axiom}
\surroundwithmdframed[style=leftbar]{axiom*}

\theoremstyle{definition}

\newtheorem*{definition}{Definition}
\surroundwithmdframed[style=leftbar]{definition}

\newtheorem*{slogan}{Slogan}
\newtheorem*{eg}{Example}
\newtheorem*{ex}{Exercise}
\newtheorem*{remark}{Remark}
\newtheorem*{notation}{Notation}
\newtheorem*{convention}{Convention}
\newtheorem*{assumption}{Assumption}
\newtheorem*{question}{Question}
\newtheorem*{answer}{Answer}
\newtheorem*{note}{Note}
\newtheorem*{application}{Application}

%operator macros

%basic
\DeclareMathOperator{\lcm}{lcm}

%matrix
\DeclareMathOperator{\tr}{tr}
\DeclareMathOperator{\Tr}{Tr}
\DeclareMathOperator{\adj}{adj}

%algebra
\DeclareMathOperator{\Hom}{Hom}
\DeclareMathOperator{\End}{End}
\DeclareMathOperator{\id}{id}
\DeclareMathOperator{\im}{im}
\DeclarePairedDelimiter{\generation}{\langle}{\rangle}

%groups
\DeclareMathOperator{\sym}{Sym}
\DeclareMathOperator{\sgn}{sgn}
\DeclareMathOperator{\inn}{Inn}
\DeclareMathOperator{\aut}{Aut}
\DeclareMathOperator{\GL}{GL}
\DeclareMathOperator{\SL}{SL}
\DeclareMathOperator{\PGL}{PGL}
\DeclareMathOperator{\PSL}{PSL}
\DeclareMathOperator{\SU}{SU}
\DeclareMathOperator{\UU}{U}
\DeclareMathOperator{\SO}{SO}
\DeclareMathOperator{\OO}{O}
\DeclareMathOperator{\PSU}{PSU}

%hyperbolic
\DeclareMathOperator{\sech}{sech}

%field, galois heory
\DeclareMathOperator{\ch}{ch}
\DeclareMathOperator{\gal}{Gal}
\DeclareMathOperator{\emb}{Emb}



%ceiling and floor
%https://tex.stackexchange.com/a/118217/26707
\DeclarePairedDelimiter\ceil{\lceil}{\rceil}
\DeclarePairedDelimiter\floor{\lfloor}{\rfloor}


\DeclarePairedDelimiter{\innerproduct}{\langle}{\rangle}

%\DeclarePairedDelimiterX{\norm}[1]{\lVert}{\rVert}{#1}
\DeclarePairedDelimiter{\norm}{\lVert}{\rVert}



%Dirac notation
%TODO: rewrite for variable number of arguments
\DeclarePairedDelimiterX{\braket}[2]{\langle}{\rangle}{#1 \delimsize\vert #2}
\DeclarePairedDelimiterX{\braketthree}[3]{\langle}{\rangle}{#1 \delimsize\vert #2 \delimsize\vert #3}

\DeclarePairedDelimiter{\bra}{\langle}{\rvert}
\DeclarePairedDelimiter{\ket}{\lvert}{\rangle}




%macros

%general

%divide, not divide
\newcommand*{\divides}{\mid}
\newcommand*{\ndivides}{\nmid}
%vector, i.e. mathbf
%https://tex.stackexchange.com/a/45746/26707
\newcommand*{\V}[1]{{\ensuremath{\symbf{#1}}}}
%closure
\newcommand*{\cl}[1]{\overline{#1}}
%conjugate
\newcommand*{\conj}[1]{\overline{#1}}
%set complement
\newcommand*{\stcomp}[1]{\overline{#1}}
\newcommand*{\compose}{\circ}
\newcommand*{\nto}{\nrightarrow}
\newcommand*{\p}{\partial}
%embed
\newcommand*{\embed}{\hookrightarrow}
%surjection
\newcommand*{\surj}{\twoheadrightarrow}
%power set
\newcommand*{\powerset}{\mathcal{P}}

%matrix
\newcommand*{\matrixring}{\mathcal{M}}

%groups
\newcommand*{\normal}{\trianglelefteq}
%rings
\newcommand*{\ideal}{\trianglelefteq}

%fields
\renewcommand*{\C}{{\mathbb{C}}}
\newcommand*{\R}{{\mathbb{R}}}
\newcommand*{\Q}{{\mathbb{Q}}}
\newcommand*{\Z}{{\mathbb{Z}}}
\newcommand*{\N}{{\mathbb{N}}}
\newcommand*{\F}{{\mathbb{F}}}
%not really but I think this belongs here
\newcommand*{\A}{{\mathbb{A}}}

%asymptotic
\newcommand*{\bigO}{O}
\newcommand*{\smallo}{o}

%probability
\newcommand*{\prob}{\mathbb{P}}
\newcommand*{\E}{\mathbb{E}}

%vector calculus
\newcommand*{\gradient}{\V \nabla}
\newcommand*{\divergence}{\gradient \cdot}
\newcommand*{\curl}{\gradient \cdot}

%logic
\newcommand*{\yields}{\vdash}
\newcommand*{\nyields}{\nvdash}

%differential geometry
\renewcommand*{\H}{\mathbb{H}}
\newcommand*{\transversal}{\pitchfork}
\renewcommand{\d}{\mathrm{d}} % exterior derivative

%number theory
\newcommand*{\legendre}[2]{\genfrac{(}{)}{}{}{#1}{#2}}%Legendre symbol


\begin{document}

\maketitle
\tableofcontents

\section{Set Theory}

\begin{itemize}
\item Codomain is a concept that takes time to appreciate. The inclusion of codomain in the definition of a function is a \href{https://math.stackexchange.com/a/65418/68241}{relatively new practice}. See also \hyperref[sec:homomorphism]{some thoughts} I jotted down when I was learning a first course in abstract algebra.
\item Equality is the \emph{finest} equivalence relation on any set $S$, in the sense that it is the relation that has the smallest equivalence classes (each is a singleton)
  
\end{itemize}

\section{Algebra}

\begin{itemize}
\item Why I found change of basis mind-boggling
  I took coordinate isomorphism for granted. A vector and its coordinates were the same thing to me, and a linear transformation and its matrix representation are the same thing. c.f. mathematical object v.s. (re)presentation
\item Some isomorphisms are taken for granted. For example, $\mathbb{Z}_n$ (the additive group of modulo $n$) and $\mathbb{Z}/n\mathbb{Z}$ (the additive group of a quotient ring of $\mathbb{Z}$) are different objects (at least from these constructions) but we almost always identify them automatically.
  \item Why is the ``order'' in an ordered field defined so? The motivation is to preserve order under addition and multiplication. Two structures on a set would be meaningless unless they interact, see \href{https://gowers.wordpress.com/2014/01/11/introduction-to-cambridge-ia-analysis-i-2014/}{Gowers' blog}.
\end{itemize}

\section{Philosophical}

\section{Observations \& Conjectures}

\begin{itemize}
\item limit inferior and limit superior
  \begin{itemize}
  \item limit in $\mathbb{R}$: in $\mathbb{R}$, given a sequence $(x_n)$,
    \begin{align*}
      \inf x_i \leq \sup\inf x_n &\leq \inf\sup x_n \leq \sup x_n \\
      \sup\inf (-x_n) &= -\inf\sup x_n\\
      (x_n) \; \text{converges } &\Leftrightarrow \sup\inf x_n = \inf\sup x_n
    \end{align*}

  \item limit in a $\sigma$-algebra $(\Omega, \mathcal{F})$: let $(A_n)$ be a sequence of events. $(\mathcal{F}, \subseteq)$ is a poset and $\inf A_n = \bigcap_{n=1}^\infty A_n,$ and $\sup A_n = \bigcup_{n=1}^\infty A_n$, with
    \begin{align*}
      \inf A_n \subseteq \sup\inf A_n &\subseteq \inf\sup A_n \subseteq \sup A_n \\
      \sup\inf \overline{A_n} &= \overline{\inf\sup A_n} \\
      (A_n) \: \text{converges } &\Leftrightarrow \mathtt{to be filled in}
    \end{align*}
  \end{itemize}
\end{itemize}
\section{Physics}

\begin{itemize}
\item Kinematics equation is just a Taylor expansion of $x(t)$, the function of displacement w.r.t. time.
\end{itemize}

\section{Miscellaneous}

\begin{itemize}
\item \href{https://en.wikipedia.org/wiki/Set_theory_(music)}{Set theory in music}, \href{https://en.wikipedia.org/wiki/Twelve-tone_technique}{twelve-tone technique}
  \item We can use orbit-stabiliser over the group of rotational symmetries to establish relations among numbers of edges, vertices and surfaces of a regular polyhedron: $V \cdot n_V = E \cdot n_E = F \cdot n_F$. Can this, together with other know relations such as $V+F=E+2$ be used to determine the number of regular polyhedrons?
\end{itemize}

\appendix

\section{My understanding of homomorphism}
\label{sec:homomorphism}

Homomorphism is really just a function\footnote{Categorically, group homomorphisms and ``naive'' functions are both elements of the $\mathbf{Hom}$ class between objects of the categories $\mathbf{Grp}$ and $\mathbf{Set}$, respectively.} --- no more, no less, don't have to be either injective or surjective. That homomorphism preserves algebraic structure is best understood as that the image of a homomorphism also forms such a structure. For example, $\phi(G)$ (i.e. $\im(\phi)$) is a group itself.

It would perhaps be useful to forget the codomain of a homomorphism for a while. Recall that when I first studied function (elementary functions such as $f(x) = 2x^2$ in secondary school), I learned domains, rules and images. Then suddenly came by the concept of codomain, and I was so confused: why would the concept of codomain ever be useful since the images are “better” characterisation of functions? But nevertheless I accepted the concept of codomain and found it useful later. 

It is exactly the same process when I study homomorphisms, which are a special kind of functions, except that codomain is introduced along with the definition of homomorphism, the lack of intuition of which is perhaps what leaves me nonplused. Some deal with it by memorising the definition and gaining intuition through practice. But if you find this approach unacceptable, do what you do when you were learning functions: forget the codomain, familiarise with the concepts, and finally add the codomain back and recognise its importance.
\end{document}
