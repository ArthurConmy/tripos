\documentclass[a4paper]{article}

\def\ntitle{Field Theory}

\ifx \nauthor\undefined
  \def\nauthor{Qiangru Kuang}
\else
\fi

\ifx \ntitle\undefined
  \def\ntitle{Template}
\else
\fi

\ifx \nauthoremail\undefined
  \def\nauthoremail{qk206@cam.ac.uk}
\else
\fi

\ifx \ndate\undefined
  \def\ndate{\today}
\else
\fi

\title{\ntitle}
\author{\nauthor}
\date{\ndate}

%\usepackage{microtype}
\usepackage{mathtools}
\usepackage{amsthm}
\usepackage{stmaryrd}%symbols used so far: \mapsfrom
\usepackage{empheq}
\usepackage{amssymb}
\let\mathbbalt\mathbb
\let\pitchforkold\pitchfork
\usepackage{unicode-math}
\let\mathbb\mathbbalt%reset to original \mathbb
\let\pitchfork\pitchforkold

\usepackage{imakeidx}
\makeindex[intoc]

%to address the problem that Latin modern doesn't have unicode support for setminus
%https://tex.stackexchange.com/a/55205/26707
\AtBeginDocument{\renewcommand*{\setminus}{\mathbin{\backslash}}}
\AtBeginDocument{\renewcommand*{\models}{\vDash}}%for \vDash is same size as \vdash but orginal \models is larger
\AtBeginDocument{\let\Re\relax}
\AtBeginDocument{\let\Im\relax}
\AtBeginDocument{\DeclareMathOperator{\Re}{Re}}
\AtBeginDocument{\DeclareMathOperator{\Im}{Im}}
\AtBeginDocument{\let\div\relax}
\AtBeginDocument{\DeclareMathOperator{\div}{div}}

\usepackage{tikz}
\usetikzlibrary{automata,positioning}
\usepackage{pgfplots}
%some preset styles
\pgfplotsset{compat=1.15}
\pgfplotsset{centre/.append style={axis x line=middle, axis y line=middle, xlabel={$x$}, ylabel={$y$}, axis equal}}
\usepackage{tikz-cd}
\usepackage{graphicx}
\usepackage{newunicodechar}

\usepackage{fancyhdr}

\fancypagestyle{mypagestyle}{
    \fancyhf{}
    \lhead{\emph{\nouppercase{\leftmark}}}
    \rhead{}
    \cfoot{\thepage}
}
\pagestyle{mypagestyle}

\usepackage{titlesec}
\newcommand{\sectionbreak}{\clearpage} % clear page after each section
\usepackage[perpage]{footmisc}
\usepackage{blindtext}

%\reallywidehat
%https://tex.stackexchange.com/a/101136/26707
\usepackage{scalerel,stackengine}
\stackMath
\newcommand\reallywidehat[1]{%
\savestack{\tmpbox}{\stretchto{%
  \scaleto{%
    \scalerel*[\widthof{\ensuremath{#1}}]{\kern-.6pt\bigwedge\kern-.6pt}%
    {\rule[-\textheight/2]{1ex}{\textheight}}%WIDTH-LIMITED BIG WEDGE
  }{\textheight}% 
}{0.5ex}}%
\stackon[1pt]{#1}{\tmpbox}%
}

%\usepackage{braket}
\usepackage{thmtools}%restate theorem
\usepackage{hyperref}

% https://en.wikibooks.org/wiki/LaTeX/Hyperlinks
\hypersetup{
    %bookmarks=true,
    unicode=true,
    pdftitle={\ntitle},
    pdfauthor={\nauthor},
    pdfsubject={Mathematics},
    pdfcreator={\nauthor},
    pdfproducer={\nauthor},
    pdfkeywords={math maths \ntitle},
    colorlinks=true,
    linkcolor={red!50!black},
    citecolor={blue!50!black},
    urlcolor={blue!80!black}
}

\usepackage{cleveref}



% TODO: mdframed often gives bad breaks that cause empty lines. Would like to switch to tcolorbox.
% The current workaround is to set innerbottommargin=0pt.

%\usepackage[theorems]{tcolorbox}





\usepackage[framemethod=tikz]{mdframed}
\mdfdefinestyle{leftbar}{
  %nobreak=true, %dirty hack
  linewidth=1.5pt,
  linecolor=gray,
  hidealllines=true,
  leftline=true,
  leftmargin=0pt,
  innerleftmargin=5pt,
  innerrightmargin=10pt,
  innertopmargin=-5pt,
  % innerbottommargin=5pt, % original
  innerbottommargin=0pt, % temporary hack 
}
%\newmdtheoremenv[style=leftbar]{theorem}{Theorem}[section]
%\newmdtheoremenv[style=leftbar]{proposition}[theorem]{proposition}
%\newmdtheoremenv[style=leftbar]{lemma}[theorem]{Lemma}
%\newmdtheoremenv[style=leftbar]{corollary}[theorem]{corollary}

\newtheorem{theorem}{Theorem}[section]
\newtheorem{proposition}[theorem]{Proposition}
\newtheorem{lemma}[theorem]{Lemma}
\newtheorem{corollary}[theorem]{Corollary}
\newtheorem{axiom}[theorem]{Axiom}
\newtheorem*{axiom*}{Axiom}

\surroundwithmdframed[style=leftbar]{theorem}
\surroundwithmdframed[style=leftbar]{proposition}
\surroundwithmdframed[style=leftbar]{lemma}
\surroundwithmdframed[style=leftbar]{corollary}
\surroundwithmdframed[style=leftbar]{axiom}
\surroundwithmdframed[style=leftbar]{axiom*}

\theoremstyle{definition}

\newtheorem*{definition}{Definition}
\surroundwithmdframed[style=leftbar]{definition}

\newtheorem*{slogan}{Slogan}
\newtheorem*{eg}{Example}
\newtheorem*{ex}{Exercise}
\newtheorem*{remark}{Remark}
\newtheorem*{notation}{Notation}
\newtheorem*{convention}{Convention}
\newtheorem*{assumption}{Assumption}
\newtheorem*{question}{Question}
\newtheorem*{answer}{Answer}
\newtheorem*{note}{Note}
\newtheorem*{application}{Application}

%operator macros

%basic
\DeclareMathOperator{\lcm}{lcm}

%matrix
\DeclareMathOperator{\tr}{tr}
\DeclareMathOperator{\Tr}{Tr}
\DeclareMathOperator{\adj}{adj}

%algebra
\DeclareMathOperator{\Hom}{Hom}
\DeclareMathOperator{\End}{End}
\DeclareMathOperator{\id}{id}
\DeclareMathOperator{\im}{im}
\DeclarePairedDelimiter{\generation}{\langle}{\rangle}

%groups
\DeclareMathOperator{\sym}{Sym}
\DeclareMathOperator{\sgn}{sgn}
\DeclareMathOperator{\inn}{Inn}
\DeclareMathOperator{\aut}{Aut}
\DeclareMathOperator{\GL}{GL}
\DeclareMathOperator{\SL}{SL}
\DeclareMathOperator{\PGL}{PGL}
\DeclareMathOperator{\PSL}{PSL}
\DeclareMathOperator{\SU}{SU}
\DeclareMathOperator{\UU}{U}
\DeclareMathOperator{\SO}{SO}
\DeclareMathOperator{\OO}{O}
\DeclareMathOperator{\PSU}{PSU}

%hyperbolic
\DeclareMathOperator{\sech}{sech}

%field, galois heory
\DeclareMathOperator{\ch}{ch}
\DeclareMathOperator{\gal}{Gal}
\DeclareMathOperator{\emb}{Emb}



%ceiling and floor
%https://tex.stackexchange.com/a/118217/26707
\DeclarePairedDelimiter\ceil{\lceil}{\rceil}
\DeclarePairedDelimiter\floor{\lfloor}{\rfloor}


\DeclarePairedDelimiter{\innerproduct}{\langle}{\rangle}

%\DeclarePairedDelimiterX{\norm}[1]{\lVert}{\rVert}{#1}
\DeclarePairedDelimiter{\norm}{\lVert}{\rVert}



%Dirac notation
%TODO: rewrite for variable number of arguments
\DeclarePairedDelimiterX{\braket}[2]{\langle}{\rangle}{#1 \delimsize\vert #2}
\DeclarePairedDelimiterX{\braketthree}[3]{\langle}{\rangle}{#1 \delimsize\vert #2 \delimsize\vert #3}

\DeclarePairedDelimiter{\bra}{\langle}{\rvert}
\DeclarePairedDelimiter{\ket}{\lvert}{\rangle}




%macros

%general

%divide, not divide
\newcommand*{\divides}{\mid}
\newcommand*{\ndivides}{\nmid}
%vector, i.e. mathbf
%https://tex.stackexchange.com/a/45746/26707
\newcommand*{\V}[1]{{\ensuremath{\symbf{#1}}}}
%closure
\newcommand*{\cl}[1]{\overline{#1}}
%conjugate
\newcommand*{\conj}[1]{\overline{#1}}
%set complement
\newcommand*{\stcomp}[1]{\overline{#1}}
\newcommand*{\compose}{\circ}
\newcommand*{\nto}{\nrightarrow}
\newcommand*{\p}{\partial}
%embed
\newcommand*{\embed}{\hookrightarrow}
%surjection
\newcommand*{\surj}{\twoheadrightarrow}
%power set
\newcommand*{\powerset}{\mathcal{P}}

%matrix
\newcommand*{\matrixring}{\mathcal{M}}

%groups
\newcommand*{\normal}{\trianglelefteq}
%rings
\newcommand*{\ideal}{\trianglelefteq}

%fields
\renewcommand*{\C}{{\mathbb{C}}}
\newcommand*{\R}{{\mathbb{R}}}
\newcommand*{\Q}{{\mathbb{Q}}}
\newcommand*{\Z}{{\mathbb{Z}}}
\newcommand*{\N}{{\mathbb{N}}}
\newcommand*{\F}{{\mathbb{F}}}
%not really but I think this belongs here
\newcommand*{\A}{{\mathbb{A}}}

%asymptotic
\newcommand*{\bigO}{O}
\newcommand*{\smallo}{o}

%probability
\newcommand*{\prob}{\mathbb{P}}
\newcommand*{\E}{\mathbb{E}}

%vector calculus
\newcommand*{\gradient}{\V \nabla}
\newcommand*{\divergence}{\gradient \cdot}
\newcommand*{\curl}{\gradient \cdot}

%logic
\newcommand*{\yields}{\vdash}
\newcommand*{\nyields}{\nvdash}

%differential geometry
\renewcommand*{\H}{\mathbb{H}}
\newcommand*{\transversal}{\pitchfork}
\renewcommand{\d}{\mathrm{d}} % exterior derivative

%number theory
\newcommand*{\legendre}[2]{\genfrac{(}{)}{}{}{#1}{#2}}%Legendre symbol


\begin{document}

\maketitle

\tableofcontents

\section{Field Extensions}

\begin{definition}
  Given a field $F$, there is a unique ring homomorphism $\phi: \mathbb{Z} \rightarrow F, 1 \mapsto 1$. Let $\ker \phi = n\mathbb{Z}$, then the \emph{characteristic} of $F$ $\ch(F) = n$.
\end{definition}

\begin{definition}
  Fiven a field $F$, the \emph{prime subfield} of $F$ is the subfield generated by $1_F$.
\end{definition}

The prime subfield of any field is isomorphic to either $\mathbb{Q}$ or $\mathbb{F}_p$ for some prime $p$.

\begin{definition}
  If $K$ is a field containing a subfield $F$, then $K$ is an extension of $F$, denoted $K/F$.
\end{definition}

\begin{definition}
  The \emph{degree of the extension} $K/F$, $[K:F] := \dim_FK$. The extension is finite if $[K:F]$ is finite, otherwise infinite.
\end{definition}

\begin{proposition}
  Let $p(x) \in F[x]$ be irreducible. Then $K:=F[x]/(p(x))\supseteq F$ and $p$ has a root in $K$.
\end{proposition}

\begin{proposition}
  Let $p(x) \in F[x]$ be irreducible of degree $n$, $K:=F[x]/(p(x)),\: \theta= x \mod{p(x)} \in K$. Then $\{\theta^i\}_{i=0}^{n-1}$ is a basis for $K/F$, so $[K:F]=n$ and $K=\{\sum_{i=0}^{n-1} a_i \theta^i:a_i \in F\}$.
\end{proposition}

\begin{proof}\leavevmode
  \begin{itemize}
  \item $F[x]$ is a Euclidean domain
  \item $p$ is irreducible
  \end{itemize}
\end{proof}

\begin{definition}
  Suppose $K/F$ and $\alpha,\beta,\ldots \in K$. Then the \emph{smallest subfield} containing both $F$ and $\alpha,\beta,\ldots$, $F(\alpha,\beta,\ldots)$ is called the field \emph{generated} by $\alpha,\beta,\ldots$ over $F$.
\end{definition}

\begin{definition}
  If $K=F(\alpha)$ then $K$ is a \emph{simple extension} of $F$ and $\alpha$ is a \emph{primitive element} for $K/F$.
\end{definition}

\begin{proposition}
  Suppose $p(x) \in F[x]$ is irreducible. Suppose $K/F$ and $\alpha\in K$ such that $p(\alpha)=0$. Then $F(\alpha) \cong F[x]/(p(x))$.
\end{proposition}

\begin{proof}
  Since $p(\alpha)=0$, we have the following induced homomorphism
  \[
    \begin{tikzcd}
      F[x] \arrow{r}{\phi} \arrow{d}[swap]{\pi} & F(\alpha)\\
      F[x]/(p(x)) \arrow[dotted]{ru}[swap]{\bar\phi}
    \end{tikzcd}
  \]

  As $F \leq \im(\bar\phi) \leq F(\alpha)$ and $\alpha\in \im(\bar\phi)$, by definition $\im(\bar\phi)=F(\alpha)$.
\end{proof}

\begin{remark}
  Roots of an irreducible $p(x)$ are algebraically indistinguishable.
\end{remark}

\begin{theorem}\label{thm:uniqueness of simple extension}
  Suppose $\phi: F\xrightarrow{\sim} F',\: p(x)\in F[x]$ be irreducible and $p'(x):=\phi(p(x))$. Suppose $\alpha,\beta$ are roots of $p$ and $p'$ respectively. Then $\sigma: F(\alpha) \rightarrow F(\beta),\:\alpha \mapsto \beta$ is an isomorphism and
  \[
    \begin{tikzcd}
      F \arrow{r}{\sim}[swap]{\phi} \arrow[hook]{d} & F' \arrow[hook]{d}\\
      F(\alpha) \arrow{r}{\sim}[swap]{\sigma} & F'(\beta)
    \end{tikzcd}
  \]
\end{theorem}

\begin{proof}
  $\phi$ as a map $F[x] \rightarrow F'[x]$ maps irreducible to irreducible so
  \[
    \begin{tikzcd}
      F[x]/(p(x)) \arrow{r}{\sim} \arrow{d} & F'[x]/(p'(x)) \arrow{d}\\
      F(\alpha) \arrow{r} &F(\beta)
    \end{tikzcd}
  \]
\end{proof}

\section{Algebraic Extensions}

Let $F$ be a field and $K$ be an extension of $F$.

\begin{definition}
  The element $\alpha\in K$ is said to be \emph{algebraic} over $F$ is $\alpha$ is a root of some nonzero polynomial $f(x)\in F[x]$. If $\alpha$ is not algebraic over $F$ then $\alpha$ is said to be \emph{transcendental} over $F$. The extension $K/F$ is said to be \emph{algebraic} if every element of $K$ is algebraic over $F$.
\end{definition}

\begin{proposition}
  Let $\alpha$ be algebraic over $F$. Then there is a unique irreducible polynomial $m_{\alpha,F}(x)\in F[x]$ which has $\alpha$ as a root. A polynomial $f(x)\in F[x]$ has $\alpha$ as a root iff $m_{\alpha,F}(x)$ divides $f(x)$ in $F[x]$.
\end{proposition}

\begin{proof}\leavevmode
  \begin{itemize}
  \item Let $g(x)\in F[x]$ be a monic polynomial of minimal degree having $\alpha$ as a root.
  \item $F[x]$ is a Euclidean domain
  \end{itemize}
\end{proof}

\begin{definition}
  The polynomial $m_{\alpha,F}$ is called the \emph{minimal polynomial} for $\alpha$ over $F$. The \emph{degree} of $m_\alpha$ is called the \emph{degree} of $\alpha$.
\end{definition}

\begin{proposition}
  Let $\alpha$ be algebraic over $F$, then
  \[
    F(\alpha) \cong F[x]/(m_\alpha(x))
  \]
  so in particular
  \[
    [F(\alpha):F] = \deg m_\alpha(x) = \deg \alpha.
  \]
\end{proposition}

\begin{proposition}
  The element $\alpha$ is algebraic over $F$ iff the simple extension $F(\alpha)/F$ is finite.
\end{proposition}

\begin{proof}
  Use the fact that $\{\alpha^i\}_{i=0}^{\deg \alpha}$ is a basis for $F(\alpha)/F$ when the extension is finite.
\end{proof}

\begin{corollary}
  A finite extension is algebraic.
\end{corollary}

\begin{eg}[Quadratic Extensions over Fields of Characteristic $\neq 2$]

  \texttt{to be filled in}
\end{eg}

\begin{theorem}[Tower Law]
  Let $F\subseteq K\subseteq L$ be fields. Then
  \[
    [L:F] = [L:K][K:F]
  \]
\end{theorem}

\begin{definition}
  An extension $K/F$ is \emph{finitely generated} if there are elements $\{\alpha_i\}_{i=1}^k$ in $K$ such that $K=F(\alpha_1,\ldots,\alpha_k)$.
\end{definition}

\begin{lemma}
  $F(\alpha,\beta) = (F(\alpha))(\beta)$
\end{lemma}

\begin{theorem}
  The extension $K/F$ is finite iff $K$ is generated by a finite number of algebraic elements over $F$. More precisely,
  \[
    [F(\alpha_1,\ldots,\alpha_k):F] \leq \prod_{i=1}^{k} \deg(\alpha_i)
  \]
\end{theorem}

\begin{corollary}
  Suppose $\alpha,\beta$ are algebraic over $F$. Then $\alpha\pm\beta,\alpha\beta,\alpha/\beta$ are all algebraic as they are elements of the extenion $F(\alpha,\beta)$.
\end{corollary}

\begin{corollary}
  Let $L/F$ be an arbitrary extension. Then the collection of elements of $L$ that are algebraic over $F$ form a subfield $K$ of $L$.
\end{corollary}

\begin{theorem}[Transitivity of Algebraic Extension]
  IF $K$ is algebraic over $F$ and $L$ is algebraic over $K$ then $L$ is algebraic over $F$.
\end{theorem}

\begin{proof}
  Let $\alpha\in L$, which is algebraic over $K$ so $\alpha$ satisfies some polynomial
  \[
    \sum_{i=0}^{n} a_i \alpha^i = 0
  \]

  with $a_i\in K$. So
  \[
    [F(\alpha,a_0,\ldots,a_n):F] = [F(\alpha,a_0,\ldots,a_n):F(a_0,\ldots,a_n)][F(a_o,\ldots,a_n):F]
  \]
  which is finite.
\end{proof}

\begin{definition}
  Let $K_1$ and $K_2$ be two subfields of a field $K$. The \emph{composite field} of $K_1$ and $K_2$, denoted $K_1K_2$ is the smallest subfield of $K$ containing both $K_1$ and $K_2$.
\end{definition}

\begin{proposition}\label{prop:composite}
  Let $K_1$ and $K_2$ be two finite extensions of a field $F$ contained in $K$. Then
  \[
    [K_1K_2:F] \leq [K_1:F][K_2:F]
  \]
\end{proposition}

\section{Classical Straightedge and Compass Constructions}

Let $1$ denote a fixed given unit distance, then any distance is determined by its length $a\in \mathbb{R}$. The collection of lengths that can be obtained by compass and straightedge constructions from a unit distance is called the \emph{constructible} elements of $\mathbb{R}$.

\begin{proposition}
  The collection of constructible elements is a \emph{subfield} of $\mathbb{R}$ strictly larger than $\mathbb{Q}$.
\end{proposition}

We can construct $\sqrt a$ for any $a > 0$. We can also show, by obtaining the coordinates of lines and circles by solving their equations, that any operation on elements of $F$ produces elements in at most a \emph{quadratic} extension of $F$. Thus

\begin{proposition}
  If the element $\alpha\in\mathbb{R}$ is obtained from a field $F \subset \mathbb{R}$ by a series of operations then $[F(\alpha):F]=2^k$ for some integer $k \geq 0$.
\end{proposition}

\begin{proposition}
  Trisecting an angle is impossible since $\cos 20^{\circ}$ is not contructible.
\end{proposition}

\begin{proof}
  Let $\beta=\cos20^\circ$. By triple angle formula we have
  \[
    4\beta^3 -3\beta - 1/2 = 0
  \]
  Let $\alpha=2\beta$, then $\alpha$ satisfies the equation
  \[
    \alpha^3-3\alpha-1=0
  \]
  which can be shown to have no rational roots by Rational Root Theorem. Thus $[\mathbb{Q}(\alpha):\mathbb{Q}]=3$.
\end{proof}

\begin{remark}
  The angles $1^\circ$ and $2^\circ$ are not constructible as otherwise addition formula show that $20^\circ$ would be constructible. The regular pentagon gives $72^\circ$ and the equilateral triangle gives $60^\circ$ so $3^\circ$ is constructible.
\end{remark}

\section{Splitting Fields and Algebraic Closures}

\begin{definition}
  The extension field $K$ over $F$ is called a \emph{splitting field} for the polynomial $f(x)\in F[x]$ if $f(x)$ factors completely into linear factors in $K[x]$ and $f(x)$ does not factor completely into linear factors over any proper subfield of $K$ containing $F$.
\end{definition}

\begin{theorem}
  For any field $F$, if $f(x)\in F[x]$ then there exists an extension $K$ of $F$ which is a splitting field for $f(x)$.
\end{theorem}

\begin{proposition}
  A splitting field of polynomial of degree $n$ over $F$ is of degree at most $n!$ over $F$.
\end{proposition}

\begin{definition}
  If $K$ is an algebraic extension of $F$ which is the splitting field over $F$ for a collection of polynomials $f(x)\in F[x]$ then $K$ is called a \emph{normal} extension of $F$.
\end{definition}

\begin{eg}[Splitting Field of $x^n-1$: Cyclotomic Fields]
  Consider the splittinf field of the polynomail $x^n-1$ over $\mathbb{Q}$. The roots of the polynomial are called the $n$th \emph{roots of unity}.

  Since any finite group of the multiplicative group of a field is cyclic, we call a generator of the cyclic group of all $n$th roots of unity a \emph{primitive} $n$th root of unity, denoted $\zeta_n$. Given a primitive root $\zeta_n$, the other primitive roots are then $\zeta_n^i$ where $1 \leq i < n$ is an integer relatively prime to $n$. Thus there are $\varphi(n)$ primitive roots of unity. Over $\mathbb{C}$ we can see this directly be letting $\zeta_n:=e^{2\pi i/n}$.

  The field $\mathbb{Q}(\zeta_n)$ is called the \emph{cyclotomic field of $n$th roots of unity}.

  When $n=p$ is a prime, we have the factorisation
  \[
    x^p-1 = (x-1)(x^{p-1}+\cdots+x+1)
  \]
  Since $\zeta_p \neq 1$ it is a root of the polynomial
  \[
    \Phi_p(x) = \frac{x^p-1}{x-1} = (x-1)(x^{p-1}+\cdots+x+1)
  \]
  which is irreducible. Thus $\Phi_p$ is the irreducible polynomial of $\zeta_p$ over $\mathbb{Q}$ so
  \[[\mathbb{Q}(\zeta_p):\mathbb{Q}] = p-1\]

  See more discussion of in Section \ref{sec:cyclotomic}.
\end{eg}

\begin{eg}[Splitting Field of $x^p-2$, $p$ is a prime]
  If $\alpha$ is a root of the equation $x^p-2$ then $\zeta\alpha$ is alos a root where $\zeta$ is a $p$th root of unity. Denote the positive real $p$th root of $2$ $\sqrt[p]2$. We can show that the splitting field is exactly $\mathbb{Q}(\sqrt[p]2, \zeta_p)$. The extension field has $\mathbb{Q}(\sqrt[p]2)$ and $\mathbb{Q}(\zeta_p)$ as subfields so by Proposition \ref{prop:composite}
  \[ [\mathbb{Q}(\sqrt[p]2,\zeta_p):\mathbb{Q}] = p(p-1) \]
\end{eg}

\begin{theorem}
  Let $\phi: F\xrightarrow{\sim} F'$ be an isomorphism of fields. Let $f(x)\in F[x]$ be a polynomial and $f'(x):=\phi(f(x))\in F'[x]$. Let $E$ be a splitting field for $f(x)$ over $F$ and let $E'$ be a splitting field for $f'(x)$ over $F'$. Then the isomorpism $\phi$ extends to an isomorphism $\sigma:E\xrightarrow{\sim} E'$:
  \[
    \begin{tikzcd}
      F \arrow{r}{\sim}[swap]{\phi} \arrow[hook]{d} & F' \arrow[hook]{d}\\
      E \arrow{r}{\sim}[swap]{\sigma} & E'
    \end{tikzcd}
  \]
\end{theorem}

\begin{proof}
  Induction on the degree $n$ of $f(x)$. The base case where $f(x)$ splits completely in $F[x]$ is easy. Suppose now that $p(x)$ is an irreducible factor of $f(x)$ of degree at least $2$. Let $\alpha\in E$ be a root of $p(x)$ and $\beta\in E'$ be a root of $p'(x)$. Then by Theorem \ref{thm:uniqueness of simple extension} we have
  \[
    \begin{tikzcd}
      F \arrow{r}{\sim}[swap]{\phi} \arrow[hook]{d} & F' \arrow[hook]{d}\\
      F(\alpha) \arrow{r}{\sim}[swap]{\sigma'} & F(\beta)
    \end{tikzcd}
  \]

  Now we have $f(x) = (x-\alpha)f_1(x)$ over $F(\alpha)[x]$ where $f_1(x)$ ahs degree $n-1$. Note that the field $E$ is a splitting field for $f_1(x)$ over $F(\alpha)[x]$: all the roots of $f_1(x)$ are in $E$ and if they were contained in any smaller extension $L$ containing $F(\alpha)[x]$, then since $F(\alpha)$ contains $\alpha$, $L$ would also contain all roots of $f(x)$, which would contradict the minimality of $E$ as the splitting field of $f(x)$ over $F$. Similar for $E'$. Since $f_1(x)$ has degree less than $n$, by induction there exist $\sigma: E\xrightarrow{\sim} E'$ such that
  \[
    \begin{tikzcd}
      F \arrow{r}{\sim}[swap]{\phi} \arrow[hook]{d} & F' \arrow[hook]{d}\\
      F(\alpha) \arrow{r}{\sim}[swap]{\sigma'} \arrow[hook]{d} & F(\beta) \arrow[hook]{d}\\
      E \arrow{r}{\sim}[swap]{\sigma} & E'
    \end{tikzcd}
  \]
\end{proof}

\begin{corollary}[Uniqueness of Splitting Fields]
  Any two splitting fields for a polynomial $f(x)\in F[x]$ over a field $F$ are isomorphic.
\end{corollary}

\begin{definition}
  The field $\cl F$ is called an \emph{algebraic closure} of $F$ if $\cl F$ is algebraic over $F$ and if every polynomial $f(x)\in F[x]$ splits completely over $\cl F$.
\end{definition}

\begin{definition}
  A field $K$ is said to be \emph{algebraically closed} if every polynomial with coefficients in $K$ has a root in $K$.
\end{definition}

$K=\cl K$ iff $K$ is algebraically closed.

\begin{proposition}
  Let $\cl F$ be an algebraic closure of $F$. Then $\cl F$ is algebraically closed.
\end{proposition}

\begin{proof}
  Let $f(x)\in\cl F[x]$ and $\alpha$ be a root of $f(x)$. Then
  \[ F \subseteq \cl F[x] \subseteq \cl F[x] \]
  is an algebraic extension by transitivity so $\alpha\in\cl F$.
\end{proof}

\begin{proposition}
  For any field $F$ there exists an algebraically closed field $K$ containing $F$.
\end{proposition}

\begin{proof}
  For every nonconstant monic polynomial $f$ with coefficients in $F$, let $x_f$ denote an indeterminate and consider the polynomial ring $F[\ldots,x_f,\ldots]$. Consider the ideal $I$ generated by the polynomials $f(x_f)$. Suppose this ideal is not proper, then there is a relation
  \[ \sum_{i=1}^n g_i f_i(x_{f_i}) = 1 \]
  where $g_i$ are elements of $F[\ldots,x_f,\ldots]$. For $i=1,2,\ldots,n$ let $x_{f_i}=x_i$ and let $x_{n+1},\ldots,x_m$ be the remaining variables occurring in the polynomials $g_j$. Then the relation becomes
  \[ \sum_{i=1}^n g_i(x_1,\ldots,x_m) f_i(x_i) = 1 \]

  Let $F'$ be a finite extension of $F$ containing a root $\alpha_i$ of $f_i$. Letting $x_i = \alpha_i$ and setting $x_{n+1}=\cdots=x_m=0$. Then the relation above reads $0=1$. Absurd.

  Since $I$ is a proper ideal, by Zorn's Lemma it is contained in a maximal ideal $M$. Then
  \[ F \subseteq K_1:= F[\ldots,x_f,\ldots]/M \]
  and each polynomial $f$ has a root in $K_1$ by construction. Iterate the process and we obtain a chain of fields
  \[ F= K_0 \subseteq K_1 \subseteq \cdots \]
  Let
  \[ K = \bigcup_{i\geq0} K_i \]
  which is a field containing $F$ and $K$ is algebraically closed.
\end{proof}

\begin{proposition}
  Let $K$ be an algebraically closed field and let $F$ be a subfield of $K$. Then the collection of elements $\cl F$ of $K$ that are algebraic over $F$ is an algebraic closure of $F$. An algebraic closure of $F$ is unique up to isomorphism.
\end{proposition}

\begin{theorem}[Fundamental Theorem of Algebra]
  $\mathbb{C}$ is algebraically closed.
\end{theorem}

\begin{corollary}
  $\mathbb{C}$ contains an algebraic closure for any of its subfields. In particular, $\cl{\mathbb{Q}}$, the collection of complex numbers algebraic over $\mathbb{Q}$, is an algebraic closure of $\mathbb{Q}$.
\end{corollary}

\section{Separable Extensions}

\begin{definition}
  A polynomial over $F$ is called \emph{separable} if it has no multiple roots. A polynomial which is not separable is called \emph{inseparable}.
\end{definition}

\begin{definition}
  The \emph{derivative} of the polynomial
  \[ f(x)=a_n x^n+a_{n-1}+x^{n-1}+\cdots + a_1 x + a_0 \in F[x] \]
  is defined to be the polynomial
  \[ D_x f(x) := n a_n x^{n-1} + (n-1) a_{n-1} x^{n-2}+\cdots + 2a_2 x + a_1 \in F[x] \]
\end{definition}

\begin{proposition}
  A polynomial $f(x)$ has a multiple root $\alpha$ iff $\alpha$ is also a root of $D_xf(x)$, i.e. both $f(x)$ and $D_xf(x)$ are divisible by the minimal polynomial for $\alpha$. In particular $f(x)$ is separable iff it is relatively prime to its derivative.
\end{proposition}

\begin{eg}
  The polynomial $x^{p^n}-x$ over $\mathbb{F}_p$ has derivative $-1$ so it is separable.
\end{eg}

\begin{eg}
  The polynomial $x^n-1$ has derivative $nx^{n-1}$. Over any field of characteristic not dividing $n$ this polynomial has the only root $0$, which is not a root of $x^n-1$. Thus $x^n-1$ is separable and there are $n$ distinct $n$th root of unity.
\end{eg}

\begin{corollary}
  Every irreducible polynomial over a field of characteristic $0$ is separable. A polynomial over such a field is separable iff it is the product of distinct irreducible polynomials.
\end{corollary}

\begin{proof}
  Suppose $F$ is a field of characteristic $0$ and $p(x)\in F[x]$ is irreducible of degree $n$. Then $D_xf(x)$ has degree $n-1$ which is coprime to $p(x)$.
\end{proof}

\begin{remark}
  In characteristic $p$ the derivative of any power of $x^p$ is identically $0$:
  \[ D_x(x^{pm}) = pmx^{pm-1} = 0 \]
  so it is possible for the degree of the derivative to decrease by more than one. Thus the proof in the corollary may fail.
\end{remark}

\begin{proposition}\label{prop:frobenius}
  Let $F$ be a field of characteristic $p$. Then for any $a,b\in F$,
  \begin{align*}
    (a+b)^p &= a^p+b^p\\
    (ab)^p &= a^p b^p
  \end{align*}
  so the map $a \mapsto a^p$ is an injective field endomorphism of $F$.
\end{proposition}

\begin{definition}
  The map in Proposition \ref{prop:frobenius} is called the \emph{Frobenius endomorphism} of $F$.
\end{definition}

\begin{corollary}\label{cor:perfect field}
  Suppose $\mathbb{F}$ is a finite field of characteristic $p$. Then every element is a $p$th power in $\mathbb F$, i.e. $\mathbb F = \mathbb F^p$.
\end{corollary}

\begin{proposition}
  Every irreducible polynomial over a finite field $\mathbb F$ is separable. A polynomial in $\mathbb F$ is separable iff it is the product of distinct irreducible polynomials in $\mathbb F[x]$.
\end{proposition}

\begin{proof}
  Let $\mathbb F$ be a finite field and suppose that $p(x)\in\mathbb F$ is an irreducible. If $p(x)$ were inseparable then $p(x)=q(x^p)$ for some polynomial $q(x)\in \mathbb F$. Let
  \[ q(x) = \sum_{i=0}^m a_i x^i \]
  By Corollary \ref{cor:perfect field}, there exist $b_i\in \mathbb F$ such that $a_i= b_i^p$. Then by Proposition \ref{prop:frobenius}
  \begin{align*}
    p(x)= q(x^p) &= \sum_{i=0}^m a_i (x^p)^i\\
                 &= \sum_{i=0}^m b_i^p (x^p)^i\\
                 &= \sum_{i=0}^m (b_i x^i)^p\\
                 &= (\sum_{i=0}^m b_i x^i)^p
  \end{align*}
  which show that $p$ is the $p$th power of a polynomial, a contradiction to the irreducibility of $f$.
\end{proof}

\begin{definition}
  A field $K$ of characteristic $p$ is called \emph{perfect} if every element of $K$ is a $p$th power in $K$, i.e. $K = K^p$. Any field of characteristic $0$ is also called perfect.
\end{definition}

\begin{eg}[Existence and Uniqueness of Finite Fields]
  Let $n$ be a positive integer and consider the splitting field of $x^{p^n}-x$ over $\mathbb F_p$. We have shown that it is separable, thus having $p^n$ roots. For any two roots we can show their sum, product and quotient are also a root by Proposition \ref{prop:frobenius}. Let $\mathbb F$ be the \emph{field} of all such roots, which must be the splitting field. Thus we have $[\mathbb F:\mathbb F_p] = n$.

  For the uniqueness, suppose $\mathbb F$ is any finite field of characteristic $p$. Suppose $[\mathbb F:\mathbb F_p] = n$. Since the multiplicative group $\mathbb F^\times$ has order $p^n - 1$, we have $\alpha^{p^n}-\alpha=0$ for every $\alpha\in\mathbb F$, so by counting argument $\mathbb F$ is a splitting field of $x^{p^n}-x$. Since splitting fields are unique up to isomorphism, we have proven that \emph{finite fields of any order $p^n$ exist and are unique up to isomorphism.} We denote the field $\mathbb F_{p^n}$.
\end{eg}

\begin{proposition}\label{prop:sep}
  Let $p(x)$ be an irreducible polynomial over a field $F$ of characteristic $p$. Then there is a unique integer $k \geq 0$ and a unique irreducible separable polynomial $p_{sep}(x)\in F[x]$ such that
  \[ p(x) = p_{sep}(x^{p^k}) \].
\end{proposition}

\begin{definition}
  Use the notation in Proposition \ref{prop:sep}, the degree of $p_{sep}(x)$ is called the \emph{separable degree} of $p(x)$, denoted $\deg_s p(x)$ while $p^k$ is called the \emph{inseparable degree} of $p(x)$, denoted $\deg_i p(x)$.
\end{definition}

From the relation $p(x) = p_{sep}(x^{p^k})$ we have
\[ \deg p(x) = \deg_s p(x) \deg_i p(x). \]

\begin{definition}
  The field $K$ is said to be \emph{separable} (or \emph{separably algebraic}) over $F$ if every element of $K$ is the root of a separable polynomial over $F$. A field which is not separable is \emph{inseparable}.
\end{definition}

\begin{corollary}
  Every finite extension of a perfect field is separable.
\end{corollary}

\section{Cyclotomic Polynomials and Extensions} \label{sec:cyclotomic}

\begin{definition}
  Let $\mu_n$ denote the \emph{group of $n$th roots of unity over $\mathbb Q$}.
\end{definition}

If $d$ is a divisor of $n$ and $\zeta$ is a $d$th root of unity then $\zeta$ is also a n $n$th root of unity so
\[ \mu_d \subseteq \mu_n \quad \text{for all } d|n. \]

Conversely, any element of $\mu_n$ which is also a $d$th root of unity has order $d$ which divides $|\mu_n|=n$.

\begin{definition}
  The \emph{$n$th cyclotomic polynomial $\Phi_n(x)$} is the polynomial whose roots are the primitive $n$th roots of unity:
  \[ \Phi_n(x) := \prod_{\text{primitive }\zeta\in \mu_n} (x - \zeta) = \prod_{\substack{1\leq a <n\\ (a,n)=1}} (x-\zeta_n^a) \]
\end{definition}

Since the roots of $x^n-1$ are precisely the $n$th roots of unity, we have the factorisaton
\begin{align*}
  x^n-1 &= \prod_{\zeta \in \mu_n} (x-\zeta)\\
        &= \prod_{d|n} \prod_{\substack{\zeta \in \mu_d\\\zeta\: \text{primitive}}} (x - \zeta) \quad \text{group together $\zeta$ by order $d$}\\
        &= \prod_{d|n} \Phi_d(x) \quad \text{definition of $\Phi_n(x)$}
\end{align*}

which allows us to compute $\Phi_n(x)$ recursively. Note comparing the degree gives the identity
\[ n = \sum_{d|n} \varphi(d). \]

\begin{lemma}
  The cyclotomic polynomial $\Phi_n(x)$ is a monic polynomial in $\mathbb Z[x]$ of degree $\varphi(n)$.
\end{lemma}

\begin{proof}
  It is clear that $\Phi_n(x)$ is monic and has degree $\varphi(n)$. To show the coefficients lie in $\mathbb Z$ we use induction on $n$. The base case is easy. Asumme it is true for all $1 \leq d < n$. Then $x^n-1 = f(x) \Phi_n(x)$ where $f(x) = \prod_{d|n,d\neq n} \Phi_d(x)$ is monic and has coefficients in $\mathbb Z$. Since $f(x)$ divides $x^n - 1$ in $F[x]$ where $F=\mathbb Q(\zeta_n)$ and both $f(x)$ and $x^n - 1$ have coefficients in $\mathbb Q$, $f(x)$ divides $x^n - 1$ in $\mathbb Q[x]$ by division algorithm (divisibility is independent of the ring). By Gauss' Lemma, $f(x)$ divides $x^n -1 $ in $\mathbb Z[x]$.
\end{proof}

\begin{theorem}
  The cyclotomic polynomial $\Phi_n(x)$ is an irreducible monic polynomial in $\mathbb Z[x]$ of degree $\varphi(n)$.
\end{theorem}

\begin{proof}
  To show that $\Phi_n(x)$ is irreducible, suppose we have a factorisation in $\mathbb Z[x]$
  \[ \Phi_n(x) = f(x)g(x), \]
  where we take $f(x)$ to be an irreducible factor of $\Phi_n(x)$. Let $\zeta$ be a primitive $n$th root of unity for which $f(x)$ is the minimal polynomial. Let $p$ be any prime not dividing $n$. Then $\zeta^p$ is also a primitive $n$th root of unity. Suppose it is a root of $g(x)$. Then $\zeta$ is a root of $g(x^p)$ and so
  \[ g(x^p) = f(x)h(x), \: h(x) \in \mathbb Z[x] \]

  Reduce modulo $p$, we obtain
  \[ (\bar g(x))^p = \bar g(x^p) = \bar f(x) \bar g(x) \quad \text{in } \mathbb F_p[x]. \]
  Since $\mathbb F_p[x]$ is a U.F.D., it folows that $\bar f(x)$ and $\bar g(x)$ have a factor in common.

  Now reduce $\Phi_n(x)$ modulo $p$, it follows that $\bar \Phi_n(x) \in \mathbb F_p[x]$ has a multiple root. But then $x^n-1$ is a multiple of $\bar \Phi_n(x)$ so would have a multiple root over $\mathbb F_p$, which is a contradiction since $p \nmid n$.

  So it must be the case that $\zeta^p$ is a root of $f(x)$, for every root $\zeta$ of $f(x)$. Write every integer $a$ coprime to $n$ as a product of primes $a = p_1p_2\cdots p_k$, then $\zeta^{p_1}$ is a root of $f(x)$, $\zeta^{p_1p_2} = (\zeta^{p_1})^{p_2}$ is a root of $f(x)$, etc. Then every primitive $n$th root of unity is a root of $f(x)$. So $f(x) = \Phi_n(x)$.
\end{proof}

\begin{corollary}
  \[ [\mathbb Q(\zeta_n): \mathbb Q] = \varphi(n) \]
\end{corollary}

\begin{proof}
  $\Phi_n(x)$ is the minimal polynomial for any primitive $n$th root of unity $\zeta_n$.
\end{proof}
\end{document}