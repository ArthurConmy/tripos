\documentclass[a4paper]{article}

\def\ntitle{Galois Theory}

\ifx \nauthor\undefined
  \def\nauthor{Qiangru Kuang}
\else
\fi

\ifx \ntitle\undefined
  \def\ntitle{Template}
\else
\fi

\ifx \nauthoremail\undefined
  \def\nauthoremail{qk206@cam.ac.uk}
\else
\fi

\ifx \ndate\undefined
  \def\ndate{\today}
\else
\fi

\title{\ntitle}
\author{\nauthor}
\date{\ndate}

%\usepackage{microtype}
\usepackage{mathtools}
\usepackage{amsthm}
\usepackage{stmaryrd}%symbols used so far: \mapsfrom
\usepackage{empheq}
\usepackage{amssymb}
\let\mathbbalt\mathbb
\let\pitchforkold\pitchfork
\usepackage{unicode-math}
\let\mathbb\mathbbalt%reset to original \mathbb
\let\pitchfork\pitchforkold

\usepackage{imakeidx}
\makeindex[intoc]

%to address the problem that Latin modern doesn't have unicode support for setminus
%https://tex.stackexchange.com/a/55205/26707
\AtBeginDocument{\renewcommand*{\setminus}{\mathbin{\backslash}}}
\AtBeginDocument{\renewcommand*{\models}{\vDash}}%for \vDash is same size as \vdash but orginal \models is larger
\AtBeginDocument{\let\Re\relax}
\AtBeginDocument{\let\Im\relax}
\AtBeginDocument{\DeclareMathOperator{\Re}{Re}}
\AtBeginDocument{\DeclareMathOperator{\Im}{Im}}
\AtBeginDocument{\let\div\relax}
\AtBeginDocument{\DeclareMathOperator{\div}{div}}

\usepackage{tikz}
\usetikzlibrary{automata,positioning}
\usepackage{pgfplots}
%some preset styles
\pgfplotsset{compat=1.15}
\pgfplotsset{centre/.append style={axis x line=middle, axis y line=middle, xlabel={$x$}, ylabel={$y$}, axis equal}}
\usepackage{tikz-cd}
\usepackage{graphicx}
\usepackage{newunicodechar}

\usepackage{fancyhdr}

\fancypagestyle{mypagestyle}{
    \fancyhf{}
    \lhead{\emph{\nouppercase{\leftmark}}}
    \rhead{}
    \cfoot{\thepage}
}
\pagestyle{mypagestyle}

\usepackage{titlesec}
\newcommand{\sectionbreak}{\clearpage} % clear page after each section
\usepackage[perpage]{footmisc}
\usepackage{blindtext}

%\reallywidehat
%https://tex.stackexchange.com/a/101136/26707
\usepackage{scalerel,stackengine}
\stackMath
\newcommand\reallywidehat[1]{%
\savestack{\tmpbox}{\stretchto{%
  \scaleto{%
    \scalerel*[\widthof{\ensuremath{#1}}]{\kern-.6pt\bigwedge\kern-.6pt}%
    {\rule[-\textheight/2]{1ex}{\textheight}}%WIDTH-LIMITED BIG WEDGE
  }{\textheight}% 
}{0.5ex}}%
\stackon[1pt]{#1}{\tmpbox}%
}

%\usepackage{braket}
\usepackage{thmtools}%restate theorem
\usepackage{hyperref}

% https://en.wikibooks.org/wiki/LaTeX/Hyperlinks
\hypersetup{
    %bookmarks=true,
    unicode=true,
    pdftitle={\ntitle},
    pdfauthor={\nauthor},
    pdfsubject={Mathematics},
    pdfcreator={\nauthor},
    pdfproducer={\nauthor},
    pdfkeywords={math maths \ntitle},
    colorlinks=true,
    linkcolor={red!50!black},
    citecolor={blue!50!black},
    urlcolor={blue!80!black}
}

\usepackage{cleveref}



% TODO: mdframed often gives bad breaks that cause empty lines. Would like to switch to tcolorbox.
% The current workaround is to set innerbottommargin=0pt.

%\usepackage[theorems]{tcolorbox}





\usepackage[framemethod=tikz]{mdframed}
\mdfdefinestyle{leftbar}{
  %nobreak=true, %dirty hack
  linewidth=1.5pt,
  linecolor=gray,
  hidealllines=true,
  leftline=true,
  leftmargin=0pt,
  innerleftmargin=5pt,
  innerrightmargin=10pt,
  innertopmargin=-5pt,
  % innerbottommargin=5pt, % original
  innerbottommargin=0pt, % temporary hack 
}
%\newmdtheoremenv[style=leftbar]{theorem}{Theorem}[section]
%\newmdtheoremenv[style=leftbar]{proposition}[theorem]{proposition}
%\newmdtheoremenv[style=leftbar]{lemma}[theorem]{Lemma}
%\newmdtheoremenv[style=leftbar]{corollary}[theorem]{corollary}

\newtheorem{theorem}{Theorem}[section]
\newtheorem{proposition}[theorem]{Proposition}
\newtheorem{lemma}[theorem]{Lemma}
\newtheorem{corollary}[theorem]{Corollary}
\newtheorem{axiom}[theorem]{Axiom}
\newtheorem*{axiom*}{Axiom}

\surroundwithmdframed[style=leftbar]{theorem}
\surroundwithmdframed[style=leftbar]{proposition}
\surroundwithmdframed[style=leftbar]{lemma}
\surroundwithmdframed[style=leftbar]{corollary}
\surroundwithmdframed[style=leftbar]{axiom}
\surroundwithmdframed[style=leftbar]{axiom*}

\theoremstyle{definition}

\newtheorem*{definition}{Definition}
\surroundwithmdframed[style=leftbar]{definition}

\newtheorem*{slogan}{Slogan}
\newtheorem*{eg}{Example}
\newtheorem*{ex}{Exercise}
\newtheorem*{remark}{Remark}
\newtheorem*{notation}{Notation}
\newtheorem*{convention}{Convention}
\newtheorem*{assumption}{Assumption}
\newtheorem*{question}{Question}
\newtheorem*{answer}{Answer}
\newtheorem*{note}{Note}
\newtheorem*{application}{Application}

%operator macros

%basic
\DeclareMathOperator{\lcm}{lcm}

%matrix
\DeclareMathOperator{\tr}{tr}
\DeclareMathOperator{\Tr}{Tr}
\DeclareMathOperator{\adj}{adj}

%algebra
\DeclareMathOperator{\Hom}{Hom}
\DeclareMathOperator{\End}{End}
\DeclareMathOperator{\id}{id}
\DeclareMathOperator{\im}{im}
\DeclarePairedDelimiter{\generation}{\langle}{\rangle}

%groups
\DeclareMathOperator{\sym}{Sym}
\DeclareMathOperator{\sgn}{sgn}
\DeclareMathOperator{\inn}{Inn}
\DeclareMathOperator{\aut}{Aut}
\DeclareMathOperator{\GL}{GL}
\DeclareMathOperator{\SL}{SL}
\DeclareMathOperator{\PGL}{PGL}
\DeclareMathOperator{\PSL}{PSL}
\DeclareMathOperator{\SU}{SU}
\DeclareMathOperator{\UU}{U}
\DeclareMathOperator{\SO}{SO}
\DeclareMathOperator{\OO}{O}
\DeclareMathOperator{\PSU}{PSU}

%hyperbolic
\DeclareMathOperator{\sech}{sech}

%field, galois heory
\DeclareMathOperator{\ch}{ch}
\DeclareMathOperator{\gal}{Gal}
\DeclareMathOperator{\emb}{Emb}



%ceiling and floor
%https://tex.stackexchange.com/a/118217/26707
\DeclarePairedDelimiter\ceil{\lceil}{\rceil}
\DeclarePairedDelimiter\floor{\lfloor}{\rfloor}


\DeclarePairedDelimiter{\innerproduct}{\langle}{\rangle}

%\DeclarePairedDelimiterX{\norm}[1]{\lVert}{\rVert}{#1}
\DeclarePairedDelimiter{\norm}{\lVert}{\rVert}



%Dirac notation
%TODO: rewrite for variable number of arguments
\DeclarePairedDelimiterX{\braket}[2]{\langle}{\rangle}{#1 \delimsize\vert #2}
\DeclarePairedDelimiterX{\braketthree}[3]{\langle}{\rangle}{#1 \delimsize\vert #2 \delimsize\vert #3}

\DeclarePairedDelimiter{\bra}{\langle}{\rvert}
\DeclarePairedDelimiter{\ket}{\lvert}{\rangle}




%macros

%general

%divide, not divide
\newcommand*{\divides}{\mid}
\newcommand*{\ndivides}{\nmid}
%vector, i.e. mathbf
%https://tex.stackexchange.com/a/45746/26707
\newcommand*{\V}[1]{{\ensuremath{\symbf{#1}}}}
%closure
\newcommand*{\cl}[1]{\overline{#1}}
%conjugate
\newcommand*{\conj}[1]{\overline{#1}}
%set complement
\newcommand*{\stcomp}[1]{\overline{#1}}
\newcommand*{\compose}{\circ}
\newcommand*{\nto}{\nrightarrow}
\newcommand*{\p}{\partial}
%embed
\newcommand*{\embed}{\hookrightarrow}
%surjection
\newcommand*{\surj}{\twoheadrightarrow}
%power set
\newcommand*{\powerset}{\mathcal{P}}

%matrix
\newcommand*{\matrixring}{\mathcal{M}}

%groups
\newcommand*{\normal}{\trianglelefteq}
%rings
\newcommand*{\ideal}{\trianglelefteq}

%fields
\renewcommand*{\C}{{\mathbb{C}}}
\newcommand*{\R}{{\mathbb{R}}}
\newcommand*{\Q}{{\mathbb{Q}}}
\newcommand*{\Z}{{\mathbb{Z}}}
\newcommand*{\N}{{\mathbb{N}}}
\newcommand*{\F}{{\mathbb{F}}}
%not really but I think this belongs here
\newcommand*{\A}{{\mathbb{A}}}

%asymptotic
\newcommand*{\bigO}{O}
\newcommand*{\smallo}{o}

%probability
\newcommand*{\prob}{\mathbb{P}}
\newcommand*{\E}{\mathbb{E}}

%vector calculus
\newcommand*{\gradient}{\V \nabla}
\newcommand*{\divergence}{\gradient \cdot}
\newcommand*{\curl}{\gradient \cdot}

%logic
\newcommand*{\yields}{\vdash}
\newcommand*{\nyields}{\nvdash}

%differential geometry
\renewcommand*{\H}{\mathbb{H}}
\newcommand*{\transversal}{\pitchfork}
\renewcommand{\d}{\mathrm{d}} % exterior derivative

%number theory
\newcommand*{\legendre}[2]{\genfrac{(}{)}{}{}{#1}{#2}}%Legendre symbol


\begin{document}

\maketitle

\tableofcontents

\section{Basic Definitions}

\begin{defi}
    $\aut(K/F) := \{\sigma \in \aut(K): \sigma|_F = \id \}$
\end{defi}

$\sigma \in \aut(K)$ always fixes its \emph{prime subfield} so e.g. $\aut(\mathbb{R}) = \aut(\mathbb{R}/\mathbb{Q})$

\begin{prop}
  \label{prop:permutation}
    Given $K/F$, $\alpha \in K$ algebraic over $F$, $\forall\sigma \in \aut(K/F)$, $\sigma\alpha$ is also a root of the minimal polynomial, i.e. $\sigma$ permutates the roots of the minimal polynomial.
\end{prop}

$\sigma \in \aut(K/F)$ is completely determined by what it does to the generators of $K$. If $K/F$ is finite then $\aut(K/F)$ is finite.

\begin{prop}
    Given $H \leq \aut(K)$, the \emph{fixed field of $H$} $F := \{\alpha \in K:\forall\sigma \in H, \sigma\alpha = \alpha\}$ is a subfield of $K$.
\end{prop}

Actually, given any $S \subseteq \aut(K), F_S = F_{\langle S \rangle} \leq K$.

\begin{prop}
    The association of the group to fields and fields to group is inclusion reversing, namely

    \begin{enumerate}
        \item if $F_1 \leq F_2 \leq K$ are two subfields of $K$ then $\aut(K/F_2) \leq \aut(K/F_1)$, and
        \item if $H_1 \leq H_2 \leq \aut(K)$ are two subgroups of automorphisms with associated subfield $F_1$ and $F_2$ then $F_2 \leq F_1$
    \end{enumerate}

\end{prop}

Let $F$ be a field and $E$ be the splitting field over $F$ of $f(x) \in F[x]$. Given any isomorphism $\phi: F \rightarrow F'$, it can be extended to an isomorphism $\sigma: E \rightarrow E'$ where $E'$ is the splitting field for $f'(x) = \phi(f(x)) \in F'[x]$.

We now show by induction on $[E : F]$ that the number of such extensions is at most $[E : F]$, with equality if $f(x)$ is separable over $F$.

If $[E : F] = 1$ then $E = F, E' = F', \sigma = \phi$ and the number of extensions is $1$. If $[E : F] > 1$ then $f(x)$ has at least one irreducible factor $p(x)$ of degree larger than $1$ with corresponding irreducible factor $p'(x)$ of $f'(x)$. Let $\alpha$ be a fixed root of $p(x)$. If $\sigma$ is any extension of $\phi$ to $E$, then $\sigma|_{F(\alpha)}$ is an isomorphism $\tau$ of $F(\alpha)$ with some subfields of $E'$. The isomorphism $\tau$ is completely determined by its action on $\alpha$, i.e. $\tau\alpha$. Thus we have the diagram
\[
  \begin{tikzcd}
    F \arrow{r}{\sim}[swap]{\phi} \arrow[hook]{d} & F' \arrow[hook]{d}\\
    F(\alpha) \arrow{r}{\sim}[swap]{\tau} \arrow[hook]{d} & F(\beta) \arrow[hook]{d}\\
    E \arrow{r}{\sim}[swap]{\sigma} & E'
  \end{tikzcd}
\]

Conversely, for any $\beta$ a root of $p'(x)$ there are extensions $\tau$ and $\sigma$ giving such a diagram. Hence to count the number of extensions $\sigma$ we need only count the possible number of these diagrams.

The number of extensions of $\phi$ to $\tau$ is equal to the number of distinct roots of $\beta$ of $p'(x)$. Since the degree of $p(x)$ and $p'(x)$ are both equal to $[F(\alpha) : F]$, we see that the number of extensions of $\phi$ to a $\tau$ is at most $[F(\alpha) : F]$, with equality if the roots of $p(x)$ are distinct.

Since $E$ is also the splitting field of $f(x)$ over $F(\alpha)$, $E'$ is the splitting field of $f'(x)$ over $F(\beta)$, and $[E : F(\alpha)] < [E : F]$, we may apply our induction hypothesis to these field extensions. By induction, the number of extensions of $\tau$ to $\sigma$ is at most $[E : F(\alpha)]$, with equality if $f(x)$ has distinct roots.

From $[E : F] = [E: F(\alpha)] [F(\alpha) : F]$ it follows that the number of extensions of $\phi$ to $\sigma$ is at most $[E : F]$, equality if $p(x)$ and $f(x)$ have distinct roots. Since $p(x)$ is a factor of $f(x)$, it is equaivalent to $f(x)$ having distinct roots.

In the particular case when $F = F'$ and $\phi$ is identity we have:

\begin{prop}
    Let $E$ be a splitting field over $F$ of the polynomial $f(x) \in F[x]$, then
    \[
        |\aut(E/F)| \leq [E : F]
    \]

    with equality if $f(x)$ is separable over $F$.
\end{prop}

\begin{defi}
    Let $K/F$ be a finite extension. Then $K$ is said to be \emph{Galois} over $F$ and $K/F$ is a \emph{Galois extension} if $|\aut(K/F)| = [K : F]$. If $K/F$ is Galois the group $\aut(K/F)$ is called the \emph{Galois group} of $K/F$, denoted $\gal(K/F)$.
\end{defi}

\begin{cor}
    If $K$ is a splitting field over $F$ of a separable polynomial $f(x)$ then $K/F$ is Galois.
\end{cor}

Note that the corollary implies that \emph{the splitting field of any polynomial $f(x)$ over a perfect field is Galois}, since the splitting field of $f(x)$ is the same as the splitting field of the product of the irreducible factors of $f(x)$ (i.e. the ``square free'' part), which is separable.

\begin{defi}
    If $f(x)$ is a separable polynomial over $F$, then the \emph{Galois group of $f(x)$ over $F$} is the Galois group of the splitting field of $f(x)$ over $F$.
\end{defi}

\begin{eg}
    \begin{enumerate}
        \item $\gal(\mathbb{Q}(\sqrt 2)/\mathbb{Q}) = \langle \sigma \rangle \cong C_2$ where $\sigma: a + b \sqrt 2 \mapsto a - b \sqrt 2$.
        \item More generally, any quadratic $K$ of any field $F$ of characteristic different from $2$ is Galois.
        \item $\mathbb{Q}(\sqrt[3] 2)/\mathbb{Q}$ is not Galois since its group of automorphism is trivial.
        \item $\mathbb{Q}(\sqrt 2, \sqrt 3)/\mathbb{Q}$ is Galois since it is the splitting field of $(x^2 - 2)(x^2 - 3)$. Any automorphism $\sigma$ is completely determined by its action on the generators $\sqrt 2$ and $\sqrt 3$, which must be mapped to $\pm 2$ and $\pm 3$ respectively. We can find $\gal(\mathbb{Q}(\sqrt 2, \sqrt 3)/\mathbb{Q}) = \langle \sigma, \tau \rangle \cong C_2^2$ where $\sigma:\sqrt 2 \mapsto \sqrt 2, \sqrt 3 \mapsto -\sqrt 3$ and $\tau: \sqrt 2 \mapsto -\sqrt 2, \sqrt 3 \mapsto \sqrt 3$.
        \item The splitting field of $x^3 - 2$ over $\mathbb{Q}$ $\mathbb{Q}(\sqrt[3] 2, \zeta_3 \sqrt[3] 2) \cong \mathbb{Q}(\sqrt[3] 2, \zeta_3)$ is Galois over $\mathbb{Q}$ with $\gal(\mathbb{Q}(\sqrt[3] 2, \zeta_3)/\mathbb{Q}) \cong S_3$.
        \item Galois extension is not transitive. Consider
        \[
            \overbrace{\underbrace{\mathbb{Q} \leq \mathbb{Q}}_2\underbrace{(\sqrt 2) \leq \mathbb{Q}(\sqrt[4] 2)}_2}^4,
        \]
        both quadratic extensions are Galois but $\mathbb{Q}(\sqrt[4] 2)$ is not Galois over $\mathbb{Q}$ since only two of $\{\pm \sqrt[4] 2, \pm i \sqrt[4] 2\}$ are elements of the field.
        \item $\mathbb F_{p^n}/\mathbb F_p$ is Galois since it is the splitting field of $x^{p^n} - x$. $\sigma_p: \mathbb F_{p^n} \rightarrow F_{p^n}, \alpha \mapsto \alpha^p$ is call the \emph{Frobenius automorphism} of the finite field. Since $\forall\alpha,\: \sigma_p^i(\alpha) = \alpha^{p^i}$, $\alpha^{p^n} = \alpha$, so $\sigma_p^n = 1$. No lower power can be the identity, since this would imply $\alpha^{p^i} = \alpha$ for all $\alpha \in \mathbb F_{p^n}$ for some $i < n$, which is impossible since the equation has at most $p^i$ roots. Thus $\gal(\mathbb F_{p^n}/\mathbb F_p) = \langle \sigma_p \rangle \cong C_n$.
        \item The inseparable extension $\mathbb F_2(x)/\mathbb F_2(t)$ where $x^2 - t = 0$ is not Galois.
    \end{enumerate}
\end{eg}

\section{The Fundamental Theorem of Galois Theory}
There is a correspondence between the lattice of subgroups of a Galois group and subfields of the extension field. The Fundamental Theorem of Galois Theory states that the relations are not conincidental and hold for any Galois extension. We first introduce \emph{group characters}:

\begin{defi}
    A \emph{character} $\chi$ of a group $G$ with values in a field $L$ is a homomorphism from $G$ to the multiplicative group of $L$:
    \[
        \chi: G \rightarrow L^\times
    \]
\end{defi}

\begin{defi}
    The characters $\chi_1, \chi_2, \ldots \chi_n$ of $G$ are said to be \emph{linearly independent} over $L$ is they are linearly independent as functions on $G$, i.e. if there is no nontrivial relation
    \[
        a_1 \chi_1 + a_2 \chi_2 + \cdots a_n \chi_n = 0
    \]
\end{defi}

\begin{thm}[Linear Independence of Characters]
    If $\chi_1, \chi_2, \ldots \chi_n$ are distinct characters of $G$ with values in $L$ then they are linearly independent over $L$.
\end{thm}

\begin{proof}
    Suppose there is a counterexample with minimal number $m$ of nonzero coefficients $a_i$. Wlog suppose
    \[
        \sum_{i=1}^m a_i \chi_i = 0
    \]
    Then for any $g \in G$ we have
    \[
        \sum_{i=1}^m a_i \chi_i(g) = 0
    \]
    Let $g_0$ be an element s.t. $\chi_1(g_0) ≠ \chi_m(g_0)$ (since $\chi_1$ and $\chi_m$ are distinct). Then
    \[
        \sum_{i=1}^m a_i \chi_i(g_0 g) = \sum_{i=1}^m a_i \chi_i(g_0) \chi_i(g) = 0
    \]
    Multiply and subtract the first equation to get
    \[
        \sum_{i=1}^{m-1} (\chi_m(g_0) - \chi_i(g_0))a_i \chi_i(g) = 0
    \]
    Since the first coefficient if nonzero this is a relation with fewer than $m$ nonzero coefficients. Contradiction.
\end{proof}

An injective homomorphism of a field $K$ into a field $L$ is called an \emph{embedding} of $K$ into $L$.

\begin{cor}
  If $\sigma_1, \sigma_2, \ldots \sigma_n$ are distinct embeddings of a field $K$ into a field $L$, then they are linearly independent as funcrions on $K$. In particular distinct automorphisms of a field $K$ are linearly independent.
\end{cor}

\begin{thm}\label{thm:degree of fixed field}
  Let $G = \{\sigma_1=1, \sigma_2,\ldots, \sigma_n\}$ be a subgroup of automorphisms of a field $K$ and let $F$ be its fixed field. Then
  \[ [K:F] = n = |G| \]
\end{thm}

\begin{proof}
  \texttt{to be filled in}
\end{proof}

\begin{cor}
  \label{cor:upper bound of order of aut group}
  Let $K/F$ be a finite extension. Then
  \[ |\aut(K/F)| \leq [K:F] \]
  with equality iff $F$ is the fixed field of $\aut(K/F)$. In other words, $K/F$ is Galois iff $F$ is the fixed field of $\aut(K/F)$.
\end{cor}

\begin{proof}
  Let $F_1$ be the fixed field of $\aut(K/F)$, so that
  \[ F \subseteq F_1 \subseteq K. \]
  By Theorem \ref{thm:degree of fixed field}, $[K:F] = [K:F_1][F_1:F] = |\aut(K/F)| [F_1:F]$.
\end{proof}

\begin{cor}\label{cor:galois group cor}
  Let $G \leq \aut(K)$ be finite and let $F$ be its fixed subfield. Then $\aut(K/F) = G$, so $K/F$ is Galois with Galois group $G$.
\end{cor}

\begin{proof}
  Since $G \leq \aut(K/F)$, $|G| \leq |\aut(K/F)|$. From Theorem \ref{thm:degree of fixed field} $|G| = [K:F]$ and from Corollary \ref{cor:galois group cor} $|\aut(K/F)| \leq [K:F]$ so
  \[ [K:F] = |G| \leq |\aut(K/F)| \leq [K:F] \]

  and it follows that we have equalities throughout.
\end{proof}

\begin{cor}
  \label{cor:unique fixed field}
  If $G_1 \neq G_2$ are two distinct finite subgroups of $\aut(K)$ then their fxed fixed fields are also distinct.
\end{cor}

\begin{proof}
  Let $F_1$ and $F_2$ be their fixed fields respectively. If $F_1 = F_2$ then $F_1$ is fixed by $G_2$. Thus By the previous corollary $G_2 \leq G_1$. Same argrument shows $G_1 \leq G_2$.
\end{proof}

\begin{thm}
  \label{thm:characterisation of galois extension}
  The extension $K/F$ is Galois iff $K$ is the splitting field of some separable polynomial over $F$. In this case every irreducible polynomial with coefficients in $F$ which has a root in $K$ is separable and has all its roots in $K$ (so in particular $K/F$ is a separable extension).
\end{thm}

\begin{proof}
  We first show that if $K/F$ is Galois then every irreducible polynomial $p(x) \in F[x]$ having a root in $K$ splits completely in $K[x]$. Let $G = \gal(K/F) = \{1, \sigma_2,\ldots,\sigma_n\}$. Let $\alpha \in K$ be a root of $p(x)$ and consider the elements
  \[ \alpha, \sigma_2(\alpha),\ldots, \sigma_n(\alpha) \]
  Let $\alpha, \alpha_2,\ldots,\alpha_r$ be \emph{distinct} elements. Give $\tau \in G$, since $G$ is a group the elements $\{\tau, \tau\sigma_2,\ldots,\tau\sigma_n\}$ are the same as $\{1,\sigma_2,\ldots,\sigma_n\}$. Thus applying $\tau$ to the elements simply permutes them. The polynomial
  \[ f(x) = (x-\alpha) (x-\alpha_2)\cdots (x-\alpha_r) \]
  therefore has coefficients whihc are fixed by all elements of $G$. Thus the coefficients lie in $F$, the fixed field of $G$. Since $p(x)$ is irreducible and has $\alpha$ as a root it is the minimal polynomial for $\alpha$ so $p(x)$ divides $f(x)$ in $F[x]$. On the other hand by Propsition \ref{prop:permutation} $f(x)$ divides $p(x)$ in $K[x]$ so $p(x) = f(x)$. In particular $p(x)$ is separable and all its roots lies in $K$.

  To show the first part, suppose $K/F$ is Galois and let $\{\omega_i\}_{i=1}^n$ be a basis for $K/F$. Let $p_i(x)$ be the minimal polynomial for $\omega_i$ over $F$, which is separable and has all its roots in $K$. Let $g(x)$ be the ``square free'' part of their products. Then the splitting field of the two polynomials is the same, which is $K$ (shown by demonstrating inclusion both ways). Thus $K$ is the splitting field of the separable polynomial $g(x)$.
\end{proof}

\begin{defi}
  Let $K/F$ be a Galois extension. If $\alpha \in F$ then the elements $\sigma\alpha$ for $\sigma \in \gal(K/F)$ are called the \emph{(Galois) conjugates} of $\alpha$ over $F$. If $F \subseteq E \subseteq K$ is a subfield then $\sigma(E)$ is the \emph{conjugate field} of $E$ over $F$.
\end{defi}

\begin{thm}[Fundamental Theorem of Galois Theory]
  Let $K/F$ be a Galois extension and $G := \gal(K/F)$. Then there is a bijection
  \[
    \left\{
      \begin{array}[h]{c}
        \text{subfields } \\
        E\: \text{of } K \\
        \text{containing } F
      \end{array}
      \begin{array}[h]{c}
        K \\
        | \\
        E \\
        | \\
        F
      \end{array}
    \right\}
    \longleftrightarrow
    \left\{
      \begin{array}[h]{c}
        \text{subgroups } \\
        H \: \text{of } G
      \end{array}
      \begin{array}[h]{c}
        1 \\
        | \\
        H \\
        | \\
        G
      \end{array}
      \right\}
  \]

  given by
  \begin{align*}
    E & \mapsto \{\sigma \in G: \sigma \: \text{fixes } E\} \\
    \{\text{fixed field of } H\} & \mapsfrom H
  \end{align*}
  which are inverses to each other, with the following properties:
  \begin{enumerate}
  \item order reversing,
  \item $[K:E] = |H|$ and $[E:F] = |G/H|$, the index of $H$ in $G$,
  \item $K/E$ is Galois with $\gal(K/E) \cong H$,
  \item $E$ is Galois over $F$ iff $H \trianglelefteq G$. In this case $\gal(E/F) \cong G/H$. More generally, even if $H$ is not necessarily normal in $G$, the isomorphisms of $E$ (into a fixed algebraic closure of $F$ containing $K$) which fix $F$ are in bijection with $G/H$, the set of cosets of $H$ in $G$.
  \item the lattice of subfields is dual to the lattice of subgroups: if $E_1, E_2$ correspond to $H_1, H_2$ respectively, then the intersection $E_1 \cap E_2$ corresponds to the group $\langle H_1, H_2 \rangle$ generated by $H_1$ and $H_2$ and the composite field $E_1E_2$ corresponds to the intersection $H_1 \cap H_2$.
  \end{enumerate}
\end{thm}

\begin{proof}
  Given any $H \leq G$ there is a unique fixed field by Corollary \ref{cor:unique fixed field} so the map is injective from right to left.

  If $K$ is the splitting field of a separable polynomial $f(x) \in F[x]$ then $K$ is also the splitting field of $f(x) \in E[x]$ so $K/E$ is Galois. By Corollary \ref{cor:upper bound of order of aut group}, $E$ is the fixed field of $\aut(K/E) \leq G$, showing that the map is surjective from right to left. By Corollary \ref{cor:upper bound of order of aut group} the maps are inverses to each other.

  By Theorem \ref{thm:degree of fixed field} $[K:E] = H$ and taking quotients gives $[E:F] = |G/H|$.

  Now we prove that $\emb(E/F)$, the set of embeddings of E into $\bar F$, the algebraic closure of $F$, which fixed $F$ is precisely $\{ \sigma|_E: \sigma \in G \}$. One direction of inclusion is clear. To prove the other inclusion, suppose $\tau : E \rightarrow \tau(E) \subseteq \bar F$ fixes $F$. We show $\tau(E) \subseteq K$: given $\alpha \in E$ with minimal polynomial $m_\alpha(x)$ over $F$, $\tau(\alpha)$ is another root of $m_\alpha(x)$ and $K$ contains all these roots by Theorem \ref{thm:characterisation of galois extension}. Since as above $K$ is the splitting field of $f(x)$ over $E$, it is also the splitting field of $\tau f(x)$. Thus we have the extension
  \[
    \begin{tikzcd}
      E \arrow{r}{\sim}[swap]{\tau} \arrow[hook]{d} & \tau(E) \arrow[hook]{d} \\
      K \arrow{r}{\sim}[swap]{\sigma} & K
    \end{tikzcd}
  \]
  and moreover $\sigma|_F = \tau|_F = \id$ so $\sigma \in G$. Thus every embedding of $E$ is of the form $\sigma|_E$ for some $\sigma \in G$.

  $\sigma, \sigma' \in G$ restrict to the same embedding of $E$ iff $\sigma^{-1} \sigma' |_E = \id$, so $\sigma^{-1} \sigma' \in H$. Thus there is a bijection between the embeddings and the cosets
  \[ \emb(E/F) \longleftrightarrow G/H. \]
  Note that $\aut(E/F) \subseteq \emb(E/F)$.

  The extension $E/F$ is Galois iff $|\aut(E/F)| = [E:F] = |\emb(E/F)|$, i.e. for every $\sigma \in G$, $\sigma(E) = E$. Now use the fact from group theory that conjugacy of stabiliser is the stabiliser of orbit, the subgroup fixing $\sigma(E)$ is precisely $\sigma H \sigma^{-1}$. Thus $\sigma(E) = E$ for all $\sigma \in G$ iff $\sigma H \sigma^{-1} = H$, i.e. $H \trianglelefteq G$. In this case the identification of the group structure of the cosets translates to the Galois group $\gal(E/F)$.

  Finally noting that elements in the composite field $E_1E_2$ are algebraic combinations of the elements of $E_1$ and $E_2$, the last part is easy to see.
\end{proof}

\begin{eg}[Splitting field of $x^8 - 2$]
  The splitting field of $x^8 - 2$ over $\mathbb{Q}$ is generated by $\theta = \sqrt[8] 2$ and a primitive $8$th root of unity $\zeta = \zeta_8$. Since $\mathbb{Q}(\zeta_8) = \mathbb{Q}(i, \sqrt 2)$ and $\theta^4 = \sqrt 2$ the splitting field is generated by $\theta$ and $i$. From
  \[ \mathbb{Q} \subseteq \mathbb{Q}(\theta) \subseteq \mathbb{Q}(\theta, i) = \mathbb{Q}(\theta, \zeta_8) \]
  we see the extension is of degree $16$. The Galois group is determined by its action on the generators $\theta$ and $i$ which gives the possibilities
  \[
    \begin{cases}
      \theta \mapsto \zeta^m \theta & m = 0, 1, \ldots, 7 \\
      i \mapsto \pm i
    \end{cases}
  \]
  Since the degree of the extionsion is $16$ and there are only $16$ such maps, it follows that each of the mpas above is an element of $\gal(\mathbb{Q}(\theta, i)/\mathbb{Q})$.

  Define two elements
  \[
    \sigma:
    \begin{cases}
      \theta \mapsto \zeta\theta \\
      i \mapsto i
    \end{cases}
    \qquad
    \tau:
    \begin{cases}
      \theta \mapsto \theta \\
      i \mapsto -i
    \end{cases}
  \]

  To facilitate computation we write
  \[ \zeta = \zeta_8 = \frac{1}{2}(1+i) \sqrt 2 = \frac{1}{2}(1+i) \theta^4 \]
  and from which we compute the following automorphisms:
  \begin{center}
    \begin{tabular}[h]{c c c c||c c c c}
       & $\theta$ & $i$ & $\zeta$ & & $\theta$ & $i$ & $\zeta$ \\
      \hline
      1 & $\theta$ & $i$ & $\zeta$ & $\tau$ & $\theta$ & $-i$ & $\zeta^7$ \\
      \hline
      $\sigma$ & $\zeta\theta$ & $i$ & $\zeta^5$ & $\tau\sigma$ & $\zeta^7 \theta$ & $-i$ & $\zeta^3$ \\
      \hline
      $\sigma^2$ & $\zeta^6\theta$ & $i$ & $\zeta$ & $\tau\sigma^2$ & $\zeta^2\sigma$ & $-i$ & $\theta^7$ \\
      \hline
      $\sigma^3$ & $\zeta^7\theta$ & $i$ & $-\zeta$ & $\tau\sigma^3$ & $\zeta\theta$ & $-i$ & $\zeta^3$ \\
      \hline
      $\sigma^4$ & $-\theta$ & $i$ & $\zeta$ & $\tau\sigma^4$ & $-\theta$ & $-i$ & $\zeta^7$ \\
      \hline
      $\sigma^5$ & $\zeta^5\theta$ & $i$ & $-\zeta$ & $\tau\sigma^5$ & $\zeta^3\theta$ & $-i$ & $\zeta^3$ \\
      \hline
      $\sigma^6$ & $\zeta^2\theta$ & $i$ & $\zeta$ & $\tau\sigma^6$ & $\zeta^6\theta$ & $-i$ & $\zeta^7$ \\
      \hline
      $\sigma^7$ & $\zeta^3\theta$ & $i$ & $-\zeta$ & $\tau\sigma^7$ & $\zeta^5\theta$ & $-i$ & $\zeta^3$ \\
      \hline
    \end{tabular}
  \end{center}

  By counting argument $\sigma$ and $\tau$ generate the Galois gropu. To determine the relation, first note that
  \[ \sigma^8 = \tau^2 = 1.\]
  Also we have
  \[ \sigma\tau = \tau\sigma^3.\]

  We can deduce that these relations define the group completely so
  \[ \gal(\mathbb{Q}(\sqrt[8] 2,i)/\mathbb{Q}) = \langle \sigma, \tau | \sigma^8 = \tau^2 = 1, \sigma\tau = \tau\sigma^3 \rangle\]
  which is a \emph{quasidihedral group} which is a subgroup of $S_8$.
\end{eg}

\begin{rmk}
  The example illustrates that one must take care in determining Galois groups from the actions on generators. Had we proceeded directly from the original generators $\theta = \sqrt[8] 2$ and $\zeta = \zeta_8$ we might have incorrectly concluded that there were a total of $32$ elements in the Galois group. The problem is that these choices are not independent, due to the relation
  \[ \theta^4 = \sqrt 2 = \zeta + \zeta^7 \]
  which shows that one cannot specify the images of $\theta$ and $\zeta$ independently.

  It is thus necessary to provide justification that maps are automorphisms. This can be accomplished, for example, by using the extension theorems or by using degree considerations as we did here.
\end{rmk}

\section{Finite Fields}

The field $\mathbb F_{p^n}$ is Glaois over $\mathbb F_p$, with cyclic Galois group of order $n$:
\[
  \gal(\mathbb F_{p^n} / \mathbb F_p) = \langle\sigma_p\rangle \cong \mathbb{Z}/n\mathbb{Z}
\]
where
\begin{align*}
  \sigma_p : \mathbb F_{p^n} & \rightarrow \mathbb F_{p^n} \\
  \alpha & \mapsto \alpha^p
\end{align*}

Since the Galois group is abelian, every subgroup is normal, so each of the subfields $\mathbb F_{p^d}$ where $d|n$ is Galois over $\mathbb F_p$.

\begin{prop}
  Any finite field is isomorphic to $\mathbb F_{p^n}$ for some prime $p$ and some integrer $n \geq 1$. The field $\mathbb F_{p^n}$ is the splitting field over $\mathbb F_p$ of the polynomial $x^{p^n}-x$, with cyclic Galois group of order $n$ generated by the Frobenius automorphism $\sigma_p$. The subfields of $\mathbb F_{p^n}$ are all Galois over $\mathbb F_p$ and are in bijection with the divisors $d$ of $n$. They are the fields of $\mathbb F_{p^d}$, the fixed fields of $\sigma^d_p$.
\end{prop}

\begin{cor}
  The irreducible polynomial $x^4+1 \in \mathbb{Z}[x]$ is reducible modulo every prime $p$.
\end{cor}

\begin{proof}
  Consider the polynomial $x^4+1 \in \mathbb F_p[x]$ for prime $p$. If $p = 2$ then $x^4 + 1 = (x+1)^4$ so reducible. Assume now that $p$ is odd. Then $p^2-1 \equiv 0 \mod 8$. Thus
  \[
    x^4+1 | x^8-1 | x^{p^2-1}-1 | x^{p^2}-x
  \]
  which shows that all the roots of $x^4+1$ are roots of $x^{p^2}-x$. Thus the extension generated by any root of $x^4+1$ is at most of degree $2$ over $\mathbb F_p$, so $x^4+1$ cannot be irreducible.
\end{proof}

The multiplicative group $\mathbb F_{p^n}^\times$ is finite so cyclic. Let $\theta$ be any generator, then $\mathbb F_{p^n} = \mathbb F_p(\theta)$. Thus

\begin{prop}
  $\mathbb F_{p^n}$ is simple. In particular, there exists an irreducivle polynomial of degree $n$ over $\mathbb F_p$ for every $n \geq 1$.
\end{prop}

Thus $\mathbb F_{p^n}$ can both be seen as the splitting field of $x^{p^n}-x$ over $\mathbb F_p$ and the quotient of $\mathbb F_p$ by the minimal polynomial of $\theta$. As $\theta$ is a root of $x^{p^n}-x$, the minimal polynomial is a divisor of $x^{p^n}-x$ of degree $n$.

Conversely, let $p(x)$ be irreducible of degree $d$ and divide $x^{p^n}-x$. If $\alpha$ is a root of $p(x)$ then $\mathbb F(\alpha)$ is a subfield of $\mathbb F_{p^n}$ of degree $d$. Thus $d|n$ and the extension if Galois by the previous proposition so all the roots of $p(x)$ are contained in $\mathbb F_p(\alpha)$. Thus if we gather all factors of $x^{p^n}-x$ in $\mathbb F_{p^n}$ according to the degree $d$ of their minimal polynomails ofver $\mathbb F_p$, then

\begin{prop}
  The polynomial $x^{p^n}-x$ is precisely the product of all the distinct irreducible polynomials in $\mathbb F_p[x]$ of degree $d$ where $d|n$.
\end{prop}

\begin{eg}
  The finite field $\mathbb F_{p^n}$ is unique up to isomorphism. If $f_1(x)$ and $f_2(x)$ are irreducible of degree $n$, then $f_2(x)$ splits completely in the field $\mathbb F_{p^n} \cong \mathbb F_p[x]/(f_1(x))$. If we denote a root of $f_2(x)$ by $\alpha(x)$, the isomorphism is given by
  \begin{align*}
    \mathbb F_p[x]/(f_2(x)) & \cong \mathbb F_p[x]/(f_1(x)) \\
    x & \mapsto \alpha(x)
  \end{align*}

  For example, if $f_1(x) = x^4+x^3+1, f_2(x) = x^4+x+1$ are two irreducible over $\mathbb F_2$ then $x \mapsto \alpha(x) = x^3+x^2$ gives an isomorphism.
\end{eg}

We can give a formula for the number of irreducible polynomials of degree $n$ using result from elementary number theory.

\begin{defi}
  The \emph{M\"obius $\mu$-function} is definded by
  \[
    \mu (n) =  
  \begin{cases}
    1 & \text{for } n = 1 \\
    0 & \text{if $n$ has a square factor} \\
    (-1)^r & \text{if $n$ has $r$ distinct prime factors}
  \end{cases}
  \]
\end{defi}

Given $f(n)$ a function defined for all nonnegative integers $n$ and
\[
  F(n) := \sum_{d|n} f(d)
\]

then the \emph{M\"obius inversion formula} states that
\[
  f(n) = \sum_{d|n} \mu(d) F(\frac{n}{d}).
\]

Now define $\psi(n)$ to be the number of irreducible polynomials of degree $n$ in $\mathbb F_p[x]$, we have
\[
  p^n = \sum_{d|n} d\psi(d)
\]

Apply the inversion formula we get
\[
  n \psi(n) = \sum_{d|n} \mu(d) p^{n/d}
\]
so
\[
  \psi(n) = \frac{1}{n} \sum_{d|n} \mu(d) p^{n/d}.
\]

Given two finite fields $\mathbb F_{p^{n_1}}, \mathbb F_{p^{n_2}}$, $\mathbb F_{p^{n_1n_2}}$ contains both of them. This gives a partial ordering on these fields and take the union of all these finite extensions we get the algebraic closure of $\mathbb F_p$
\[
  \cl{\mathbb F_p} = \bigcup_{n\geq 1} \mathbb F_{p^n}.
\]

\iffalse
\appendix

\section{From Dexter Chua Notes}

\begin{lem}
  Given $L/K$, $f(t) \in K[t]$ irreducible, there is a bijection between
  \[ \{ \text{roots of } f \: \text{in } L \} \longleftrightarrow \Hom_K(K[t]/(f(t)), L) \]
\end{lem}
\fi
\end{document}